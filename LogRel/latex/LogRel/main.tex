\documentclass[submission,copyright,creativecommons]{eptcs}
\providecommand{\event}{Peter Thiemann's Festschrift 2024} % Name of the event you are submitting to

\usepackage{iftex}

\ifpdf
  \usepackage{underscore}         % Only needed if you use pdflatex.
  \usepackage[T1]{fontenc}        % Recommended with pdflatex
\else
  \usepackage{breakurl}           % Not needed if you use pdflatex only.
\fi

\usepackage{fancyvrb}
\usepackage{agda}
%\usepackage{mathabx}
\usepackage{amssymb}
\usepackage{stmaryrd}
\usepackage[numbers]{natbib}
%\usepackage{nath}

%% \usepackage{ucs}
%% \usepackage[utf8x]{inputenc}
%% \usepackage{autofe}
\usepackage{newunicodechar}
\newunicodechar{∷}{{::}}
\newunicodechar{⊤}{\ensuremath{\top}}
\newunicodechar{⊥}{\ensuremath{\bot}}
\newunicodechar{₁}{\ensuremath{_1}}
\newunicodechar{₂}{\ensuremath{_2}}
\newunicodechar{₃}{\ensuremath{_3}}

\newunicodechar{β}{\ensuremath{\beta}}
\newunicodechar{γ}{\ensuremath{\gamma}}
\newunicodechar{λ}{\ensuremath{\lambda}}
\newunicodechar{Λ}{\ensuremath{\Lambda}}
\newunicodechar{μ}{\ensuremath{\mu}}
\newunicodechar{ν}{\ensuremath{\nu}}
\newunicodechar{ρ}{\ensuremath{\rho}}
\newunicodechar{σ}{\ensuremath{\sigma}}
\newunicodechar{Π}{\ensuremath{\Pi}}
\newunicodechar{ξ}{\ensuremath{\xi}}
\newunicodechar{Γ}{\ensuremath{\Gamma}}
\newunicodechar{Δ}{\ensuremath{\Delta}}
\newunicodechar{Σ}{\ensuremath{\Sigma}}

\newunicodechar{∀}{\ensuremath{\forall}}
\newunicodechar{∃}{\ensuremath{\exists}}
\newunicodechar{⊎}{\ensuremath{\uplus}}
\newunicodechar{ᵖ}{\ensuremath{^p}}
\newunicodechar{≡}{\ensuremath{\equiv}}
\newunicodechar{≢}{\ensuremath{\not\equiv}}
\newunicodechar{⊑}{\ensuremath{\sqsubseteq}}
\newunicodechar{★}{\ensuremath{\star}}
\newunicodechar{∼}{\ensuremath{\sim}}
\newunicodechar{⌈}{\ensuremath{\lceil}}
\newunicodechar{⌉}{\ensuremath{\rceil}}
\newunicodechar{·}{\ensuremath{\cdot}}
\newunicodechar{•}{\ensuremath{\bullet}}

\newunicodechar{⇒}{\ensuremath{\Rightarrow}}
\newunicodechar{↠}{\ensuremath{\longrightarrow^{*}}}
\newunicodechar{⇓}{\ensuremath{\Downarrow}}
\newunicodechar{⇑}{\ensuremath{\Uparrow}}
\newunicodechar{⟶}{\ensuremath{\longrightarrow}}

\newunicodechar{ι}{\ensuremath{\iota}}
\newunicodechar{ₜ}{\ensuremath{_t}}
\newunicodechar{⊢}{\ensuremath{\vdash}}
\newunicodechar{⊩}{\ensuremath{\Vdash}}
\newunicodechar{⊨}{\ensuremath{\vDash}}
\newunicodechar{⦂}{\ensuremath{\mathop{:}}}
\newunicodechar{∈}{\ensuremath{\in}}
\newunicodechar{∋}{\ensuremath{\ni}}
\newunicodechar{′}{\ensuremath{'}}
\newunicodechar{″}{\ensuremath{''}}
\newunicodechar{ƛ}{\ensuremath{\lambda}}
\newunicodechar{≼}{\ensuremath{\preceq}}
\newunicodechar{≽}{\ensuremath{\succeq}}
\newunicodechar{ᵥ}{\ensuremath{_v}}
\newunicodechar{ˢ}{\ensuremath{^s}}
\newunicodechar{∅}{\ensuremath{\emptyset}}
\newunicodechar{∣}{\ensuremath{|}}
\newunicodechar{ᴸ}{\ensuremath{^L}}
\newunicodechar{ᴿ}{\ensuremath{^R}}
\newunicodechar{▷}{\ensuremath{\triangleright}}
\newunicodechar{ᵍ}{\ensuremath{^g}}
\newunicodechar{ᵒ}{\ensuremath{^{\circ}}}
\newunicodechar{⩦}{\ensuremath{\equiv}}
\newunicodechar{∎}{\ensuremath{\blacksquare}}
\newunicodechar{□}{\ensuremath{\square}}
\newunicodechar{■}{\ensuremath{\blacksquare}}
\newunicodechar{⇔}{\ensuremath{\iff}}
\newunicodechar{𝒱}{\ensuremath{\mathcal{V}}}
\newunicodechar{̬}{}
\newunicodechar{𝒫}{\ensuremath{\mathcal{P}}}
\newunicodechar{ℰ}{\ensuremath{\mathcal{E}}}
\newunicodechar{≤}{\ensuremath{\leq}}

\newunicodechar{ℕ}{\ensuremath{\mathbb{N}}}
\newunicodechar{𝔹}{\ensuremath{\mathbb{B}}}

\newunicodechar{⦉}{\ensuremath{\llfloor}}
\newunicodechar{⦊}{\ensuremath{\rrfloor}}
\newunicodechar{⟦}{\ensuremath{[}}
\newunicodechar{⟧}{\ensuremath{]}}
\newunicodechar{⦅}{\ensuremath{\llparenthesis}}
\newunicodechar{⦆}{\ensuremath{\rrparenthesis}}
\newunicodechar{⟪}{\ensuremath{\langle\!\!\langle}}
\newunicodechar{⟫}{\ensuremath{\rangle\!\!\rangle}}
\newunicodechar{⟨}{\ensuremath{\langle}}
\newunicodechar{⟩}{\ensuremath{\rangle}}
\newunicodechar{∶}{:}
\newunicodechar{⨟}{\ensuremath{\fatsemi}}
\newunicodechar{↑}{\ensuremath{\uparrow}}
\newunicodechar{≠}{\ensuremath{\neq}}

%% \DefineVerbatimEnvironment
%%   {code}{Verbatim}
%%   {} % Add fancy options here if you like.

\title{Gradual Guarantee via \\ Step-Indexed Logical Relations in Agda}
\author{Jeremy G. Siek
\institute{School of Informatics, Computing, and Engineering\\
Indiana University \\
Bloomington, IN, USA}
\email{jsiek@iu.edu}
%% \and
%% Co Author \qquad\qquad Yet S. Else
%% \institute{Stanford University\\
%% California, USA}
%% \email{\quad is@gmail.com \quad\qquad somebody@else.org}
}
\def\titlerunning{Gradual Guarantee via Step-Indexed Logical Relations in Agda}
\def\authorrunning{Jeremy G. Siek}
\begin{document}
\maketitle

\begin{abstract}
  The gradual guarantee is an important litmus test for gradually
  typed languages, that is, languages that enable a mixture of static
  and dynamic typing. The gradual guarantee states that changing the
  precision of a type annotation does not change the behavior of the
  program, except perhaps to trigger an error if the type annotation
  is incorrect. Siek et al. (2015) proved that the Gradually Typed
  Lambda Calculus (GTLC) satisfies the gradual guarantee using a
  simulation-based proof and mechanized their proof in Isabelle. In
  the following decade, researchers have proved the gradual guarantee
  for more sophisticated calculi, using step-indexed logical
  relations.  However, given the complexity of that style of proof,
  there has not yet been a mechanized proof of the gradual guarantee
  using step-indexed logical relations. This paper reports on a
  mechanized proof of the gradual guarantee for the GTLC carried out
  in the Agda proof assistant.
\end{abstract}

\section{Introduction}

Gradually typed languages introduce the unknown type ★ as a way for
programmers to control the amount of type precision, and compile-time
type checking, in their program \cite{Siek:2006bh,Siek:2007qy}. If all
type annotations are just ★, then the program behaves like a
dynamically typed language. At the other end of the spectrum, if no
type annotation contains ★, then the program behaves like a statically
typed language.

Siek et al.~\cite{Siek:2015ac} introduce the \emph{gradual guarantee}
as a litmus test for gradually typed languages.  This property says
that the behavior of a program should not change (except for errors)
when the programmer changes type annotations to be more or less
precise.  Siek et al.~\cite{Siek:2015ac} prove that the Gradually
Typed Lambda Calculus (GTLC) satisfies the gradual guarantee using a
simulation-based proof and mechanize the result in the Isabelle proof
assistant~\cite{Nipkow:2002jl}. Using logical relations, New and
Ahmed~\cite{New:2018aa} prove the gradual guarantee for the GTLC and
New et al.~\cite{New:2019ab} prove the gradual guarantee for a
polymorphic calculus. Several researchers apply logical relations to
prove other properties of gradually typed languages, such as
noninterference\cite{Toro:2018aa},
parametricity~\cite{Ahmed:2011fk,New:2019ab,Labrada:2020tk}, and fully
abstract embedding~\cite{Jacobs:2021aa}. Of this later work, only the
abstract embedding was mechanized (in
Coq~\cite{The-Coq-Development-Team:2004kf} using the Iris
framework~\cite{JUNG:2018aa}.)

There are several technical challenges to overcome in developing a
mechanized proof of the gradual guarantee using step-indexed logical
relations. As in any programming language mechanization, one must
choose how to represent variables and perform substitution. Moreover,
proofs based on logical relations rely on the fact that substitutions
commute, and the proof of this standard result is quite technical and
lengthy. Repeating these proofs for each new programming language is
quite tedious, but this metatheory can be developed in a
language-independent way using the notion of \emph{abstract binding
  trees}~\cite{Harper:2012aa}. To this end we developed the Abstract
Binding Tree library in Agda~\cite{Siek:2021to}, representing
variables as de Bruijn indices and implementing substitution via
parallel renaming and
substitution~\cite{McBride:2005aa,Wadler:2020aa}.

The second technical challenge is that step-indexed logical relations
``involve tedious, error-prone, and proof-obscuring step-index
arithmetic''~\cite{Dreyer:2011wl}. Dreyer, Ahmed, and
Birkedal~\cite{Dreyer:2011wl} propose to abstract over the
step-indexing using a modal logic named LSLR. Dreyer and Birkedal,
with many colleagues, implemented this logic as part of the Iris
framework~\cite{JUNG:2018aa} in the Coq proof assistant. To make a
similar modal logic available in Agda, we developed the Step-Indexed
Logic (SIL)~\cite{Siek:2023aa}.  The proof of the gradual guarantee in
this paper is the first application of SIL and we report on the
experience.

The semantics of the GTLC is defined by translation to a Cast
Calculus, so we present the Cast Calculus in
Section~\ref{sec:cast-calculus}. We define the precision relation on
types and terms in Section~\ref{sec:precision} and we review
Step-Indexed Logic in Section~\ref{sec:SIL}.  We define a logical
relation for precision in Section~\ref{sec:log-rel}. We prove the
Fundamental Theorem of the logical relation in
Section~\ref{sec:fundamental}. To finish the proof of the gradual
guarantee, in Section~\ref{sec:gradual-guarantee} we prove that the
logical relation implies the gradual
guarantee. Section~\ref{sec:conclusion} concludes this paper with a
comparison of using logical-relations versus simulation to prove the
gradual guarantee and it acknowledges Peter Thiemann and Philip Wadler
for their contributions to this work.


\input{PeterCastCalculus}
\begin{code}[hide]%
\>[0]\AgdaSymbol{\{-\#}\AgdaSpace{}%
\AgdaKeyword{OPTIONS}\AgdaSpace{}%
\AgdaPragma{--rewriting}\AgdaSpace{}%
\AgdaSymbol{\#-\}}\<%
\\
\>[0]\AgdaKeyword{module}\AgdaSpace{}%
\AgdaModule{LogRel.PeterFestschrift}\AgdaSpace{}%
\AgdaKeyword{where}\<%
\\
%
\\[\AgdaEmptyExtraSkip]%
\>[0]\AgdaKeyword{open}\AgdaSpace{}%
\AgdaKeyword{import}\AgdaSpace{}%
\AgdaModule{Data.Empty}\AgdaSpace{}%
\AgdaKeyword{using}\AgdaSpace{}%
\AgdaSymbol{(}\AgdaDatatype{⊥}\AgdaSymbol{;}\AgdaSpace{}%
\AgdaFunction{⊥-elim}\AgdaSymbol{)}\<%
\\
\>[0]\AgdaKeyword{open}\AgdaSpace{}%
\AgdaKeyword{import}\AgdaSpace{}%
\AgdaModule{Data.List}\AgdaSpace{}%
\AgdaKeyword{using}\AgdaSpace{}%
\AgdaSymbol{(}\AgdaDatatype{List}\AgdaSymbol{;}\AgdaSpace{}%
\AgdaInductiveConstructor{[]}\AgdaSymbol{;}\AgdaSpace{}%
\AgdaOperator{\AgdaInductiveConstructor{\AgdaUnderscore{}∷\AgdaUnderscore{}}}\AgdaSymbol{;}\AgdaSpace{}%
\AgdaFunction{map}\AgdaSymbol{;}\AgdaSpace{}%
\AgdaFunction{length}\AgdaSymbol{)}\<%
\\
\>[0]\AgdaKeyword{open}\AgdaSpace{}%
\AgdaKeyword{import}\AgdaSpace{}%
\AgdaModule{Data.Nat}\<%
\\
\>[0]\AgdaKeyword{open}\AgdaSpace{}%
\AgdaKeyword{import}\AgdaSpace{}%
\AgdaModule{Data.Nat.Properties}\<%
\\
\>[0]\AgdaKeyword{open}\AgdaSpace{}%
\AgdaKeyword{import}\AgdaSpace{}%
\AgdaModule{Data.Bool}\AgdaSpace{}%
\AgdaKeyword{using}\AgdaSpace{}%
\AgdaSymbol{(}\AgdaInductiveConstructor{true}\AgdaSymbol{;}\AgdaSpace{}%
\AgdaInductiveConstructor{false}\AgdaSymbol{)}\AgdaSpace{}%
\AgdaKeyword{renaming}\AgdaSpace{}%
\AgdaSymbol{(}\AgdaDatatype{Bool}\AgdaSpace{}%
\AgdaSymbol{to}\AgdaSpace{}%
\AgdaDatatype{𝔹}\AgdaSymbol{)}\<%
\\
\>[0]\AgdaKeyword{open}\AgdaSpace{}%
\AgdaKeyword{import}\AgdaSpace{}%
\AgdaModule{Data.Product}\AgdaSpace{}%
\AgdaKeyword{using}\AgdaSpace{}%
\AgdaSymbol{(}\AgdaOperator{\AgdaInductiveConstructor{\AgdaUnderscore{},\AgdaUnderscore{}}}\AgdaSymbol{;}\AgdaOperator{\AgdaFunction{\AgdaUnderscore{}×\AgdaUnderscore{}}}\AgdaSymbol{;}\AgdaSpace{}%
\AgdaField{proj₁}\AgdaSymbol{;}\AgdaSpace{}%
\AgdaField{proj₂}\AgdaSymbol{;}\AgdaSpace{}%
\AgdaFunction{Σ-syntax}\AgdaSymbol{;}\AgdaSpace{}%
\AgdaFunction{∃-syntax}\AgdaSymbol{)}\<%
\\
\>[0]\AgdaKeyword{open}\AgdaSpace{}%
\AgdaKeyword{import}\AgdaSpace{}%
\AgdaModule{Data.Sum}\AgdaSpace{}%
\AgdaKeyword{using}\AgdaSpace{}%
\AgdaSymbol{(}\AgdaOperator{\AgdaDatatype{\AgdaUnderscore{}⊎\AgdaUnderscore{}}}\AgdaSymbol{;}\AgdaSpace{}%
\AgdaInductiveConstructor{inj₁}\AgdaSymbol{;}\AgdaSpace{}%
\AgdaInductiveConstructor{inj₂}\AgdaSymbol{)}\<%
\\
\>[0]\AgdaKeyword{open}\AgdaSpace{}%
\AgdaKeyword{import}\AgdaSpace{}%
\AgdaModule{Data.Unit}\AgdaSpace{}%
\AgdaKeyword{using}\AgdaSpace{}%
\AgdaSymbol{(}\AgdaRecord{⊤}\AgdaSymbol{;}\AgdaSpace{}%
\AgdaInductiveConstructor{tt}\AgdaSymbol{)}\<%
\\
\>[0]\AgdaKeyword{open}\AgdaSpace{}%
\AgdaKeyword{import}\AgdaSpace{}%
\AgdaModule{Data.Unit.Polymorphic}\AgdaSpace{}%
\AgdaKeyword{renaming}\AgdaSpace{}%
\AgdaSymbol{(}\AgdaFunction{⊤}\AgdaSpace{}%
\AgdaSymbol{to}\AgdaSpace{}%
\AgdaFunction{topᵖ}\AgdaSymbol{;}\AgdaSpace{}%
\AgdaFunction{tt}\AgdaSpace{}%
\AgdaSymbol{to}\AgdaSpace{}%
\AgdaFunction{ttᵖ}\AgdaSymbol{)}\<%
\\
\>[0]\AgdaKeyword{open}\AgdaSpace{}%
\AgdaKeyword{import}\AgdaSpace{}%
\AgdaModule{Relation.Binary.PropositionalEquality}\AgdaSpace{}%
\AgdaSymbol{as}\AgdaSpace{}%
\AgdaModule{Eq}\<%
\\
\>[0][@{}l@{\AgdaIndent{0}}]%
\>[2]\AgdaKeyword{using}\AgdaSpace{}%
\AgdaSymbol{(}\AgdaOperator{\AgdaDatatype{\AgdaUnderscore{}≡\AgdaUnderscore{}}}\AgdaSymbol{;}\AgdaSpace{}%
\AgdaOperator{\AgdaFunction{\AgdaUnderscore{}≢\AgdaUnderscore{}}}\AgdaSymbol{;}\AgdaSpace{}%
\AgdaInductiveConstructor{refl}\AgdaSymbol{;}\AgdaSpace{}%
\AgdaFunction{sym}\AgdaSymbol{;}\AgdaSpace{}%
\AgdaFunction{cong}\AgdaSymbol{;}\AgdaSpace{}%
\AgdaFunction{subst}\AgdaSymbol{;}\AgdaSpace{}%
\AgdaFunction{trans}\AgdaSymbol{)}\<%
\\
\>[0]\AgdaKeyword{open}\AgdaSpace{}%
\AgdaKeyword{import}\AgdaSpace{}%
\AgdaModule{Relation.Nullary}\AgdaSpace{}%
\AgdaKeyword{using}\AgdaSpace{}%
\AgdaSymbol{(}\AgdaOperator{\AgdaFunction{¬\AgdaUnderscore{}}}\AgdaSymbol{;}\AgdaSpace{}%
\AgdaRecord{Dec}\AgdaSymbol{;}\AgdaSpace{}%
\AgdaInductiveConstructor{yes}\AgdaSymbol{;}\AgdaSpace{}%
\AgdaInductiveConstructor{no}\AgdaSymbol{)}\<%
\\
%
\\[\AgdaEmptyExtraSkip]%
\>[0]\AgdaKeyword{open}\AgdaSpace{}%
\AgdaKeyword{import}\AgdaSpace{}%
\AgdaModule{Var}\<%
\\
\>[0]\AgdaKeyword{open}\AgdaSpace{}%
\AgdaKeyword{import}\AgdaSpace{}%
\AgdaModule{Sig}\<%
\\
\>[0]\AgdaKeyword{open}\AgdaSpace{}%
\AgdaKeyword{import}\AgdaSpace{}%
\AgdaModule{LogRel.PeterCastCalculus}\<%
\\
\>[0]\AgdaKeyword{open}\AgdaSpace{}%
\AgdaKeyword{import}\AgdaSpace{}%
\AgdaModule{StepIndexedLogic}\<%
\end{code}

\section{Precision on Types and Terms}
\label{sec:precision}

To talk about the gradual guarantee, we first define when one type is
less precise than another one. The following definition says that the
unknown type ★ is less precise than any other type.

\begin{code}%
\>[0]\AgdaKeyword{infixr}\AgdaSpace{}%
\AgdaNumber{6}\AgdaSpace{}%
\AgdaOperator{\AgdaDatatype{\AgdaUnderscore{}⊑\AgdaUnderscore{}}}\<%
\\
\>[0]\AgdaKeyword{data}\AgdaSpace{}%
\AgdaOperator{\AgdaDatatype{\AgdaUnderscore{}⊑\AgdaUnderscore{}}}\AgdaSpace{}%
\AgdaSymbol{:}\AgdaSpace{}%
\AgdaDatatype{Type}\AgdaSpace{}%
\AgdaSymbol{→}\AgdaSpace{}%
\AgdaDatatype{Type}\AgdaSpace{}%
\AgdaSymbol{→}\AgdaSpace{}%
\AgdaPrimitive{Set}\AgdaSpace{}%
\AgdaKeyword{where}\<%
\\
\>[0][@{}l@{\AgdaIndent{0}}]%
\>[2]\AgdaInductiveConstructor{unk⊑unk}\AgdaSpace{}%
\AgdaSymbol{:}\AgdaSpace{}%
\AgdaInductiveConstructor{★}\AgdaSpace{}%
\AgdaOperator{\AgdaDatatype{⊑}}\AgdaSpace{}%
\AgdaInductiveConstructor{★}\<%
\\
%
\>[2]\AgdaInductiveConstructor{unk⊑}\AgdaSpace{}%
\AgdaSymbol{:}\AgdaSpace{}%
\AgdaSymbol{∀\{}\AgdaBound{G}\AgdaSymbol{\}\{}\AgdaBound{B}\AgdaSymbol{\}}\AgdaSpace{}%
\AgdaSymbol{→}\AgdaSpace{}%
\AgdaOperator{\AgdaFunction{⌈}}\AgdaSpace{}%
\AgdaBound{G}\AgdaSpace{}%
\AgdaOperator{\AgdaFunction{⌉}}\AgdaSpace{}%
\AgdaOperator{\AgdaDatatype{⊑}}\AgdaSpace{}%
\AgdaBound{B}\AgdaSpace{}%
\AgdaSymbol{→}\AgdaSpace{}%
\AgdaInductiveConstructor{★}\AgdaSpace{}%
\AgdaOperator{\AgdaDatatype{⊑}}\AgdaSpace{}%
\AgdaBound{B}\<%
\\
%
\>[2]\AgdaInductiveConstructor{base⊑}\AgdaSpace{}%
\AgdaSymbol{:}\AgdaSpace{}%
\AgdaSymbol{∀\{}\AgdaBound{ι}\AgdaSymbol{\}}\AgdaSpace{}%
\AgdaSymbol{→}\AgdaSpace{}%
\AgdaOperator{\AgdaInductiveConstructor{\$ₜ}}\AgdaSpace{}%
\AgdaBound{ι}\AgdaSpace{}%
\AgdaOperator{\AgdaDatatype{⊑}}\AgdaSpace{}%
\AgdaOperator{\AgdaInductiveConstructor{\$ₜ}}\AgdaSpace{}%
\AgdaBound{ι}\<%
\\
%
\>[2]\AgdaInductiveConstructor{fun⊑}\AgdaSpace{}%
\AgdaSymbol{:}\AgdaSpace{}%
\AgdaSymbol{∀\{}\AgdaBound{A}\AgdaSpace{}%
\AgdaBound{B}\AgdaSpace{}%
\AgdaBound{C}\AgdaSpace{}%
\AgdaBound{D}\AgdaSymbol{\}}%
\>[21]\AgdaSymbol{→}%
\>[24]\AgdaBound{A}\AgdaSpace{}%
\AgdaOperator{\AgdaDatatype{⊑}}\AgdaSpace{}%
\AgdaBound{C}%
\>[31]\AgdaSymbol{→}%
\>[34]\AgdaBound{B}\AgdaSpace{}%
\AgdaOperator{\AgdaDatatype{⊑}}\AgdaSpace{}%
\AgdaBound{D}%
\>[41]\AgdaSymbol{→}%
\>[44]\AgdaBound{A}\AgdaSpace{}%
\AgdaOperator{\AgdaInductiveConstructor{⇒}}\AgdaSpace{}%
\AgdaBound{B}\AgdaSpace{}%
\AgdaOperator{\AgdaDatatype{⊑}}\AgdaSpace{}%
\AgdaBound{C}\AgdaSpace{}%
\AgdaOperator{\AgdaInductiveConstructor{⇒}}\AgdaSpace{}%
\AgdaBound{D}\<%
\end{code}

The first two rules for precision are usually presented as a single
rule that says ★ is less precise than any type.  Instead we have
separated out the case for when both types are ★ from the case when
only the less-precise type is ★.  Also, for the rule \textsf{unk⊑},
instead of writing $B ≢ ★$ we have written $⌈ G ⌉ ⊑ B$, which turns
out to be important later when we define the logical relation and use
recursion on the precision relation.
%
The \textsf{Prec} type bundles two types in the precision relation and
of course, precision is reflexive.

\begin{code}%
\>[0]\AgdaFunction{Prec}\AgdaSpace{}%
\AgdaSymbol{:}\AgdaSpace{}%
\AgdaPrimitive{Set}\<%
\\
\>[0]\AgdaFunction{Prec}\AgdaSpace{}%
\AgdaSymbol{=}\AgdaSpace{}%
\AgdaSymbol{(}\AgdaFunction{∃[}\AgdaSpace{}%
\AgdaBound{A}\AgdaSpace{}%
\AgdaFunction{]}\AgdaSpace{}%
\AgdaFunction{∃[}\AgdaSpace{}%
\AgdaBound{B}\AgdaSpace{}%
\AgdaFunction{]}\AgdaSpace{}%
\AgdaBound{A}\AgdaSpace{}%
\AgdaOperator{\AgdaDatatype{⊑}}\AgdaSpace{}%
\AgdaBound{B}\AgdaSymbol{)}\<%
\\
%
\\[\AgdaEmptyExtraSkip]%
\>[0]\AgdaFunction{Refl⊑}\AgdaSpace{}%
\AgdaSymbol{:}\AgdaSpace{}%
\AgdaSymbol{∀\{}\AgdaBound{A}\AgdaSymbol{\}}\AgdaSpace{}%
\AgdaSymbol{→}\AgdaSpace{}%
\AgdaBound{A}\AgdaSpace{}%
\AgdaOperator{\AgdaDatatype{⊑}}\AgdaSpace{}%
\AgdaBound{A}\<%
\end{code}
\begin{code}[hide]%
\>[0]\AgdaFunction{Refl⊑}\AgdaSpace{}%
\AgdaSymbol{\{}\AgdaInductiveConstructor{★}\AgdaSymbol{\}}\AgdaSpace{}%
\AgdaSymbol{=}\AgdaSpace{}%
\AgdaInductiveConstructor{unk⊑unk}\<%
\\
\>[0]\AgdaFunction{Refl⊑}\AgdaSpace{}%
\AgdaSymbol{\{}\AgdaOperator{\AgdaInductiveConstructor{\$ₜ}}\AgdaSpace{}%
\AgdaBound{ι}\AgdaSymbol{\}}\AgdaSpace{}%
\AgdaSymbol{=}\AgdaSpace{}%
\AgdaInductiveConstructor{base⊑}\<%
\\
\>[0]\AgdaFunction{Refl⊑}\AgdaSpace{}%
\AgdaSymbol{\{}\AgdaBound{A}\AgdaSpace{}%
\AgdaOperator{\AgdaInductiveConstructor{⇒}}\AgdaSpace{}%
\AgdaBound{B}\AgdaSymbol{\}}\AgdaSpace{}%
\AgdaSymbol{=}\AgdaSpace{}%
\AgdaInductiveConstructor{fun⊑}\AgdaSpace{}%
\AgdaFunction{Refl⊑}\AgdaSpace{}%
\AgdaFunction{Refl⊑}\<%
\end{code}

\begin{code}[hide]%
\>[0]\AgdaFunction{unk⊑gnd-inv}\AgdaSpace{}%
\AgdaSymbol{:}\AgdaSpace{}%
\AgdaSymbol{∀\{}\AgdaBound{G}\AgdaSymbol{\}}\<%
\\
\>[0][@{}l@{\AgdaIndent{0}}]%
\>[3]\AgdaSymbol{→}\AgdaSpace{}%
\AgdaSymbol{(}\AgdaBound{c}\AgdaSpace{}%
\AgdaSymbol{:}\AgdaSpace{}%
\AgdaInductiveConstructor{★}\AgdaSpace{}%
\AgdaOperator{\AgdaDatatype{⊑}}\AgdaSpace{}%
\AgdaOperator{\AgdaFunction{⌈}}\AgdaSpace{}%
\AgdaBound{G}\AgdaSpace{}%
\AgdaOperator{\AgdaFunction{⌉}}\AgdaSymbol{)}\<%
\\
%
\>[3]\AgdaSymbol{→}\AgdaSpace{}%
\AgdaFunction{∃[}\AgdaSpace{}%
\AgdaBound{d}\AgdaSpace{}%
\AgdaFunction{]}\AgdaSpace{}%
\AgdaBound{c}\AgdaSpace{}%
\AgdaOperator{\AgdaDatatype{≡}}\AgdaSpace{}%
\AgdaInductiveConstructor{unk⊑}\AgdaSymbol{\{}\AgdaBound{G}\AgdaSymbol{\}\{}\AgdaOperator{\AgdaFunction{⌈}}\AgdaSpace{}%
\AgdaBound{G}\AgdaSpace{}%
\AgdaOperator{\AgdaFunction{⌉}}\AgdaSymbol{\}}\AgdaSpace{}%
\AgdaBound{d}\<%
\\
\>[0]\AgdaFunction{unk⊑gnd-inv}\AgdaSpace{}%
\AgdaSymbol{\{}\AgdaOperator{\AgdaInductiveConstructor{\$ᵍ}}\AgdaSpace{}%
\AgdaBound{ι}\AgdaSymbol{\}}\AgdaSpace{}%
\AgdaSymbol{(}\AgdaInductiveConstructor{unk⊑}\AgdaSpace{}%
\AgdaSymbol{\{}\AgdaOperator{\AgdaInductiveConstructor{\$ᵍ}}\AgdaSpace{}%
\AgdaDottedPattern{\AgdaSymbol{.}}\AgdaDottedPattern{\AgdaBound{ι}}\AgdaSymbol{\}}\AgdaSpace{}%
\AgdaInductiveConstructor{base⊑}\AgdaSymbol{)}\AgdaSpace{}%
\AgdaSymbol{=}\AgdaSpace{}%
\AgdaInductiveConstructor{base⊑}\AgdaSpace{}%
\AgdaOperator{\AgdaInductiveConstructor{,}}\AgdaSpace{}%
\AgdaInductiveConstructor{refl}\<%
\\
\>[0]\AgdaFunction{unk⊑gnd-inv}\AgdaSpace{}%
\AgdaSymbol{\{}\AgdaInductiveConstructor{★⇒★}\AgdaSymbol{\}}\AgdaSpace{}%
\AgdaSymbol{(}\AgdaInductiveConstructor{unk⊑}\AgdaSpace{}%
\AgdaSymbol{\{}\AgdaInductiveConstructor{★⇒★}\AgdaSymbol{\}}\AgdaSpace{}%
\AgdaSymbol{(}\AgdaInductiveConstructor{fun⊑}\AgdaSpace{}%
\AgdaBound{c}\AgdaSpace{}%
\AgdaBound{d}\AgdaSymbol{))}\AgdaSpace{}%
\AgdaSymbol{=}\AgdaSpace{}%
\AgdaInductiveConstructor{fun⊑}\AgdaSpace{}%
\AgdaBound{c}\AgdaSpace{}%
\AgdaBound{d}\AgdaSpace{}%
\AgdaOperator{\AgdaInductiveConstructor{,}}\AgdaSpace{}%
\AgdaInductiveConstructor{refl}\<%
\\
%
\\[\AgdaEmptyExtraSkip]%
\>[0]\AgdaFunction{dyn-prec-unique}\AgdaSpace{}%
\AgdaSymbol{:}\AgdaSpace{}%
\AgdaSymbol{∀\{}\AgdaBound{A}\AgdaSymbol{\}}\<%
\\
\>[0][@{}l@{\AgdaIndent{0}}]%
\>[2]\AgdaSymbol{→}\AgdaSpace{}%
\AgdaSymbol{(}\AgdaBound{c}\AgdaSpace{}%
\AgdaSymbol{:}\AgdaSpace{}%
\AgdaInductiveConstructor{★}\AgdaSpace{}%
\AgdaOperator{\AgdaDatatype{⊑}}\AgdaSpace{}%
\AgdaBound{A}\AgdaSymbol{)}\<%
\\
%
\>[2]\AgdaSymbol{→}\AgdaSpace{}%
\AgdaSymbol{(}\AgdaBound{d}\AgdaSpace{}%
\AgdaSymbol{:}\AgdaSpace{}%
\AgdaInductiveConstructor{★}\AgdaSpace{}%
\AgdaOperator{\AgdaDatatype{⊑}}\AgdaSpace{}%
\AgdaBound{A}\AgdaSymbol{)}\<%
\\
%
\>[2]\AgdaSymbol{→}\AgdaSpace{}%
\AgdaBound{c}\AgdaSpace{}%
\AgdaOperator{\AgdaDatatype{≡}}\AgdaSpace{}%
\AgdaBound{d}\<%
\\
\>[0]\AgdaFunction{dyn-prec-unique}\AgdaSpace{}%
\AgdaSymbol{\{}\AgdaInductiveConstructor{★}\AgdaSymbol{\}}\AgdaSpace{}%
\AgdaInductiveConstructor{unk⊑unk}\AgdaSpace{}%
\AgdaInductiveConstructor{unk⊑unk}\AgdaSpace{}%
\AgdaSymbol{=}\AgdaSpace{}%
\AgdaInductiveConstructor{refl}\<%
\\
\>[0]\AgdaFunction{dyn-prec-unique}\AgdaSpace{}%
\AgdaSymbol{\{}\AgdaInductiveConstructor{★}\AgdaSymbol{\}}\AgdaSpace{}%
\AgdaInductiveConstructor{unk⊑unk}\AgdaSpace{}%
\AgdaSymbol{(}\AgdaInductiveConstructor{unk⊑}\AgdaSpace{}%
\AgdaSymbol{\{}\AgdaOperator{\AgdaInductiveConstructor{\$ᵍ}}\AgdaSpace{}%
\AgdaBound{ι}\AgdaSymbol{\}}\AgdaSpace{}%
\AgdaSymbol{())}\<%
\\
\>[0]\AgdaFunction{dyn-prec-unique}\AgdaSpace{}%
\AgdaSymbol{\{}\AgdaInductiveConstructor{★}\AgdaSymbol{\}}\AgdaSpace{}%
\AgdaInductiveConstructor{unk⊑unk}\AgdaSpace{}%
\AgdaSymbol{(}\AgdaInductiveConstructor{unk⊑}\AgdaSpace{}%
\AgdaSymbol{\{}\AgdaInductiveConstructor{★⇒★}\AgdaSymbol{\}}\AgdaSpace{}%
\AgdaSymbol{())}\<%
\\
\>[0]\AgdaFunction{dyn-prec-unique}\AgdaSpace{}%
\AgdaSymbol{\{}\AgdaInductiveConstructor{★}\AgdaSymbol{\}}\AgdaSpace{}%
\AgdaSymbol{(}\AgdaInductiveConstructor{unk⊑}\AgdaSpace{}%
\AgdaSymbol{\{}\AgdaOperator{\AgdaInductiveConstructor{\$ᵍ}}\AgdaSpace{}%
\AgdaBound{ι}\AgdaSymbol{\}}\AgdaSpace{}%
\AgdaSymbol{())}\AgdaSpace{}%
\AgdaBound{d}\<%
\\
\>[0]\AgdaFunction{dyn-prec-unique}\AgdaSpace{}%
\AgdaSymbol{\{}\AgdaInductiveConstructor{★}\AgdaSymbol{\}}\AgdaSpace{}%
\AgdaSymbol{(}\AgdaInductiveConstructor{unk⊑}\AgdaSpace{}%
\AgdaSymbol{\{}\AgdaInductiveConstructor{★⇒★}\AgdaSymbol{\}}\AgdaSpace{}%
\AgdaSymbol{())}\AgdaSpace{}%
\AgdaBound{d}\<%
\\
\>[0]\AgdaFunction{dyn-prec-unique}\AgdaSpace{}%
\AgdaSymbol{\{}\AgdaOperator{\AgdaInductiveConstructor{\$ₜ}}\AgdaSpace{}%
\AgdaBound{ι}\AgdaSymbol{\}}\AgdaSpace{}%
\AgdaSymbol{(}\AgdaInductiveConstructor{unk⊑}\AgdaSpace{}%
\AgdaSymbol{\{}\AgdaOperator{\AgdaInductiveConstructor{\$ᵍ}}\AgdaSpace{}%
\AgdaDottedPattern{\AgdaSymbol{.}}\AgdaDottedPattern{\AgdaBound{ι}}\AgdaSymbol{\}}\AgdaSpace{}%
\AgdaInductiveConstructor{base⊑}\AgdaSymbol{)}\AgdaSpace{}%
\AgdaSymbol{(}\AgdaInductiveConstructor{unk⊑}\AgdaSpace{}%
\AgdaSymbol{\{}\AgdaOperator{\AgdaInductiveConstructor{\$ᵍ}}\AgdaSpace{}%
\AgdaDottedPattern{\AgdaSymbol{.}}\AgdaDottedPattern{\AgdaBound{ι}}\AgdaSymbol{\}}\AgdaSpace{}%
\AgdaInductiveConstructor{base⊑}\AgdaSymbol{)}\AgdaSpace{}%
\AgdaSymbol{=}\AgdaSpace{}%
\AgdaInductiveConstructor{refl}\<%
\\
\>[0]\AgdaFunction{dyn-prec-unique}\AgdaSpace{}%
\AgdaSymbol{\{}\AgdaBound{A}\AgdaSpace{}%
\AgdaOperator{\AgdaInductiveConstructor{⇒}}\AgdaSpace{}%
\AgdaBound{A₁}\AgdaSymbol{\}}\AgdaSpace{}%
\AgdaSymbol{(}\AgdaInductiveConstructor{unk⊑}\AgdaSpace{}%
\AgdaSymbol{\{}\AgdaInductiveConstructor{★⇒★}\AgdaSymbol{\}}\AgdaSpace{}%
\AgdaSymbol{(}\AgdaInductiveConstructor{fun⊑}\AgdaSpace{}%
\AgdaBound{c}\AgdaSpace{}%
\AgdaBound{c₁}\AgdaSymbol{))}\AgdaSpace{}%
\AgdaSymbol{(}\AgdaInductiveConstructor{unk⊑}\AgdaSpace{}%
\AgdaSymbol{\{}\AgdaInductiveConstructor{★⇒★}\AgdaSymbol{\}}\AgdaSpace{}%
\AgdaSymbol{(}\AgdaInductiveConstructor{fun⊑}\AgdaSpace{}%
\AgdaBound{d}\AgdaSpace{}%
\AgdaBound{d₁}\AgdaSymbol{))}\<%
\\
\>[0][@{}l@{\AgdaIndent{0}}]%
\>[4]\AgdaKeyword{with}\AgdaSpace{}%
\AgdaFunction{dyn-prec-unique}\AgdaSpace{}%
\AgdaBound{c}\AgdaSpace{}%
\AgdaBound{d}\AgdaSpace{}%
\AgdaSymbol{|}\AgdaSpace{}%
\AgdaFunction{dyn-prec-unique}\AgdaSpace{}%
\AgdaBound{c₁}\AgdaSpace{}%
\AgdaBound{d₁}\<%
\\
\>[0]\AgdaSymbol{...}\AgdaSpace{}%
\AgdaSymbol{|}\AgdaSpace{}%
\AgdaInductiveConstructor{refl}\AgdaSpace{}%
\AgdaSymbol{|}\AgdaSpace{}%
\AgdaInductiveConstructor{refl}\AgdaSpace{}%
\AgdaSymbol{=}\AgdaSpace{}%
\AgdaInductiveConstructor{refl}\<%
\\
%
\\[\AgdaEmptyExtraSkip]%
\>[0]\AgdaFunction{gnd-prec-unique}\AgdaSpace{}%
\AgdaSymbol{:}\AgdaSpace{}%
\AgdaSymbol{∀\{}\AgdaBound{G}\AgdaSpace{}%
\AgdaBound{A}\AgdaSymbol{\}}\<%
\\
\>[0][@{}l@{\AgdaIndent{0}}]%
\>[3]\AgdaSymbol{→}\AgdaSpace{}%
\AgdaSymbol{(}\AgdaBound{c}\AgdaSpace{}%
\AgdaSymbol{:}\AgdaSpace{}%
\AgdaOperator{\AgdaFunction{⌈}}\AgdaSpace{}%
\AgdaBound{G}\AgdaSpace{}%
\AgdaOperator{\AgdaFunction{⌉}}\AgdaSpace{}%
\AgdaOperator{\AgdaDatatype{⊑}}\AgdaSpace{}%
\AgdaBound{A}\AgdaSymbol{)}\<%
\\
%
\>[3]\AgdaSymbol{→}\AgdaSpace{}%
\AgdaSymbol{(}\AgdaBound{d}\AgdaSpace{}%
\AgdaSymbol{:}\AgdaSpace{}%
\AgdaOperator{\AgdaFunction{⌈}}\AgdaSpace{}%
\AgdaBound{G}\AgdaSpace{}%
\AgdaOperator{\AgdaFunction{⌉}}\AgdaSpace{}%
\AgdaOperator{\AgdaDatatype{⊑}}\AgdaSpace{}%
\AgdaBound{A}\AgdaSymbol{)}\<%
\\
%
\>[3]\AgdaSymbol{→}\AgdaSpace{}%
\AgdaBound{c}\AgdaSpace{}%
\AgdaOperator{\AgdaDatatype{≡}}\AgdaSpace{}%
\AgdaBound{d}\<%
\\
\>[0]\AgdaFunction{gnd-prec-unique}\AgdaSpace{}%
\AgdaSymbol{\{}\AgdaOperator{\AgdaInductiveConstructor{\$ᵍ}}\AgdaSpace{}%
\AgdaBound{ι}\AgdaSymbol{\}}\AgdaSpace{}%
\AgdaSymbol{\{}\AgdaDottedPattern{\AgdaSymbol{.(}}\AgdaDottedPattern{\AgdaOperator{\AgdaInductiveConstructor{\$ₜ}}}\AgdaSpace{}%
\AgdaDottedPattern{\AgdaBound{ι}}\AgdaDottedPattern{\AgdaSymbol{)}}\AgdaSymbol{\}}\AgdaSpace{}%
\AgdaInductiveConstructor{base⊑}\AgdaSpace{}%
\AgdaInductiveConstructor{base⊑}\AgdaSpace{}%
\AgdaSymbol{=}\AgdaSpace{}%
\AgdaInductiveConstructor{refl}\<%
\\
\>[0]\AgdaFunction{gnd-prec-unique}\AgdaSpace{}%
\AgdaSymbol{\{}\AgdaInductiveConstructor{★⇒★}\AgdaSymbol{\}}\AgdaSpace{}%
\AgdaSymbol{\{}\AgdaDottedPattern{\AgdaSymbol{.(\AgdaUnderscore{}}}\AgdaSpace{}%
\AgdaDottedPattern{\AgdaOperator{\AgdaInductiveConstructor{⇒}}}\AgdaSpace{}%
\AgdaDottedPattern{\AgdaSymbol{\AgdaUnderscore{})}}\AgdaSymbol{\}}\AgdaSpace{}%
\AgdaSymbol{(}\AgdaInductiveConstructor{fun⊑}\AgdaSpace{}%
\AgdaBound{c}\AgdaSpace{}%
\AgdaBound{c₁}\AgdaSymbol{)}\AgdaSpace{}%
\AgdaSymbol{(}\AgdaInductiveConstructor{fun⊑}\AgdaSpace{}%
\AgdaBound{d}\AgdaSpace{}%
\AgdaBound{d₁}\AgdaSymbol{)}\<%
\\
\>[0][@{}l@{\AgdaIndent{0}}]%
\>[4]\AgdaKeyword{with}\AgdaSpace{}%
\AgdaFunction{dyn-prec-unique}\AgdaSpace{}%
\AgdaBound{c}\AgdaSpace{}%
\AgdaBound{d}\AgdaSpace{}%
\AgdaSymbol{|}\AgdaSpace{}%
\AgdaFunction{dyn-prec-unique}\AgdaSpace{}%
\AgdaBound{c₁}\AgdaSpace{}%
\AgdaBound{d₁}\<%
\\
\>[0]\AgdaSymbol{...}\AgdaSpace{}%
\AgdaSymbol{|}\AgdaSpace{}%
\AgdaInductiveConstructor{refl}\AgdaSpace{}%
\AgdaSymbol{|}\AgdaSpace{}%
\AgdaInductiveConstructor{refl}\AgdaSpace{}%
\AgdaSymbol{=}\AgdaSpace{}%
\AgdaInductiveConstructor{refl}\<%
\end{code}

Figure~\ref{fig:term-precision} defines the precision relation on
terms.  The judgement has the form $Γ ⊩ M ⊑ M′ ⦂ c$ where Γ is a list
of precision-related types and $c : A ⊑ A′$ is a precision derivation
for the types of $M$ and $M′$. There are two rules for injection and
also for projection, allowing either to appear on the left or right
across from an arbitrary term. However, when injection is on the
right, the term on the left must have type ★ (rule
\textsf{⊑-inj-R}).  Similarly, when projection is on the right, the
term on the left must have type ★ (rule \textsf{⊑-proj-R}).

\begin{figure}[tbp]
\begin{code}%
\>[0]\AgdaKeyword{infix}\AgdaSpace{}%
\AgdaNumber{3}\AgdaSpace{}%
\AgdaOperator{\AgdaDatatype{\AgdaUnderscore{}⊩\AgdaUnderscore{}⊑\AgdaUnderscore{}⦂\AgdaUnderscore{}}}\<%
\\
\>[0]\AgdaKeyword{data}\AgdaSpace{}%
\AgdaOperator{\AgdaDatatype{\AgdaUnderscore{}⊩\AgdaUnderscore{}⊑\AgdaUnderscore{}⦂\AgdaUnderscore{}}}\AgdaSpace{}%
\AgdaSymbol{:}\AgdaSpace{}%
\AgdaDatatype{List}\AgdaSpace{}%
\AgdaFunction{Prec}\AgdaSpace{}%
\AgdaSymbol{→}\AgdaSpace{}%
\AgdaDatatype{Term}\AgdaSpace{}%
\AgdaSymbol{→}\AgdaSpace{}%
\AgdaDatatype{Term}\AgdaSpace{}%
\AgdaSymbol{→}\AgdaSpace{}%
\AgdaSymbol{∀\{}\AgdaBound{A}\AgdaSpace{}%
\AgdaBound{B}\AgdaSpace{}%
\AgdaSymbol{:}\AgdaSpace{}%
\AgdaDatatype{Type}\AgdaSymbol{\}}\AgdaSpace{}%
\AgdaSymbol{→}\AgdaSpace{}%
\AgdaBound{A}\AgdaSpace{}%
\AgdaOperator{\AgdaDatatype{⊑}}\AgdaSpace{}%
\AgdaBound{B}\AgdaSpace{}%
\AgdaSymbol{→}\AgdaSpace{}%
\AgdaPrimitive{Set}%
\>[70]\AgdaKeyword{where}\<%
\\
\>[0][@{}l@{\AgdaIndent{0}}]%
\>[2]\AgdaInductiveConstructor{⊑-var}\AgdaSpace{}%
\AgdaSymbol{:}\AgdaSpace{}%
\AgdaSymbol{∀}\AgdaSpace{}%
\AgdaSymbol{\{}\AgdaBound{Γ}\AgdaSpace{}%
\AgdaBound{x}\AgdaSpace{}%
\AgdaBound{A⊑B}\AgdaSymbol{\}}%
\>[23]\AgdaSymbol{→}%
\>[26]\AgdaBound{Γ}\AgdaSpace{}%
\AgdaOperator{\AgdaFunction{∋}}\AgdaSpace{}%
\AgdaBound{x}\AgdaSpace{}%
\AgdaOperator{\AgdaFunction{⦂}}\AgdaSpace{}%
\AgdaBound{A⊑B}%
\>[39]\AgdaSymbol{→}%
\>[42]\AgdaBound{Γ}\AgdaSpace{}%
\AgdaOperator{\AgdaDatatype{⊩}}\AgdaSpace{}%
\AgdaSymbol{(}\AgdaOperator{\AgdaInductiveConstructor{`}}\AgdaSpace{}%
\AgdaBound{x}\AgdaSymbol{)}\AgdaSpace{}%
\AgdaOperator{\AgdaDatatype{⊑}}\AgdaSpace{}%
\AgdaSymbol{(}\AgdaOperator{\AgdaInductiveConstructor{`}}\AgdaSpace{}%
\AgdaBound{x}\AgdaSymbol{)}\AgdaSpace{}%
\AgdaOperator{\AgdaDatatype{⦂}}\AgdaSpace{}%
\AgdaField{proj₂}\AgdaSpace{}%
\AgdaSymbol{(}\AgdaField{proj₂}\AgdaSpace{}%
\AgdaBound{A⊑B}\AgdaSymbol{)}\<%
\\
%
\>[2]\AgdaInductiveConstructor{⊑-lit}\AgdaSpace{}%
\AgdaSymbol{:}\AgdaSpace{}%
\AgdaSymbol{∀}\AgdaSpace{}%
\AgdaSymbol{\{}\AgdaBound{Γ}\AgdaSpace{}%
\AgdaBound{c}\AgdaSymbol{\}}\AgdaSpace{}%
\AgdaSymbol{→}%
\>[21]\AgdaBound{Γ}\AgdaSpace{}%
\AgdaOperator{\AgdaDatatype{⊩}}\AgdaSpace{}%
\AgdaSymbol{(}\AgdaInductiveConstructor{\$}\AgdaSpace{}%
\AgdaBound{c}\AgdaSymbol{)}\AgdaSpace{}%
\AgdaOperator{\AgdaDatatype{⊑}}\AgdaSpace{}%
\AgdaSymbol{(}\AgdaInductiveConstructor{\$}\AgdaSpace{}%
\AgdaBound{c}\AgdaSymbol{)}\AgdaSpace{}%
\AgdaOperator{\AgdaDatatype{⦂}}\AgdaSpace{}%
\AgdaInductiveConstructor{base⊑}\AgdaSymbol{\{}\AgdaFunction{typeof}\AgdaSpace{}%
\AgdaBound{c}\AgdaSymbol{\}}\<%
\\
%
\>[2]\AgdaInductiveConstructor{⊑-app}\AgdaSpace{}%
\AgdaSymbol{:}\AgdaSpace{}%
\AgdaSymbol{∀\{}\AgdaBound{Γ}\AgdaSpace{}%
\AgdaBound{L}\AgdaSpace{}%
\AgdaBound{M}\AgdaSpace{}%
\AgdaBound{L′}\AgdaSpace{}%
\AgdaBound{M′}\AgdaSpace{}%
\AgdaBound{A}\AgdaSpace{}%
\AgdaBound{B}\AgdaSpace{}%
\AgdaBound{C}\AgdaSpace{}%
\AgdaBound{D}\AgdaSymbol{\}\{}\AgdaBound{c}\AgdaSpace{}%
\AgdaSymbol{:}\AgdaSpace{}%
\AgdaBound{A}\AgdaSpace{}%
\AgdaOperator{\AgdaDatatype{⊑}}\AgdaSpace{}%
\AgdaBound{C}\AgdaSymbol{\}\{}\AgdaBound{d}\AgdaSpace{}%
\AgdaSymbol{:}\AgdaSpace{}%
\AgdaBound{B}\AgdaSpace{}%
\AgdaOperator{\AgdaDatatype{⊑}}\AgdaSpace{}%
\AgdaBound{D}\AgdaSymbol{\}}\<%
\\
\>[2][@{}l@{\AgdaIndent{0}}]%
\>[5]\AgdaSymbol{→}\AgdaSpace{}%
\AgdaBound{Γ}\AgdaSpace{}%
\AgdaOperator{\AgdaDatatype{⊩}}\AgdaSpace{}%
\AgdaBound{L}\AgdaSpace{}%
\AgdaOperator{\AgdaDatatype{⊑}}\AgdaSpace{}%
\AgdaBound{L′}\AgdaSpace{}%
\AgdaOperator{\AgdaDatatype{⦂}}\AgdaSpace{}%
\AgdaInductiveConstructor{fun⊑}\AgdaSpace{}%
\AgdaBound{c}\AgdaSpace{}%
\AgdaBound{d}%
\>[30]\AgdaSymbol{→}%
\>[33]\AgdaBound{Γ}\AgdaSpace{}%
\AgdaOperator{\AgdaDatatype{⊩}}\AgdaSpace{}%
\AgdaBound{M}\AgdaSpace{}%
\AgdaOperator{\AgdaDatatype{⊑}}\AgdaSpace{}%
\AgdaBound{M′}\AgdaSpace{}%
\AgdaOperator{\AgdaDatatype{⦂}}\AgdaSpace{}%
\AgdaBound{c}\<%
\\
%
\>[5]\AgdaSymbol{→}\AgdaSpace{}%
\AgdaBound{Γ}\AgdaSpace{}%
\AgdaOperator{\AgdaDatatype{⊩}}\AgdaSpace{}%
\AgdaBound{L}\AgdaSpace{}%
\AgdaOperator{\AgdaInductiveConstructor{·}}\AgdaSpace{}%
\AgdaBound{M}\AgdaSpace{}%
\AgdaOperator{\AgdaDatatype{⊑}}\AgdaSpace{}%
\AgdaBound{L′}\AgdaSpace{}%
\AgdaOperator{\AgdaInductiveConstructor{·}}\AgdaSpace{}%
\AgdaBound{M′}\AgdaSpace{}%
\AgdaOperator{\AgdaDatatype{⦂}}\AgdaSpace{}%
\AgdaBound{d}\<%
\\
%
\>[2]\AgdaInductiveConstructor{⊑-lam}\AgdaSpace{}%
\AgdaSymbol{:}\AgdaSpace{}%
\AgdaSymbol{∀\{}\AgdaBound{Γ}\AgdaSpace{}%
\AgdaBound{N}\AgdaSpace{}%
\AgdaBound{N′}\AgdaSpace{}%
\AgdaBound{A}\AgdaSpace{}%
\AgdaBound{B}\AgdaSpace{}%
\AgdaBound{C}\AgdaSpace{}%
\AgdaBound{D}\AgdaSymbol{\}\{}\AgdaBound{c}\AgdaSpace{}%
\AgdaSymbol{:}\AgdaSpace{}%
\AgdaBound{A}\AgdaSpace{}%
\AgdaOperator{\AgdaDatatype{⊑}}\AgdaSpace{}%
\AgdaBound{C}\AgdaSymbol{\}\{}\AgdaBound{d}\AgdaSpace{}%
\AgdaSymbol{:}\AgdaSpace{}%
\AgdaBound{B}\AgdaSpace{}%
\AgdaOperator{\AgdaDatatype{⊑}}\AgdaSpace{}%
\AgdaBound{D}\AgdaSymbol{\}}\<%
\\
\>[2][@{}l@{\AgdaIndent{0}}]%
\>[5]\AgdaSymbol{→}\AgdaSpace{}%
\AgdaSymbol{(}\AgdaBound{A}\AgdaSpace{}%
\AgdaOperator{\AgdaInductiveConstructor{,}}\AgdaSpace{}%
\AgdaBound{C}\AgdaSpace{}%
\AgdaOperator{\AgdaInductiveConstructor{,}}\AgdaSpace{}%
\AgdaBound{c}\AgdaSymbol{)}\AgdaSpace{}%
\AgdaOperator{\AgdaInductiveConstructor{∷}}\AgdaSpace{}%
\AgdaBound{Γ}\AgdaSpace{}%
\AgdaOperator{\AgdaDatatype{⊩}}\AgdaSpace{}%
\AgdaBound{N}\AgdaSpace{}%
\AgdaOperator{\AgdaDatatype{⊑}}\AgdaSpace{}%
\AgdaBound{N′}\AgdaSpace{}%
\AgdaOperator{\AgdaDatatype{⦂}}\AgdaSpace{}%
\AgdaBound{d}%
\>[37]\AgdaSymbol{→}%
\>[40]\AgdaBound{Γ}\AgdaSpace{}%
\AgdaOperator{\AgdaDatatype{⊩}}\AgdaSpace{}%
\AgdaInductiveConstructor{ƛ}\AgdaSpace{}%
\AgdaBound{N}\AgdaSpace{}%
\AgdaOperator{\AgdaDatatype{⊑}}\AgdaSpace{}%
\AgdaInductiveConstructor{ƛ}\AgdaSpace{}%
\AgdaBound{N′}\AgdaSpace{}%
\AgdaOperator{\AgdaDatatype{⦂}}\AgdaSpace{}%
\AgdaInductiveConstructor{fun⊑}\AgdaSpace{}%
\AgdaBound{c}\AgdaSpace{}%
\AgdaBound{d}\<%
\\
%
\>[2]\AgdaInductiveConstructor{⊑-inj-L}\AgdaSpace{}%
\AgdaSymbol{:}\AgdaSpace{}%
\AgdaSymbol{∀\{}\AgdaBound{Γ}\AgdaSpace{}%
\AgdaBound{M}\AgdaSpace{}%
\AgdaBound{M′}\AgdaSymbol{\}\{}\AgdaBound{G}\AgdaSpace{}%
\AgdaBound{B}\AgdaSymbol{\}\{}\AgdaBound{c}\AgdaSpace{}%
\AgdaSymbol{:}\AgdaSpace{}%
\AgdaOperator{\AgdaFunction{⌈}}\AgdaSpace{}%
\AgdaBound{G}\AgdaSpace{}%
\AgdaOperator{\AgdaFunction{⌉}}\AgdaSpace{}%
\AgdaOperator{\AgdaDatatype{⊑}}\AgdaSpace{}%
\AgdaBound{B}\AgdaSymbol{\}}\<%
\\
\>[2][@{}l@{\AgdaIndent{0}}]%
\>[5]\AgdaSymbol{→}\AgdaSpace{}%
\AgdaBound{Γ}\AgdaSpace{}%
\AgdaOperator{\AgdaDatatype{⊩}}\AgdaSpace{}%
\AgdaBound{M}\AgdaSpace{}%
\AgdaOperator{\AgdaDatatype{⊑}}\AgdaSpace{}%
\AgdaBound{M′}\AgdaSpace{}%
\AgdaOperator{\AgdaDatatype{⦂}}\AgdaSpace{}%
\AgdaBound{c}%
\>[23]\AgdaSymbol{→}%
\>[26]\AgdaBound{Γ}\AgdaSpace{}%
\AgdaOperator{\AgdaDatatype{⊩}}\AgdaSpace{}%
\AgdaBound{M}\AgdaSpace{}%
\AgdaOperator{\AgdaInductiveConstructor{⟨}}\AgdaSpace{}%
\AgdaBound{G}\AgdaSpace{}%
\AgdaOperator{\AgdaInductiveConstructor{!⟩}}\AgdaSpace{}%
\AgdaOperator{\AgdaDatatype{⊑}}\AgdaSpace{}%
\AgdaBound{M′}\AgdaSpace{}%
\AgdaOperator{\AgdaDatatype{⦂}}\AgdaSpace{}%
\AgdaInductiveConstructor{unk⊑}\AgdaSymbol{\{}\AgdaBound{G}\AgdaSymbol{\}\{}\AgdaBound{B}\AgdaSymbol{\}}\AgdaSpace{}%
\AgdaBound{c}\<%
\\
%
\>[2]\AgdaInductiveConstructor{⊑-inj-R}\AgdaSpace{}%
\AgdaSymbol{:}\AgdaSpace{}%
\AgdaSymbol{∀\{}\AgdaBound{Γ}\AgdaSpace{}%
\AgdaBound{M}\AgdaSpace{}%
\AgdaBound{M′}\AgdaSymbol{\}\{}\AgdaBound{G}\AgdaSymbol{\}\{}\AgdaBound{c}\AgdaSpace{}%
\AgdaSymbol{:}\AgdaSpace{}%
\AgdaInductiveConstructor{★}\AgdaSpace{}%
\AgdaOperator{\AgdaDatatype{⊑}}\AgdaSpace{}%
\AgdaOperator{\AgdaFunction{⌈}}\AgdaSpace{}%
\AgdaBound{G}\AgdaSpace{}%
\AgdaOperator{\AgdaFunction{⌉}}\AgdaSymbol{\}}\<%
\\
\>[2][@{}l@{\AgdaIndent{0}}]%
\>[5]\AgdaSymbol{→}\AgdaSpace{}%
\AgdaBound{Γ}\AgdaSpace{}%
\AgdaOperator{\AgdaDatatype{⊩}}\AgdaSpace{}%
\AgdaBound{M}\AgdaSpace{}%
\AgdaOperator{\AgdaDatatype{⊑}}\AgdaSpace{}%
\AgdaBound{M′}\AgdaSpace{}%
\AgdaOperator{\AgdaDatatype{⦂}}\AgdaSpace{}%
\AgdaBound{c}%
\>[23]\AgdaSymbol{→}%
\>[26]\AgdaBound{Γ}\AgdaSpace{}%
\AgdaOperator{\AgdaDatatype{⊩}}\AgdaSpace{}%
\AgdaBound{M}\AgdaSpace{}%
\AgdaOperator{\AgdaDatatype{⊑}}\AgdaSpace{}%
\AgdaBound{M′}\AgdaSpace{}%
\AgdaOperator{\AgdaInductiveConstructor{⟨}}\AgdaSpace{}%
\AgdaBound{G}\AgdaSpace{}%
\AgdaOperator{\AgdaInductiveConstructor{!⟩}}\AgdaSpace{}%
\AgdaOperator{\AgdaDatatype{⦂}}\AgdaSpace{}%
\AgdaInductiveConstructor{unk⊑unk}\<%
\\
%
\>[2]\AgdaInductiveConstructor{⊑-proj-L}\AgdaSpace{}%
\AgdaSymbol{:}\AgdaSpace{}%
\AgdaSymbol{∀\{}\AgdaBound{Γ}\AgdaSpace{}%
\AgdaBound{M}\AgdaSpace{}%
\AgdaBound{M′}\AgdaSpace{}%
\AgdaBound{H}\AgdaSpace{}%
\AgdaBound{B}\AgdaSymbol{\}\{}\AgdaBound{c}\AgdaSpace{}%
\AgdaSymbol{:}\AgdaSpace{}%
\AgdaOperator{\AgdaFunction{⌈}}\AgdaSpace{}%
\AgdaBound{H}\AgdaSpace{}%
\AgdaOperator{\AgdaFunction{⌉}}\AgdaSpace{}%
\AgdaOperator{\AgdaDatatype{⊑}}\AgdaSpace{}%
\AgdaBound{B}\AgdaSymbol{\}}\<%
\\
\>[2][@{}l@{\AgdaIndent{0}}]%
\>[5]\AgdaSymbol{→}\AgdaSpace{}%
\AgdaBound{Γ}\AgdaSpace{}%
\AgdaOperator{\AgdaDatatype{⊩}}\AgdaSpace{}%
\AgdaBound{M}\AgdaSpace{}%
\AgdaOperator{\AgdaDatatype{⊑}}\AgdaSpace{}%
\AgdaBound{M′}\AgdaSpace{}%
\AgdaOperator{\AgdaDatatype{⦂}}\AgdaSpace{}%
\AgdaInductiveConstructor{unk⊑}\AgdaSpace{}%
\AgdaBound{c}%
\>[28]\AgdaSymbol{→}%
\>[31]\AgdaBound{Γ}\AgdaSpace{}%
\AgdaOperator{\AgdaDatatype{⊩}}\AgdaSpace{}%
\AgdaBound{M}\AgdaSpace{}%
\AgdaOperator{\AgdaInductiveConstructor{⟨}}\AgdaSpace{}%
\AgdaBound{H}\AgdaSpace{}%
\AgdaOperator{\AgdaInductiveConstructor{?⟩}}\AgdaSpace{}%
\AgdaOperator{\AgdaDatatype{⊑}}\AgdaSpace{}%
\AgdaBound{M′}\AgdaSpace{}%
\AgdaOperator{\AgdaDatatype{⦂}}\AgdaSpace{}%
\AgdaBound{c}\<%
\\
%
\>[2]\AgdaInductiveConstructor{⊑-proj-R}\AgdaSpace{}%
\AgdaSymbol{:}\AgdaSpace{}%
\AgdaSymbol{∀\{}\AgdaBound{Γ}\AgdaSpace{}%
\AgdaBound{M}\AgdaSpace{}%
\AgdaBound{M′}\AgdaSpace{}%
\AgdaBound{H}\AgdaSymbol{\}\{}\AgdaBound{c}\AgdaSpace{}%
\AgdaSymbol{:}\AgdaSpace{}%
\AgdaInductiveConstructor{★}\AgdaSpace{}%
\AgdaOperator{\AgdaDatatype{⊑}}\AgdaSpace{}%
\AgdaOperator{\AgdaFunction{⌈}}\AgdaSpace{}%
\AgdaBound{H}\AgdaSpace{}%
\AgdaOperator{\AgdaFunction{⌉}}\AgdaSymbol{\}}\<%
\\
\>[2][@{}l@{\AgdaIndent{0}}]%
\>[5]\AgdaSymbol{→}\AgdaSpace{}%
\AgdaBound{Γ}\AgdaSpace{}%
\AgdaOperator{\AgdaDatatype{⊩}}\AgdaSpace{}%
\AgdaBound{M}\AgdaSpace{}%
\AgdaOperator{\AgdaDatatype{⊑}}\AgdaSpace{}%
\AgdaBound{M′}\AgdaSpace{}%
\AgdaOperator{\AgdaDatatype{⦂}}\AgdaSpace{}%
\AgdaInductiveConstructor{unk⊑unk}%
\>[29]\AgdaSymbol{→}%
\>[32]\AgdaBound{Γ}\AgdaSpace{}%
\AgdaOperator{\AgdaDatatype{⊩}}\AgdaSpace{}%
\AgdaBound{M}\AgdaSpace{}%
\AgdaOperator{\AgdaDatatype{⊑}}\AgdaSpace{}%
\AgdaBound{M′}\AgdaSpace{}%
\AgdaOperator{\AgdaInductiveConstructor{⟨}}\AgdaSpace{}%
\AgdaBound{H}\AgdaSpace{}%
\AgdaOperator{\AgdaInductiveConstructor{?⟩}}%
\>[51]\AgdaOperator{\AgdaDatatype{⦂}}\AgdaSpace{}%
\AgdaBound{c}\<%
\\
%
\>[2]\AgdaInductiveConstructor{⊑-blame}\AgdaSpace{}%
\AgdaSymbol{:}\AgdaSpace{}%
\AgdaSymbol{∀\{}\AgdaBound{Γ}\AgdaSpace{}%
\AgdaBound{M}\AgdaSpace{}%
\AgdaBound{A}\AgdaSymbol{\}}%
\>[22]\AgdaSymbol{→}%
\>[25]\AgdaFunction{map}\AgdaSpace{}%
\AgdaField{proj₁}\AgdaSpace{}%
\AgdaBound{Γ}\AgdaSpace{}%
\AgdaOperator{\AgdaDatatype{⊢}}\AgdaSpace{}%
\AgdaBound{M}\AgdaSpace{}%
\AgdaOperator{\AgdaDatatype{⦂}}\AgdaSpace{}%
\AgdaBound{A}%
\>[46]\AgdaSymbol{→}%
\>[49]\AgdaBound{Γ}\AgdaSpace{}%
\AgdaOperator{\AgdaDatatype{⊩}}\AgdaSpace{}%
\AgdaBound{M}\AgdaSpace{}%
\AgdaOperator{\AgdaDatatype{⊑}}\AgdaSpace{}%
\AgdaInductiveConstructor{blame}\AgdaSpace{}%
\AgdaOperator{\AgdaDatatype{⦂}}\AgdaSpace{}%
\AgdaFunction{Refl⊑}\AgdaSymbol{\{}\AgdaBound{A}\AgdaSymbol{\}}\<%
\end{code}
\caption{Precision Relation on Terms}
\label{fig:term-precision}
\end{figure}

With precision defined, we are ready to discuss the gradual guarantee.
It states that if $M$ is less precise than $M′$, then $M$ and $M′$
behave in a similar way, as defined below by the predicate
$\mathsf{gradual}\,M\,M′$. In particular, it says that the
less-precise term behaves exactly like the more-precise term. On the
other hand the more-precise term may reduce to \textsf{blame} even
though the less-precise term does not.

\begin{code}%
\>[0]\AgdaFunction{gradual}\AgdaSpace{}%
\AgdaSymbol{:}\AgdaSpace{}%
\AgdaSymbol{(}\AgdaBound{M}\AgdaSpace{}%
\AgdaBound{M′}\AgdaSpace{}%
\AgdaSymbol{:}\AgdaSpace{}%
\AgdaDatatype{Term}\AgdaSymbol{)}\AgdaSpace{}%
\AgdaSymbol{→}\AgdaSpace{}%
\AgdaPrimitive{Set}\<%
\\
\>[0]\AgdaFunction{gradual}\AgdaSpace{}%
\AgdaBound{M}\AgdaSpace{}%
\AgdaBound{M′}\AgdaSpace{}%
\AgdaSymbol{=}\AgdaSpace{}%
\AgdaSymbol{(}\AgdaBound{M′}\AgdaSpace{}%
\AgdaOperator{\AgdaFunction{⇓}}\AgdaSpace{}%
\AgdaSymbol{→}\AgdaSpace{}%
\AgdaBound{M}\AgdaSpace{}%
\AgdaOperator{\AgdaFunction{⇓}}\AgdaSymbol{)}\AgdaSpace{}%
\AgdaOperator{\AgdaFunction{×}}\AgdaSpace{}%
\AgdaSymbol{(}\AgdaBound{M′}\AgdaSpace{}%
\AgdaOperator{\AgdaFunction{⇑}}\AgdaSpace{}%
\AgdaSymbol{→}\AgdaSpace{}%
\AgdaBound{M}\AgdaSpace{}%
\AgdaOperator{\AgdaFunction{⇑}}\AgdaSymbol{)}\<%
\\
\>[0][@{}l@{\AgdaIndent{0}}]%
\>[4]\AgdaOperator{\AgdaFunction{×}}\AgdaSpace{}%
\AgdaSymbol{(}\AgdaBound{M}\AgdaSpace{}%
\AgdaOperator{\AgdaFunction{⇓}}\AgdaSpace{}%
\AgdaSymbol{→}\AgdaSpace{}%
\AgdaBound{M′}\AgdaSpace{}%
\AgdaOperator{\AgdaFunction{⇓}}\AgdaSpace{}%
\AgdaOperator{\AgdaDatatype{⊎}}\AgdaSpace{}%
\AgdaBound{M′}\AgdaSpace{}%
\AgdaOperator{\AgdaDatatype{↠}}\AgdaSpace{}%
\AgdaInductiveConstructor{blame}\AgdaSymbol{)}\AgdaSpace{}%
\AgdaOperator{\AgdaFunction{×}}\AgdaSpace{}%
\AgdaSymbol{(}\AgdaBound{M}\AgdaSpace{}%
\AgdaOperator{\AgdaFunction{⇑}}\AgdaSpace{}%
\AgdaSymbol{→}\AgdaSpace{}%
\AgdaBound{M′}\AgdaSpace{}%
\AgdaOperator{\AgdaFunction{⇑⊎blame}}\AgdaSymbol{)}\AgdaSpace{}%
\AgdaOperator{\AgdaFunction{×}}\AgdaSpace{}%
\AgdaSymbol{(}\AgdaBound{M}\AgdaSpace{}%
\AgdaOperator{\AgdaDatatype{↠}}\AgdaSpace{}%
\AgdaInductiveConstructor{blame}\AgdaSpace{}%
\AgdaSymbol{→}\AgdaSpace{}%
\AgdaBound{M′}\AgdaSpace{}%
\AgdaOperator{\AgdaDatatype{↠}}\AgdaSpace{}%
\AgdaInductiveConstructor{blame}\AgdaSymbol{)}\<%
\end{code}

\section{Step-Indexed Logic}
\label{sec:SIL}

The Step-Indexed Logic (SIL) library~\cite{Siek:2023aa} is a shallow
embedding of a modal logic into Agda. The formulas of this logic have
type \textsf{Setᵒ}, which is a record with three fields, the most
important of which is named \textsf{\#} and is a function from ℕ to
\textsf{Set} which expresses the meaning of the formula in Agda.
Think of the ℕ as a count-down clock, with smaller numbers
representing later points in time. The other two fields of the record
contain proofs of the LSLR invariants: (1) that the formula is true at
0, and (2) if the formula is true at some number, then it is true at
all smaller numbers.

SIL includes the connectives of first-order logic (conjunction,
disjunction, existential and universal quantification, etc.).

What makes SIL special is that it includes an operator μᵒ for defining
recursive predicates. In the body of the μᵒ, de Bruijn index 0 refers
to itself, that is, the entire μᵒ. However, variable 0 may only be
used ``later'', that is, underneath at least one use of the modal
operator ▷ᵒ.  The formula in the body of a μᵒ has type
$\mathsf{Set}ˢ\,Γ\,Δ$, where $Γ$ gives the type for each recursive
predicate in scope and $Δ$ records when each recursive predicate is
used (now or later). \textsf{Setˢ} is a record whose field \textsf{\#}
is a function from a list of step-indexed predicates to \textsf{Setᵒ}.
The majority of the lines of code in the SIL library are dedicated to
proving the \textsf{fixpointᵒ} theorem, which states that a recursive
predicate is equivalent to one unrolling of itself. The proof of
\textsf{fixpointᵒ}is an adaptation of the fixpoint theorem of Appel
and McAllester~\cite{Appel:2001aa}.

\begin{code}%
\>[0]\AgdaFunction{\AgdaUnderscore{}}\AgdaSpace{}%
\AgdaSymbol{:}%
\>[644I]\AgdaSymbol{∀(}\AgdaBound{A}\AgdaSpace{}%
\AgdaSymbol{:}\AgdaSpace{}%
\AgdaPrimitive{Set}\AgdaSymbol{)}\AgdaSpace{}%
\AgdaSymbol{(}\AgdaBound{P}\AgdaSpace{}%
\AgdaSymbol{:}\AgdaSpace{}%
\AgdaBound{A}\AgdaSpace{}%
\AgdaSymbol{→}\AgdaSpace{}%
\AgdaRecord{Setˢ}\AgdaSpace{}%
\AgdaSymbol{(}\AgdaBound{A}\AgdaSpace{}%
\AgdaOperator{\AgdaInductiveConstructor{∷}}\AgdaSpace{}%
\AgdaInductiveConstructor{[]}\AgdaSymbol{)}\AgdaSpace{}%
\AgdaSymbol{(}\AgdaInductiveConstructor{cons}\AgdaSpace{}%
\AgdaInductiveConstructor{Later}\AgdaSpace{}%
\AgdaInductiveConstructor{∅}\AgdaSymbol{))}\AgdaSpace{}%
\AgdaSymbol{(}\AgdaBound{a}\AgdaSpace{}%
\AgdaSymbol{:}\AgdaSpace{}%
\AgdaBound{A}\AgdaSymbol{)}\<%
\\
\>[.][@{}l@{}]\<[644I]%
\>[4]\AgdaSymbol{→}\AgdaSpace{}%
\AgdaFunction{μᵒ}\AgdaSpace{}%
\AgdaBound{P}\AgdaSpace{}%
\AgdaBound{a}\AgdaSpace{}%
\AgdaOperator{\AgdaFunction{≡ᵒ}}\AgdaSpace{}%
\AgdaField{\#}\AgdaSpace{}%
\AgdaSymbol{(}\AgdaBound{P}\AgdaSpace{}%
\AgdaBound{a}\AgdaSymbol{)}\AgdaSpace{}%
\AgdaSymbol{(}\AgdaFunction{μᵒ}\AgdaSpace{}%
\AgdaBound{P}\AgdaSpace{}%
\AgdaOperator{\AgdaInductiveConstructor{,}}\AgdaSpace{}%
\AgdaFunction{ttᵖ}\AgdaSymbol{)}\<%
\\
\>[0]\AgdaSymbol{\AgdaUnderscore{}}\AgdaSpace{}%
\AgdaSymbol{=}\AgdaSpace{}%
\AgdaSymbol{λ}\AgdaSpace{}%
\AgdaBound{A}\AgdaSpace{}%
\AgdaBound{P}\AgdaSpace{}%
\AgdaBound{a}\AgdaSpace{}%
\AgdaSymbol{→}\AgdaSpace{}%
\AgdaFunction{fixpointᵒ}\AgdaSpace{}%
\AgdaBound{P}\AgdaSpace{}%
\AgdaBound{a}\<%
\end{code}


\section{A Logical Relation for Precision}
\label{sec:log-rel}

To define a logical relation for precision, we adapt the logical
relation of New~\cite{New:2020ab}, which used explicit step indexing,
into the Step-Indexed Logic. So the logical relation has two directions:
the ≼ direction has the less-precise term taking the lead whereas the
≽ direction has the more-precise term in the lead.

\begin{code}%
\>[0]\AgdaKeyword{data}\AgdaSpace{}%
\AgdaDatatype{Dir}\AgdaSpace{}%
\AgdaSymbol{:}\AgdaSpace{}%
\AgdaPrimitive{Set}\AgdaSpace{}%
\AgdaKeyword{where}\<%
\\
\>[0][@{}l@{\AgdaIndent{0}}]%
\>[2]\AgdaInductiveConstructor{≼}\AgdaSpace{}%
\AgdaSymbol{:}\AgdaSpace{}%
\AgdaDatatype{Dir}\<%
\\
%
\>[2]\AgdaInductiveConstructor{≽}\AgdaSpace{}%
\AgdaSymbol{:}\AgdaSpace{}%
\AgdaDatatype{Dir}\<%
\end{code}

In addition, the logical relation consists of mutually-recursive
relations on both terms and values. SIL does not directly support
mutual recursion, but that can be expressed by combining the two
relations into a single relation whose input is a disjoint sum.  The
formula for expressing membership in these recursive relations is
verbose, so we define the below shorthands.

\begin{code}%
\>[0]\AgdaFunction{LR-type}\AgdaSpace{}%
\AgdaSymbol{:}\AgdaSpace{}%
\AgdaPrimitive{Set}\<%
\\
\>[0]\AgdaFunction{LR-type}\AgdaSpace{}%
\AgdaSymbol{=}\AgdaSpace{}%
\AgdaSymbol{(}\AgdaFunction{Prec}\AgdaSpace{}%
\AgdaOperator{\AgdaFunction{×}}\AgdaSpace{}%
\AgdaDatatype{Dir}\AgdaSpace{}%
\AgdaOperator{\AgdaFunction{×}}\AgdaSpace{}%
\AgdaDatatype{Term}\AgdaSpace{}%
\AgdaOperator{\AgdaFunction{×}}\AgdaSpace{}%
\AgdaDatatype{Term}\AgdaSymbol{)}\AgdaSpace{}%
\AgdaOperator{\AgdaDatatype{⊎}}\AgdaSpace{}%
\AgdaSymbol{(}\AgdaFunction{Prec}\AgdaSpace{}%
\AgdaOperator{\AgdaFunction{×}}\AgdaSpace{}%
\AgdaDatatype{Dir}\AgdaSpace{}%
\AgdaOperator{\AgdaFunction{×}}\AgdaSpace{}%
\AgdaDatatype{Term}\AgdaSpace{}%
\AgdaOperator{\AgdaFunction{×}}\AgdaSpace{}%
\AgdaDatatype{Term}\AgdaSymbol{)}\<%
\\
%
\\[\AgdaEmptyExtraSkip]%
\>[0]\AgdaFunction{LR-ctx}\AgdaSpace{}%
\AgdaSymbol{:}\AgdaSpace{}%
\AgdaFunction{Context}\<%
\\
\>[0]\AgdaFunction{LR-ctx}\AgdaSpace{}%
\AgdaSymbol{=}\AgdaSpace{}%
\AgdaFunction{LR-type}\AgdaSpace{}%
\AgdaOperator{\AgdaInductiveConstructor{∷}}\AgdaSpace{}%
\AgdaInductiveConstructor{[]}\<%
\\
%
\\[\AgdaEmptyExtraSkip]%
\>[0]\AgdaOperator{\AgdaFunction{\AgdaUnderscore{}∣\AgdaUnderscore{}ˢ⊑ᴸᴿₜ\AgdaUnderscore{}⦂\AgdaUnderscore{}}}\AgdaSpace{}%
\AgdaSymbol{:}\AgdaSpace{}%
\AgdaDatatype{Dir}\AgdaSpace{}%
\AgdaSymbol{→}\AgdaSpace{}%
\AgdaDatatype{Term}\AgdaSpace{}%
\AgdaSymbol{→}\AgdaSpace{}%
\AgdaDatatype{Term}\AgdaSpace{}%
\AgdaSymbol{→}\AgdaSpace{}%
\AgdaSymbol{∀\{}\AgdaBound{A}\AgdaSymbol{\}\{}\AgdaBound{A′}\AgdaSymbol{\}}\AgdaSpace{}%
\AgdaSymbol{(}\AgdaBound{c}\AgdaSpace{}%
\AgdaSymbol{:}\AgdaSpace{}%
\AgdaBound{A}\AgdaSpace{}%
\AgdaOperator{\AgdaDatatype{⊑}}\AgdaSpace{}%
\AgdaBound{A′}\AgdaSymbol{)}\AgdaSpace{}%
\AgdaSymbol{→}\AgdaSpace{}%
\AgdaRecord{Setˢ}\AgdaSpace{}%
\AgdaFunction{LR-ctx}\AgdaSpace{}%
\AgdaSymbol{(}\AgdaInductiveConstructor{cons}\AgdaSpace{}%
\AgdaInductiveConstructor{Now}\AgdaSpace{}%
\AgdaInductiveConstructor{∅}\AgdaSymbol{)}\<%
\\
\>[0]\AgdaBound{dir}\AgdaSpace{}%
\AgdaOperator{\AgdaFunction{∣}}\AgdaSpace{}%
\AgdaBound{M}\AgdaSpace{}%
\AgdaOperator{\AgdaFunction{ˢ⊑ᴸᴿₜ}}\AgdaSpace{}%
\AgdaBound{M′}\AgdaSpace{}%
\AgdaOperator{\AgdaFunction{⦂}}\AgdaSpace{}%
\AgdaBound{c}\AgdaSpace{}%
\AgdaSymbol{=}\AgdaSpace{}%
\AgdaSymbol{(}\AgdaInductiveConstructor{inj₂}\AgdaSpace{}%
\AgdaSymbol{((\AgdaUnderscore{}}\AgdaSpace{}%
\AgdaOperator{\AgdaInductiveConstructor{,}}\AgdaSpace{}%
\AgdaSymbol{\AgdaUnderscore{}}\AgdaSpace{}%
\AgdaOperator{\AgdaInductiveConstructor{,}}\AgdaSpace{}%
\AgdaBound{c}\AgdaSymbol{)}\AgdaSpace{}%
\AgdaOperator{\AgdaInductiveConstructor{,}}\AgdaSpace{}%
\AgdaBound{dir}\AgdaSpace{}%
\AgdaOperator{\AgdaInductiveConstructor{,}}\AgdaSpace{}%
\AgdaBound{M}\AgdaSpace{}%
\AgdaOperator{\AgdaInductiveConstructor{,}}\AgdaSpace{}%
\AgdaBound{M′}\AgdaSymbol{))}\AgdaSpace{}%
\AgdaOperator{\AgdaFunction{∈}}\AgdaSpace{}%
\AgdaInductiveConstructor{zeroˢ}\<%
\\
%
\\[\AgdaEmptyExtraSkip]%
\>[0]\AgdaOperator{\AgdaFunction{\AgdaUnderscore{}∣\AgdaUnderscore{}ˢ⊑ᴸᴿᵥ\AgdaUnderscore{}⦂\AgdaUnderscore{}}}\AgdaSpace{}%
\AgdaSymbol{:}\AgdaSpace{}%
\AgdaDatatype{Dir}\AgdaSpace{}%
\AgdaSymbol{→}\AgdaSpace{}%
\AgdaDatatype{Term}\AgdaSpace{}%
\AgdaSymbol{→}\AgdaSpace{}%
\AgdaDatatype{Term}\AgdaSpace{}%
\AgdaSymbol{→}\AgdaSpace{}%
\AgdaSymbol{∀\{}\AgdaBound{A}\AgdaSymbol{\}\{}\AgdaBound{A′}\AgdaSymbol{\}}\AgdaSpace{}%
\AgdaSymbol{(}\AgdaBound{c}\AgdaSpace{}%
\AgdaSymbol{:}\AgdaSpace{}%
\AgdaBound{A}\AgdaSpace{}%
\AgdaOperator{\AgdaDatatype{⊑}}\AgdaSpace{}%
\AgdaBound{A′}\AgdaSymbol{)}\AgdaSpace{}%
\AgdaSymbol{→}\AgdaSpace{}%
\AgdaRecord{Setˢ}\AgdaSpace{}%
\AgdaFunction{LR-ctx}\AgdaSpace{}%
\AgdaSymbol{(}\AgdaInductiveConstructor{cons}\AgdaSpace{}%
\AgdaInductiveConstructor{Now}\AgdaSpace{}%
\AgdaInductiveConstructor{∅}\AgdaSymbol{)}\<%
\\
\>[0]\AgdaBound{dir}\AgdaSpace{}%
\AgdaOperator{\AgdaFunction{∣}}\AgdaSpace{}%
\AgdaBound{V}\AgdaSpace{}%
\AgdaOperator{\AgdaFunction{ˢ⊑ᴸᴿᵥ}}\AgdaSpace{}%
\AgdaBound{V′}\AgdaSpace{}%
\AgdaOperator{\AgdaFunction{⦂}}\AgdaSpace{}%
\AgdaBound{c}\AgdaSpace{}%
\AgdaSymbol{=}\AgdaSpace{}%
\AgdaSymbol{(}\AgdaInductiveConstructor{inj₁}\AgdaSpace{}%
\AgdaSymbol{((\AgdaUnderscore{}}\AgdaSpace{}%
\AgdaOperator{\AgdaInductiveConstructor{,}}\AgdaSpace{}%
\AgdaSymbol{\AgdaUnderscore{}}\AgdaSpace{}%
\AgdaOperator{\AgdaInductiveConstructor{,}}\AgdaSpace{}%
\AgdaBound{c}\AgdaSymbol{)}\AgdaSpace{}%
\AgdaOperator{\AgdaInductiveConstructor{,}}\AgdaSpace{}%
\AgdaBound{dir}\AgdaSpace{}%
\AgdaOperator{\AgdaInductiveConstructor{,}}\AgdaSpace{}%
\AgdaBound{V}\AgdaSpace{}%
\AgdaOperator{\AgdaInductiveConstructor{,}}\AgdaSpace{}%
\AgdaBound{V′}\AgdaSymbol{))}\AgdaSpace{}%
\AgdaOperator{\AgdaFunction{∈}}\AgdaSpace{}%
\AgdaInductiveConstructor{zeroˢ}\<%
\end{code}
\begin{code}[hide]%
\>[0]\AgdaKeyword{instance}\<%
\\
\>[0][@{}l@{\AgdaIndent{0}}]%
\>[2]\AgdaFunction{TermInhabited}\AgdaSpace{}%
\AgdaSymbol{:}\AgdaSpace{}%
\AgdaRecord{Inhabited}\AgdaSpace{}%
\AgdaDatatype{Term}\<%
\\
%
\>[2]\AgdaFunction{TermInhabited}\AgdaSpace{}%
\AgdaSymbol{=}\AgdaSpace{}%
\AgdaKeyword{record}\AgdaSpace{}%
\AgdaSymbol{\{}\AgdaSpace{}%
\AgdaField{elt}\AgdaSpace{}%
\AgdaSymbol{=}\AgdaSpace{}%
\AgdaOperator{\AgdaInductiveConstructor{`}}\AgdaSpace{}%
\AgdaNumber{0}\AgdaSpace{}%
\AgdaSymbol{\}}\<%
\end{code}

\begin{figure}[tbp]
\begin{code}%
\>[0]\AgdaFunction{LRₜ}\AgdaSpace{}%
\AgdaSymbol{:}\AgdaSpace{}%
\AgdaFunction{Prec}\AgdaSpace{}%
\AgdaSymbol{→}\AgdaSpace{}%
\AgdaDatatype{Dir}\AgdaSpace{}%
\AgdaSymbol{→}\AgdaSpace{}%
\AgdaDatatype{Term}\AgdaSpace{}%
\AgdaSymbol{→}\AgdaSpace{}%
\AgdaDatatype{Term}\AgdaSpace{}%
\AgdaSymbol{→}\AgdaSpace{}%
\AgdaRecord{Setˢ}\AgdaSpace{}%
\AgdaFunction{LR-ctx}\AgdaSpace{}%
\AgdaSymbol{(}\AgdaInductiveConstructor{cons}\AgdaSpace{}%
\AgdaInductiveConstructor{Later}\AgdaSpace{}%
\AgdaInductiveConstructor{∅}\AgdaSymbol{)}\<%
\\
\>[0]\AgdaFunction{LRᵥ}\AgdaSpace{}%
\AgdaSymbol{:}\AgdaSpace{}%
\AgdaFunction{Prec}\AgdaSpace{}%
\AgdaSymbol{→}\AgdaSpace{}%
\AgdaDatatype{Dir}\AgdaSpace{}%
\AgdaSymbol{→}\AgdaSpace{}%
\AgdaDatatype{Term}\AgdaSpace{}%
\AgdaSymbol{→}\AgdaSpace{}%
\AgdaDatatype{Term}\AgdaSpace{}%
\AgdaSymbol{→}\AgdaSpace{}%
\AgdaRecord{Setˢ}\AgdaSpace{}%
\AgdaFunction{LR-ctx}\AgdaSpace{}%
\AgdaSymbol{(}\AgdaInductiveConstructor{cons}\AgdaSpace{}%
\AgdaInductiveConstructor{Later}\AgdaSpace{}%
\AgdaInductiveConstructor{∅}\AgdaSymbol{)}\<%
\\
%
\\[\AgdaEmptyExtraSkip]%
\>[0]\AgdaFunction{LRₜ}\AgdaSpace{}%
\AgdaSymbol{(}\AgdaBound{A}\AgdaSpace{}%
\AgdaOperator{\AgdaInductiveConstructor{,}}\AgdaSpace{}%
\AgdaBound{A′}\AgdaSpace{}%
\AgdaOperator{\AgdaInductiveConstructor{,}}\AgdaSpace{}%
\AgdaBound{c}\AgdaSymbol{)}\AgdaSpace{}%
\AgdaInductiveConstructor{≼}\AgdaSpace{}%
\AgdaBound{M}\AgdaSpace{}%
\AgdaBound{M′}\AgdaSpace{}%
\AgdaSymbol{=}\<%
\\
\>[0][@{}l@{\AgdaIndent{0}}]%
\>[3]\AgdaSymbol{(}\AgdaFunction{∃ˢ[}\AgdaSpace{}%
\AgdaBound{N}\AgdaSpace{}%
\AgdaFunction{]}\AgdaSpace{}%
\AgdaSymbol{(}\AgdaBound{M}\AgdaSpace{}%
\AgdaOperator{\AgdaDatatype{⟶}}\AgdaSpace{}%
\AgdaBound{N}\AgdaSymbol{)}\AgdaOperator{\AgdaFunction{ˢ}}\AgdaSpace{}%
\AgdaOperator{\AgdaFunction{×ˢ}}\AgdaSpace{}%
\AgdaFunction{▷ˢ}\AgdaSpace{}%
\AgdaSymbol{(}\AgdaInductiveConstructor{≼}\AgdaSpace{}%
\AgdaOperator{\AgdaFunction{∣}}\AgdaSpace{}%
\AgdaBound{N}\AgdaSpace{}%
\AgdaOperator{\AgdaFunction{ˢ⊑ᴸᴿₜ}}\AgdaSpace{}%
\AgdaBound{M′}\AgdaSpace{}%
\AgdaOperator{\AgdaFunction{⦂}}\AgdaSpace{}%
\AgdaBound{c}\AgdaSymbol{))}\<%
\\
%
\>[3]\AgdaOperator{\AgdaFunction{⊎ˢ}}\AgdaSpace{}%
\AgdaSymbol{(}\AgdaBound{M′}\AgdaSpace{}%
\AgdaOperator{\AgdaDatatype{↠}}\AgdaSpace{}%
\AgdaInductiveConstructor{blame}\AgdaSymbol{)}\AgdaOperator{\AgdaFunction{ˢ}}\<%
\\
%
\>[3]\AgdaOperator{\AgdaFunction{⊎ˢ}}\AgdaSpace{}%
\AgdaSymbol{((}\AgdaDatatype{Value}\AgdaSpace{}%
\AgdaBound{M}\AgdaSymbol{)}\AgdaOperator{\AgdaFunction{ˢ}}\AgdaSpace{}%
\AgdaOperator{\AgdaFunction{×ˢ}}\AgdaSpace{}%
\AgdaSymbol{(}\AgdaFunction{∃ˢ[}\AgdaSpace{}%
\AgdaBound{V′}\AgdaSpace{}%
\AgdaFunction{]}\AgdaSpace{}%
\AgdaSymbol{(}\AgdaBound{M′}\AgdaSpace{}%
\AgdaOperator{\AgdaDatatype{↠}}\AgdaSpace{}%
\AgdaBound{V′}\AgdaSymbol{)}\AgdaOperator{\AgdaFunction{ˢ}}\AgdaSpace{}%
\AgdaOperator{\AgdaFunction{×ˢ}}\AgdaSpace{}%
\AgdaSymbol{(}\AgdaDatatype{Value}\AgdaSpace{}%
\AgdaBound{V′}\AgdaSymbol{)}\AgdaOperator{\AgdaFunction{ˢ}}\AgdaSpace{}%
\AgdaOperator{\AgdaFunction{×ˢ}}\AgdaSpace{}%
\AgdaSymbol{(}\AgdaFunction{LRᵥ}\AgdaSpace{}%
\AgdaSymbol{(\AgdaUnderscore{}}\AgdaSpace{}%
\AgdaOperator{\AgdaInductiveConstructor{,}}\AgdaSpace{}%
\AgdaSymbol{\AgdaUnderscore{}}\AgdaSpace{}%
\AgdaOperator{\AgdaInductiveConstructor{,}}\AgdaSpace{}%
\AgdaBound{c}\AgdaSymbol{)}\AgdaSpace{}%
\AgdaInductiveConstructor{≼}\AgdaSpace{}%
\AgdaBound{M}\AgdaSpace{}%
\AgdaBound{V′}\AgdaSymbol{)))}\<%
\\
\>[0]\AgdaFunction{LRₜ}\AgdaSpace{}%
\AgdaSymbol{(}\AgdaBound{A}\AgdaSpace{}%
\AgdaOperator{\AgdaInductiveConstructor{,}}\AgdaSpace{}%
\AgdaBound{A′}\AgdaSpace{}%
\AgdaOperator{\AgdaInductiveConstructor{,}}\AgdaSpace{}%
\AgdaBound{c}\AgdaSymbol{)}\AgdaSpace{}%
\AgdaInductiveConstructor{≽}\AgdaSpace{}%
\AgdaBound{M}\AgdaSpace{}%
\AgdaBound{M′}\AgdaSpace{}%
\AgdaSymbol{=}\<%
\\
\>[0][@{}l@{\AgdaIndent{0}}]%
\>[3]\AgdaSymbol{(}\AgdaFunction{∃ˢ[}\AgdaSpace{}%
\AgdaBound{N′}\AgdaSpace{}%
\AgdaFunction{]}\AgdaSpace{}%
\AgdaSymbol{(}\AgdaBound{M′}\AgdaSpace{}%
\AgdaOperator{\AgdaDatatype{⟶}}\AgdaSpace{}%
\AgdaBound{N′}\AgdaSymbol{)}\AgdaOperator{\AgdaFunction{ˢ}}\AgdaSpace{}%
\AgdaOperator{\AgdaFunction{×ˢ}}\AgdaSpace{}%
\AgdaFunction{▷ˢ}\AgdaSpace{}%
\AgdaSymbol{(}\AgdaInductiveConstructor{≽}\AgdaSpace{}%
\AgdaOperator{\AgdaFunction{∣}}\AgdaSpace{}%
\AgdaBound{M}\AgdaSpace{}%
\AgdaOperator{\AgdaFunction{ˢ⊑ᴸᴿₜ}}\AgdaSpace{}%
\AgdaBound{N′}\AgdaSpace{}%
\AgdaOperator{\AgdaFunction{⦂}}\AgdaSpace{}%
\AgdaBound{c}\AgdaSymbol{))}\<%
\\
%
\>[3]\AgdaOperator{\AgdaFunction{⊎ˢ}}\AgdaSpace{}%
\AgdaSymbol{(}\AgdaDatatype{Blame}\AgdaSpace{}%
\AgdaBound{M′}\AgdaSymbol{)}\AgdaOperator{\AgdaFunction{ˢ}}\<%
\\
%
\>[3]\AgdaOperator{\AgdaFunction{⊎ˢ}}\AgdaSpace{}%
\AgdaSymbol{((}\AgdaDatatype{Value}\AgdaSpace{}%
\AgdaBound{M′}\AgdaSymbol{)}\AgdaOperator{\AgdaFunction{ˢ}}\AgdaSpace{}%
\AgdaOperator{\AgdaFunction{×ˢ}}\AgdaSpace{}%
\AgdaSymbol{(}\AgdaFunction{∃ˢ[}\AgdaSpace{}%
\AgdaBound{V}\AgdaSpace{}%
\AgdaFunction{]}\AgdaSpace{}%
\AgdaSymbol{(}\AgdaBound{M}\AgdaSpace{}%
\AgdaOperator{\AgdaDatatype{↠}}\AgdaSpace{}%
\AgdaBound{V}\AgdaSymbol{)}\AgdaOperator{\AgdaFunction{ˢ}}\AgdaSpace{}%
\AgdaOperator{\AgdaFunction{×ˢ}}\AgdaSpace{}%
\AgdaSymbol{(}\AgdaDatatype{Value}\AgdaSpace{}%
\AgdaBound{V}\AgdaSymbol{)}\AgdaOperator{\AgdaFunction{ˢ}}\AgdaSpace{}%
\AgdaOperator{\AgdaFunction{×ˢ}}\AgdaSpace{}%
\AgdaSymbol{(}\AgdaFunction{LRᵥ}\AgdaSpace{}%
\AgdaSymbol{(\AgdaUnderscore{}}\AgdaSpace{}%
\AgdaOperator{\AgdaInductiveConstructor{,}}\AgdaSpace{}%
\AgdaSymbol{\AgdaUnderscore{}}\AgdaSpace{}%
\AgdaOperator{\AgdaInductiveConstructor{,}}\AgdaSpace{}%
\AgdaBound{c}\AgdaSymbol{)}\AgdaSpace{}%
\AgdaInductiveConstructor{≽}\AgdaSpace{}%
\AgdaBound{V}\AgdaSpace{}%
\AgdaBound{M′}\AgdaSymbol{)))}\<%
\\
%
\\[\AgdaEmptyExtraSkip]%
\>[0]\AgdaFunction{LRᵥ}\AgdaSpace{}%
\AgdaSymbol{(}\AgdaDottedPattern{\AgdaSymbol{.(}}\AgdaDottedPattern{\AgdaOperator{\AgdaInductiveConstructor{\$ₜ}}}\AgdaSpace{}%
\AgdaDottedPattern{\AgdaBound{ι}}\AgdaDottedPattern{\AgdaSymbol{)}}\AgdaSpace{}%
\AgdaOperator{\AgdaInductiveConstructor{,}}\AgdaSpace{}%
\AgdaDottedPattern{\AgdaSymbol{.(}}\AgdaDottedPattern{\AgdaOperator{\AgdaInductiveConstructor{\$ₜ}}}\AgdaSpace{}%
\AgdaDottedPattern{\AgdaBound{ι}}\AgdaDottedPattern{\AgdaSymbol{)}}\AgdaSpace{}%
\AgdaOperator{\AgdaInductiveConstructor{,}}\AgdaSpace{}%
\AgdaInductiveConstructor{base⊑}\AgdaSymbol{\{}\AgdaBound{ι}\AgdaSymbol{\})}\AgdaSpace{}%
\AgdaBound{dir}\AgdaSpace{}%
\AgdaSymbol{(}\AgdaInductiveConstructor{\$}\AgdaSpace{}%
\AgdaBound{c}\AgdaSymbol{)}\AgdaSpace{}%
\AgdaSymbol{(}\AgdaInductiveConstructor{\$}\AgdaSpace{}%
\AgdaBound{c′}\AgdaSymbol{)}\AgdaSpace{}%
\AgdaSymbol{=}\AgdaSpace{}%
\AgdaSymbol{(}\AgdaBound{c}\AgdaSpace{}%
\AgdaOperator{\AgdaDatatype{≡}}\AgdaSpace{}%
\AgdaBound{c′}\AgdaSymbol{)}\AgdaSpace{}%
\AgdaOperator{\AgdaFunction{ˢ}}\<%
\\
\>[0]\AgdaCatchallClause{\AgdaFunction{LRᵥ}}\AgdaSpace{}%
\AgdaCatchallClause{\AgdaSymbol{(}}\AgdaDottedPattern{\AgdaCatchallClause{\AgdaSymbol{.(}}}\AgdaDottedPattern{\AgdaCatchallClause{\AgdaOperator{\AgdaInductiveConstructor{\$ₜ}}}}\AgdaSpace{}%
\AgdaDottedPattern{\AgdaCatchallClause{\AgdaBound{ι}}}\AgdaDottedPattern{\AgdaCatchallClause{\AgdaSymbol{)}}}\AgdaSpace{}%
\AgdaCatchallClause{\AgdaOperator{\AgdaInductiveConstructor{,}}}\AgdaSpace{}%
\AgdaDottedPattern{\AgdaCatchallClause{\AgdaSymbol{.(}}}\AgdaDottedPattern{\AgdaCatchallClause{\AgdaOperator{\AgdaInductiveConstructor{\$ₜ}}}}\AgdaSpace{}%
\AgdaDottedPattern{\AgdaCatchallClause{\AgdaBound{ι}}}\AgdaDottedPattern{\AgdaCatchallClause{\AgdaSymbol{)}}}\AgdaSpace{}%
\AgdaCatchallClause{\AgdaOperator{\AgdaInductiveConstructor{,}}}\AgdaSpace{}%
\AgdaCatchallClause{\AgdaInductiveConstructor{base⊑}}\AgdaCatchallClause{\AgdaSymbol{\{}}\AgdaCatchallClause{\AgdaBound{ι}}\AgdaCatchallClause{\AgdaSymbol{\})}}\AgdaSpace{}%
\AgdaCatchallClause{\AgdaBound{dir}}\AgdaSpace{}%
\AgdaCatchallClause{\AgdaBound{V}}\AgdaSpace{}%
\AgdaCatchallClause{\AgdaBound{V′}}\AgdaSpace{}%
\AgdaSymbol{=}\AgdaSpace{}%
\AgdaDatatype{⊥}\AgdaSpace{}%
\AgdaOperator{\AgdaFunction{ˢ}}\<%
\\
\>[0]\AgdaFunction{LRᵥ}%
\>[957I]\AgdaSymbol{(}\AgdaDottedPattern{\AgdaSymbol{.(}}\AgdaDottedPattern{\AgdaBound{A}}\AgdaSpace{}%
\AgdaDottedPattern{\AgdaOperator{\AgdaInductiveConstructor{⇒}}}\AgdaSpace{}%
\AgdaDottedPattern{\AgdaBound{B}}\AgdaDottedPattern{\AgdaSymbol{)}}\AgdaSpace{}%
\AgdaOperator{\AgdaInductiveConstructor{,}}\AgdaSpace{}%
\AgdaDottedPattern{\AgdaSymbol{.(}}\AgdaDottedPattern{\AgdaBound{A′}}\AgdaSpace{}%
\AgdaDottedPattern{\AgdaOperator{\AgdaInductiveConstructor{⇒}}}\AgdaSpace{}%
\AgdaDottedPattern{\AgdaBound{B′}}\AgdaDottedPattern{\AgdaSymbol{)}}\AgdaSpace{}%
\AgdaOperator{\AgdaInductiveConstructor{,}}\AgdaSpace{}%
\AgdaInductiveConstructor{fun⊑}\AgdaSymbol{\{}\AgdaBound{A}\AgdaSymbol{\}\{}\AgdaBound{B}\AgdaSymbol{\}\{}\AgdaBound{A′}\AgdaSymbol{\}\{}\AgdaBound{B′}\AgdaSymbol{\}}\AgdaSpace{}%
\AgdaBound{A⊑A′}\AgdaSpace{}%
\AgdaBound{B⊑B′}\AgdaSymbol{)}\AgdaSpace{}%
\AgdaBound{dir}\AgdaSpace{}%
\AgdaSymbol{(}\AgdaInductiveConstructor{ƛ}\AgdaSpace{}%
\AgdaBound{N}\AgdaSymbol{)(}\AgdaInductiveConstructor{ƛ}\AgdaSpace{}%
\AgdaBound{N′}\AgdaSymbol{)}\AgdaSpace{}%
\AgdaSymbol{=}\<%
\\
\>[.][@{}l@{}]\<[957I]%
\>[4]\AgdaFunction{∀ˢ[}\AgdaSpace{}%
\AgdaBound{W}\AgdaSpace{}%
\AgdaFunction{]}\AgdaSpace{}%
\AgdaFunction{∀ˢ[}%
\>[976I]\AgdaBound{W′}\AgdaSpace{}%
\AgdaFunction{]}\AgdaSpace{}%
\AgdaFunction{▷ˢ}\AgdaSpace{}%
\AgdaSymbol{(}\AgdaBound{dir}\AgdaSpace{}%
\AgdaOperator{\AgdaFunction{∣}}\AgdaSpace{}%
\AgdaBound{W}\AgdaSpace{}%
\AgdaOperator{\AgdaFunction{ˢ⊑ᴸᴿᵥ}}\AgdaSpace{}%
\AgdaBound{W′}\AgdaSpace{}%
\AgdaOperator{\AgdaFunction{⦂}}\AgdaSpace{}%
\AgdaBound{A⊑A′}\AgdaSymbol{)}\<%
\\
\>[976I][@{}l@{\AgdaIndent{0}}]%
\>[18]\AgdaOperator{\AgdaFunction{→ˢ}}\AgdaSpace{}%
\AgdaFunction{▷ˢ}\AgdaSpace{}%
\AgdaSymbol{(}\AgdaBound{dir}\AgdaSpace{}%
\AgdaOperator{\AgdaFunction{∣}}\AgdaSpace{}%
\AgdaSymbol{(}\AgdaBound{N}\AgdaSpace{}%
\AgdaOperator{\AgdaFunction{[}}\AgdaSpace{}%
\AgdaBound{W}\AgdaSpace{}%
\AgdaOperator{\AgdaFunction{]}}\AgdaSymbol{)}\AgdaSpace{}%
\AgdaOperator{\AgdaFunction{ˢ⊑ᴸᴿₜ}}\AgdaSpace{}%
\AgdaSymbol{(}\AgdaBound{N′}\AgdaSpace{}%
\AgdaOperator{\AgdaFunction{[}}\AgdaSpace{}%
\AgdaBound{W′}\AgdaSpace{}%
\AgdaOperator{\AgdaFunction{]}}\AgdaSymbol{)}\AgdaSpace{}%
\AgdaOperator{\AgdaFunction{⦂}}\AgdaSpace{}%
\AgdaBound{B⊑B′}\AgdaSymbol{)}\<%
\\
\>[0]\AgdaCatchallClause{\AgdaFunction{LRᵥ}}\AgdaSpace{}%
\AgdaCatchallClause{\AgdaSymbol{(}}\AgdaDottedPattern{\AgdaCatchallClause{\AgdaSymbol{.(}}}\AgdaDottedPattern{\AgdaCatchallClause{\AgdaBound{A}}}\AgdaSpace{}%
\AgdaDottedPattern{\AgdaCatchallClause{\AgdaOperator{\AgdaInductiveConstructor{⇒}}}}\AgdaSpace{}%
\AgdaDottedPattern{\AgdaCatchallClause{\AgdaBound{B}}}\AgdaDottedPattern{\AgdaCatchallClause{\AgdaSymbol{)}}}\AgdaSpace{}%
\AgdaCatchallClause{\AgdaOperator{\AgdaInductiveConstructor{,}}}\AgdaSpace{}%
\AgdaDottedPattern{\AgdaCatchallClause{\AgdaSymbol{.(}}}\AgdaDottedPattern{\AgdaCatchallClause{\AgdaBound{A′}}}\AgdaSpace{}%
\AgdaDottedPattern{\AgdaCatchallClause{\AgdaOperator{\AgdaInductiveConstructor{⇒}}}}\AgdaSpace{}%
\AgdaDottedPattern{\AgdaCatchallClause{\AgdaBound{B′}}}\AgdaDottedPattern{\AgdaCatchallClause{\AgdaSymbol{)}}}\AgdaSpace{}%
\AgdaCatchallClause{\AgdaOperator{\AgdaInductiveConstructor{,}}}\AgdaSpace{}%
\AgdaCatchallClause{\AgdaInductiveConstructor{fun⊑}}\AgdaCatchallClause{\AgdaSymbol{\{}}\AgdaCatchallClause{\AgdaBound{A}}\AgdaCatchallClause{\AgdaSymbol{\}\{}}\AgdaCatchallClause{\AgdaBound{B}}\AgdaCatchallClause{\AgdaSymbol{\}\{}}\AgdaCatchallClause{\AgdaBound{A′}}\AgdaCatchallClause{\AgdaSymbol{\}\{}}\AgdaCatchallClause{\AgdaBound{B′}}\AgdaCatchallClause{\AgdaSymbol{\}}}\AgdaSpace{}%
\AgdaCatchallClause{\AgdaBound{A⊑A′}}\AgdaSpace{}%
\AgdaCatchallClause{\AgdaBound{B⊑B′}}\AgdaCatchallClause{\AgdaSymbol{)}}\AgdaSpace{}%
\AgdaCatchallClause{\AgdaBound{dir}}\AgdaSpace{}%
\AgdaCatchallClause{\AgdaBound{V}}\AgdaSpace{}%
\AgdaCatchallClause{\AgdaBound{V′}}\AgdaSpace{}%
\AgdaSymbol{=}\AgdaSpace{}%
\AgdaDatatype{⊥}\AgdaSpace{}%
\AgdaOperator{\AgdaFunction{ˢ}}\<%
\\
\>[0]\AgdaFunction{LRᵥ}%
\>[1017I]\AgdaSymbol{(}\AgdaDottedPattern{\AgdaSymbol{.}}\AgdaDottedPattern{\AgdaInductiveConstructor{★}}\AgdaSpace{}%
\AgdaOperator{\AgdaInductiveConstructor{,}}\AgdaSpace{}%
\AgdaDottedPattern{\AgdaSymbol{.}}\AgdaDottedPattern{\AgdaInductiveConstructor{★}}\AgdaSpace{}%
\AgdaOperator{\AgdaInductiveConstructor{,}}\AgdaSpace{}%
\AgdaInductiveConstructor{unk⊑unk}\AgdaSymbol{)}\AgdaSpace{}%
\AgdaBound{dir}\AgdaSpace{}%
\AgdaSymbol{(}\AgdaBound{V}\AgdaSpace{}%
\AgdaOperator{\AgdaInductiveConstructor{⟨}}\AgdaSpace{}%
\AgdaBound{G}\AgdaSpace{}%
\AgdaOperator{\AgdaInductiveConstructor{!⟩}}\AgdaSymbol{)}\AgdaSpace{}%
\AgdaSymbol{(}\AgdaBound{V′}\AgdaSpace{}%
\AgdaOperator{\AgdaInductiveConstructor{⟨}}\AgdaSpace{}%
\AgdaBound{H}\AgdaSpace{}%
\AgdaOperator{\AgdaInductiveConstructor{!⟩}}\AgdaSymbol{)}\<%
\\
\>[.][@{}l@{}]\<[1017I]%
\>[4]\AgdaKeyword{with}\AgdaSpace{}%
\AgdaBound{G}\AgdaSpace{}%
\AgdaOperator{\AgdaFunction{≡ᵍ}}\AgdaSpace{}%
\AgdaBound{H}\<%
\\
\>[0]\AgdaSymbol{...}\AgdaSpace{}%
\AgdaSymbol{|}\AgdaSpace{}%
\AgdaInductiveConstructor{yes}\AgdaSpace{}%
\AgdaInductiveConstructor{refl}\AgdaSpace{}%
\AgdaSymbol{=}\AgdaSpace{}%
\AgdaSymbol{(}\AgdaDatatype{Value}\AgdaSpace{}%
\AgdaBound{V}\AgdaSymbol{)}\AgdaOperator{\AgdaFunction{ˢ}}\AgdaSpace{}%
\AgdaOperator{\AgdaFunction{×ˢ}}\AgdaSpace{}%
\AgdaSymbol{(}\AgdaDatatype{Value}\AgdaSpace{}%
\AgdaBound{V′}\AgdaSymbol{)}\AgdaOperator{\AgdaFunction{ˢ}}\AgdaSpace{}%
\AgdaOperator{\AgdaFunction{×ˢ}}\AgdaSpace{}%
\AgdaSymbol{(}\AgdaFunction{▷ˢ}\AgdaSpace{}%
\AgdaSymbol{(}\AgdaBound{dir}\AgdaSpace{}%
\AgdaOperator{\AgdaFunction{∣}}\AgdaSpace{}%
\AgdaBound{V}\AgdaSpace{}%
\AgdaOperator{\AgdaFunction{ˢ⊑ᴸᴿᵥ}}\AgdaSpace{}%
\AgdaBound{V′}\AgdaSpace{}%
\AgdaOperator{\AgdaFunction{⦂}}\AgdaSpace{}%
\AgdaFunction{Refl⊑}\AgdaSymbol{\{}\AgdaOperator{\AgdaFunction{⌈}}\AgdaSpace{}%
\AgdaBound{G}\AgdaSpace{}%
\AgdaOperator{\AgdaFunction{⌉}}\AgdaSymbol{\}))}\<%
\\
\>[0]\AgdaSymbol{...}\AgdaSpace{}%
\AgdaSymbol{|}\AgdaSpace{}%
\AgdaInductiveConstructor{no}\AgdaSpace{}%
\AgdaBound{neq}\AgdaSpace{}%
\AgdaSymbol{=}\AgdaSpace{}%
\AgdaDatatype{⊥}\AgdaSpace{}%
\AgdaOperator{\AgdaFunction{ˢ}}\<%
\\
\>[0]\AgdaCatchallClause{\AgdaFunction{LRᵥ}}\AgdaSpace{}%
\AgdaCatchallClause{\AgdaSymbol{(}}\AgdaDottedPattern{\AgdaCatchallClause{\AgdaSymbol{.}}}\AgdaDottedPattern{\AgdaCatchallClause{\AgdaInductiveConstructor{★}}}\AgdaSpace{}%
\AgdaCatchallClause{\AgdaOperator{\AgdaInductiveConstructor{,}}}\AgdaSpace{}%
\AgdaDottedPattern{\AgdaCatchallClause{\AgdaSymbol{.}}}\AgdaDottedPattern{\AgdaCatchallClause{\AgdaInductiveConstructor{★}}}\AgdaSpace{}%
\AgdaCatchallClause{\AgdaOperator{\AgdaInductiveConstructor{,}}}\AgdaSpace{}%
\AgdaCatchallClause{\AgdaInductiveConstructor{unk⊑unk}}\AgdaCatchallClause{\AgdaSymbol{)}}\AgdaSpace{}%
\AgdaCatchallClause{\AgdaBound{dir}}\AgdaSpace{}%
\AgdaCatchallClause{\AgdaBound{V}}\AgdaSpace{}%
\AgdaCatchallClause{\AgdaBound{V′}}\AgdaSpace{}%
\AgdaSymbol{=}\AgdaSpace{}%
\AgdaDatatype{⊥}\AgdaSpace{}%
\AgdaOperator{\AgdaFunction{ˢ}}\<%
\\
\>[0]\AgdaFunction{LRᵥ}%
\>[1071I]\AgdaSymbol{(}\AgdaDottedPattern{\AgdaSymbol{.}}\AgdaDottedPattern{\AgdaInductiveConstructor{★}}\AgdaSpace{}%
\AgdaOperator{\AgdaInductiveConstructor{,}}\AgdaSpace{}%
\AgdaDottedPattern{\AgdaSymbol{.}}\AgdaDottedPattern{\AgdaBound{A′}}\AgdaSpace{}%
\AgdaOperator{\AgdaInductiveConstructor{,}}\AgdaSpace{}%
\AgdaInductiveConstructor{unk⊑}\AgdaSymbol{\{}\AgdaBound{H}\AgdaSymbol{\}\{}\AgdaBound{A′}\AgdaSymbol{\}}\AgdaSpace{}%
\AgdaBound{d}\AgdaSymbol{)}\AgdaSpace{}%
\AgdaInductiveConstructor{≼}\AgdaSpace{}%
\AgdaSymbol{(}\AgdaBound{V}\AgdaSpace{}%
\AgdaOperator{\AgdaInductiveConstructor{⟨}}\AgdaSpace{}%
\AgdaBound{G}\AgdaSpace{}%
\AgdaOperator{\AgdaInductiveConstructor{!⟩}}\AgdaSymbol{)}\AgdaSpace{}%
\AgdaBound{V′}\<%
\\
\>[.][@{}l@{}]\<[1071I]%
\>[4]\AgdaKeyword{with}\AgdaSpace{}%
\AgdaBound{G}\AgdaSpace{}%
\AgdaOperator{\AgdaFunction{≡ᵍ}}\AgdaSpace{}%
\AgdaBound{H}\<%
\\
\>[0]\AgdaSymbol{...}\AgdaSpace{}%
\AgdaSymbol{|}\AgdaSpace{}%
\AgdaInductiveConstructor{yes}\AgdaSpace{}%
\AgdaInductiveConstructor{refl}\AgdaSpace{}%
\AgdaSymbol{=}\AgdaSpace{}%
\AgdaSymbol{(}\AgdaDatatype{Value}\AgdaSpace{}%
\AgdaBound{V}\AgdaSymbol{)}\AgdaOperator{\AgdaFunction{ˢ}}\AgdaSpace{}%
\AgdaOperator{\AgdaFunction{×ˢ}}\AgdaSpace{}%
\AgdaSymbol{(}\AgdaDatatype{Value}\AgdaSpace{}%
\AgdaBound{V′}\AgdaSymbol{)}\AgdaOperator{\AgdaFunction{ˢ}}\AgdaSpace{}%
\AgdaOperator{\AgdaFunction{×ˢ}}\AgdaSpace{}%
\AgdaFunction{▷ˢ}\AgdaSpace{}%
\AgdaSymbol{(}\AgdaInductiveConstructor{≼}\AgdaSpace{}%
\AgdaOperator{\AgdaFunction{∣}}\AgdaSpace{}%
\AgdaBound{V}\AgdaSpace{}%
\AgdaOperator{\AgdaFunction{ˢ⊑ᴸᴿᵥ}}\AgdaSpace{}%
\AgdaBound{V′}\AgdaSpace{}%
\AgdaOperator{\AgdaFunction{⦂}}\AgdaSpace{}%
\AgdaBound{d}\AgdaSymbol{)}\<%
\\
\>[0]\AgdaSymbol{...}\AgdaSpace{}%
\AgdaSymbol{|}\AgdaSpace{}%
\AgdaInductiveConstructor{no}\AgdaSpace{}%
\AgdaBound{neq}\AgdaSpace{}%
\AgdaSymbol{=}\AgdaSpace{}%
\AgdaDatatype{⊥}\AgdaSpace{}%
\AgdaOperator{\AgdaFunction{ˢ}}\<%
\\
\>[0]\AgdaFunction{LRᵥ}%
\>[1110I]\AgdaSymbol{(}\AgdaDottedPattern{\AgdaSymbol{.}}\AgdaDottedPattern{\AgdaInductiveConstructor{★}}\AgdaSpace{}%
\AgdaOperator{\AgdaInductiveConstructor{,}}\AgdaSpace{}%
\AgdaDottedPattern{\AgdaSymbol{.}}\AgdaDottedPattern{\AgdaBound{A′}}\AgdaSpace{}%
\AgdaOperator{\AgdaInductiveConstructor{,}}\AgdaSpace{}%
\AgdaInductiveConstructor{unk⊑}\AgdaSymbol{\{}\AgdaBound{H}\AgdaSymbol{\}\{}\AgdaBound{A′}\AgdaSymbol{\}}\AgdaSpace{}%
\AgdaBound{d}\AgdaSymbol{)}\AgdaSpace{}%
\AgdaInductiveConstructor{≽}\AgdaSpace{}%
\AgdaSymbol{(}\AgdaBound{V}\AgdaSpace{}%
\AgdaOperator{\AgdaInductiveConstructor{⟨}}\AgdaSpace{}%
\AgdaBound{G}\AgdaSpace{}%
\AgdaOperator{\AgdaInductiveConstructor{!⟩}}\AgdaSymbol{)}\AgdaSpace{}%
\AgdaBound{V′}\<%
\\
\>[.][@{}l@{}]\<[1110I]%
\>[4]\AgdaKeyword{with}\AgdaSpace{}%
\AgdaBound{G}\AgdaSpace{}%
\AgdaOperator{\AgdaFunction{≡ᵍ}}\AgdaSpace{}%
\AgdaBound{H}\<%
\\
\>[0]\AgdaSymbol{...}\AgdaSpace{}%
\AgdaSymbol{|}\AgdaSpace{}%
\AgdaInductiveConstructor{yes}\AgdaSpace{}%
\AgdaInductiveConstructor{refl}\AgdaSpace{}%
\AgdaSymbol{=}\AgdaSpace{}%
\AgdaSymbol{(}\AgdaDatatype{Value}\AgdaSpace{}%
\AgdaBound{V}\AgdaSymbol{)}\AgdaOperator{\AgdaFunction{ˢ}}\AgdaSpace{}%
\AgdaOperator{\AgdaFunction{×ˢ}}\AgdaSpace{}%
\AgdaSymbol{(}\AgdaDatatype{Value}\AgdaSpace{}%
\AgdaBound{V′}\AgdaSymbol{)}\AgdaOperator{\AgdaFunction{ˢ}}\AgdaSpace{}%
\AgdaOperator{\AgdaFunction{×ˢ}}\AgdaSpace{}%
\AgdaSymbol{(}\AgdaFunction{LRᵥ}\AgdaSpace{}%
\AgdaSymbol{(}\AgdaOperator{\AgdaFunction{⌈}}\AgdaSpace{}%
\AgdaBound{G}\AgdaSpace{}%
\AgdaOperator{\AgdaFunction{⌉}}\AgdaSpace{}%
\AgdaOperator{\AgdaInductiveConstructor{,}}\AgdaSpace{}%
\AgdaBound{A′}\AgdaSpace{}%
\AgdaOperator{\AgdaInductiveConstructor{,}}\AgdaSpace{}%
\AgdaBound{d}\AgdaSymbol{)}\AgdaSpace{}%
\AgdaInductiveConstructor{≽}\AgdaSpace{}%
\AgdaBound{V}\AgdaSpace{}%
\AgdaBound{V′}\AgdaSymbol{)}\<%
\\
\>[0]\AgdaSymbol{...}\AgdaSpace{}%
\AgdaSymbol{|}\AgdaSpace{}%
\AgdaInductiveConstructor{no}\AgdaSpace{}%
\AgdaBound{neq}\AgdaSpace{}%
\AgdaSymbol{=}\AgdaSpace{}%
\AgdaDatatype{⊥}\AgdaSpace{}%
\AgdaOperator{\AgdaFunction{ˢ}}\<%
\\
\>[0]\AgdaCatchallClause{\AgdaFunction{LRᵥ}}\AgdaSpace{}%
\AgdaCatchallClause{\AgdaSymbol{(}}\AgdaCatchallClause{\AgdaInductiveConstructor{★}}\AgdaSpace{}%
\AgdaCatchallClause{\AgdaOperator{\AgdaInductiveConstructor{,}}}\AgdaSpace{}%
\AgdaDottedPattern{\AgdaCatchallClause{\AgdaSymbol{.}}}\AgdaDottedPattern{\AgdaCatchallClause{\AgdaBound{A′}}}\AgdaSpace{}%
\AgdaCatchallClause{\AgdaOperator{\AgdaInductiveConstructor{,}}}\AgdaSpace{}%
\AgdaCatchallClause{\AgdaInductiveConstructor{unk⊑}}\AgdaCatchallClause{\AgdaSymbol{\{}}\AgdaCatchallClause{\AgdaBound{H}}\AgdaCatchallClause{\AgdaSymbol{\}\{}}\AgdaCatchallClause{\AgdaBound{A′}}\AgdaCatchallClause{\AgdaSymbol{\}}}\AgdaSpace{}%
\AgdaCatchallClause{\AgdaBound{d}}\AgdaCatchallClause{\AgdaSymbol{)}}\AgdaSpace{}%
\AgdaCatchallClause{\AgdaBound{dir}}\AgdaSpace{}%
\AgdaCatchallClause{\AgdaBound{V}}\AgdaSpace{}%
\AgdaCatchallClause{\AgdaBound{V′}}\AgdaSpace{}%
\AgdaSymbol{=}\AgdaSpace{}%
\AgdaDatatype{⊥}\AgdaSpace{}%
\AgdaOperator{\AgdaFunction{ˢ}}\<%
\end{code}

\caption{Logical Relation for Precision on Terms $\mathsf{LR}_t$
  and Values $\mathsf{LR}_v$}
\label{fig:log-rel-precision}
\end{figure}

The logical relation is defined in Figure~\ref{fig:log-rel-precision}
and explained in the following paragraphs.  The definition of the
logical relation for terms is based on the requirements of the gradual
guarantee, but it only talks about one step at a time of the leading
term. In the ≼ direction, the first clause says that the less-precise
$M$ takes a step to $N$ and that $N$ is related to $M′$ at one tick
later in time. The second clause allows the more-precise $M′$ to
reduce to an error. The third clause says that the less-precise $M$ is
already a value, and requires $M′$ to reduce to a value that is
related to $M$. The other direction ≽ is defined in a similar way,
but with the more precise term $M′$ taking one step at a time.

The following definitions combine the LRᵥ and LRₜ functions into a
single function, pre-LRₜ⊎LRᵥ, and than applies the μᵒ operator to
produce the recursive relation LRₜ⊎LRᵥ. Finally, we define some
shorthand for the logical relation on values, written ⊑ᴸᴿᵥ, and the
logical relation on terms, ⊑ᴸᴿₜ.

\begin{code}%
\>[0]\AgdaFunction{pre-LRₜ⊎LRᵥ}\AgdaSpace{}%
\AgdaSymbol{:}\AgdaSpace{}%
\AgdaFunction{LR-type}\AgdaSpace{}%
\AgdaSymbol{→}\AgdaSpace{}%
\AgdaRecord{Setˢ}\AgdaSpace{}%
\AgdaFunction{LR-ctx}\AgdaSpace{}%
\AgdaSymbol{(}\AgdaInductiveConstructor{cons}\AgdaSpace{}%
\AgdaInductiveConstructor{Later}\AgdaSpace{}%
\AgdaInductiveConstructor{∅}\AgdaSymbol{)}\<%
\\
\>[0]\AgdaFunction{pre-LRₜ⊎LRᵥ}\AgdaSpace{}%
\AgdaSymbol{(}\AgdaInductiveConstructor{inj₁}\AgdaSpace{}%
\AgdaSymbol{(}\AgdaBound{c}\AgdaSpace{}%
\AgdaOperator{\AgdaInductiveConstructor{,}}\AgdaSpace{}%
\AgdaBound{dir}\AgdaSpace{}%
\AgdaOperator{\AgdaInductiveConstructor{,}}\AgdaSpace{}%
\AgdaBound{V}\AgdaSpace{}%
\AgdaOperator{\AgdaInductiveConstructor{,}}\AgdaSpace{}%
\AgdaBound{V′}\AgdaSymbol{))}\AgdaSpace{}%
\AgdaSymbol{=}\AgdaSpace{}%
\AgdaFunction{LRᵥ}\AgdaSpace{}%
\AgdaBound{c}\AgdaSpace{}%
\AgdaBound{dir}\AgdaSpace{}%
\AgdaBound{V}\AgdaSpace{}%
\AgdaBound{V′}\<%
\\
\>[0]\AgdaFunction{pre-LRₜ⊎LRᵥ}\AgdaSpace{}%
\AgdaSymbol{(}\AgdaInductiveConstructor{inj₂}\AgdaSpace{}%
\AgdaSymbol{(}\AgdaBound{c}\AgdaSpace{}%
\AgdaOperator{\AgdaInductiveConstructor{,}}\AgdaSpace{}%
\AgdaBound{dir}\AgdaSpace{}%
\AgdaOperator{\AgdaInductiveConstructor{,}}\AgdaSpace{}%
\AgdaBound{M}\AgdaSpace{}%
\AgdaOperator{\AgdaInductiveConstructor{,}}\AgdaSpace{}%
\AgdaBound{M′}\AgdaSymbol{))}\AgdaSpace{}%
\AgdaSymbol{=}\AgdaSpace{}%
\AgdaFunction{LRₜ}\AgdaSpace{}%
\AgdaBound{c}\AgdaSpace{}%
\AgdaBound{dir}\AgdaSpace{}%
\AgdaBound{M}\AgdaSpace{}%
\AgdaBound{M′}\<%
\\
%
\\[\AgdaEmptyExtraSkip]%
\>[0]\AgdaFunction{LRₜ⊎LRᵥ}\AgdaSpace{}%
\AgdaSymbol{:}\AgdaSpace{}%
\AgdaFunction{LR-type}\AgdaSpace{}%
\AgdaSymbol{→}\AgdaSpace{}%
\AgdaRecord{Setᵒ}\<%
\\
\>[0]\AgdaFunction{LRₜ⊎LRᵥ}\AgdaSpace{}%
\AgdaBound{X}\AgdaSpace{}%
\AgdaSymbol{=}\AgdaSpace{}%
\AgdaFunction{μᵒ}\AgdaSpace{}%
\AgdaFunction{pre-LRₜ⊎LRᵥ}\AgdaSpace{}%
\AgdaBound{X}\<%
\\
%
\\[\AgdaEmptyExtraSkip]%
\>[0]\AgdaOperator{\AgdaFunction{\AgdaUnderscore{}∣\AgdaUnderscore{}⊑ᴸᴿᵥ\AgdaUnderscore{}⦂\AgdaUnderscore{}}}\AgdaSpace{}%
\AgdaSymbol{:}\AgdaSpace{}%
\AgdaDatatype{Dir}\AgdaSpace{}%
\AgdaSymbol{→}\AgdaSpace{}%
\AgdaDatatype{Term}\AgdaSpace{}%
\AgdaSymbol{→}\AgdaSpace{}%
\AgdaDatatype{Term}\AgdaSpace{}%
\AgdaSymbol{→}\AgdaSpace{}%
\AgdaSymbol{∀\{}\AgdaBound{A}\AgdaSpace{}%
\AgdaBound{A′}\AgdaSymbol{\}}\AgdaSpace{}%
\AgdaSymbol{→}\AgdaSpace{}%
\AgdaBound{A}\AgdaSpace{}%
\AgdaOperator{\AgdaDatatype{⊑}}\AgdaSpace{}%
\AgdaBound{A′}\AgdaSpace{}%
\AgdaSymbol{→}\AgdaSpace{}%
\AgdaRecord{Setᵒ}\<%
\\
\>[0]\AgdaBound{dir}\AgdaSpace{}%
\AgdaOperator{\AgdaFunction{∣}}\AgdaSpace{}%
\AgdaBound{V}\AgdaSpace{}%
\AgdaOperator{\AgdaFunction{⊑ᴸᴿᵥ}}\AgdaSpace{}%
\AgdaBound{V′}\AgdaSpace{}%
\AgdaOperator{\AgdaFunction{⦂}}\AgdaSpace{}%
\AgdaBound{A⊑A′}\AgdaSpace{}%
\AgdaSymbol{=}\AgdaSpace{}%
\AgdaFunction{LRₜ⊎LRᵥ}\AgdaSpace{}%
\AgdaSymbol{(}\AgdaInductiveConstructor{inj₁}\AgdaSpace{}%
\AgdaSymbol{((\AgdaUnderscore{}}\AgdaSpace{}%
\AgdaOperator{\AgdaInductiveConstructor{,}}\AgdaSpace{}%
\AgdaSymbol{\AgdaUnderscore{}}\AgdaSpace{}%
\AgdaOperator{\AgdaInductiveConstructor{,}}\AgdaSpace{}%
\AgdaBound{A⊑A′}\AgdaSymbol{)}\AgdaSpace{}%
\AgdaOperator{\AgdaInductiveConstructor{,}}\AgdaSpace{}%
\AgdaBound{dir}\AgdaSpace{}%
\AgdaOperator{\AgdaInductiveConstructor{,}}\AgdaSpace{}%
\AgdaBound{V}\AgdaSpace{}%
\AgdaOperator{\AgdaInductiveConstructor{,}}\AgdaSpace{}%
\AgdaBound{V′}\AgdaSymbol{))}\<%
\\
%
\\[\AgdaEmptyExtraSkip]%
\>[0]\AgdaOperator{\AgdaFunction{\AgdaUnderscore{}∣\AgdaUnderscore{}⊑ᴸᴿₜ\AgdaUnderscore{}⦂\AgdaUnderscore{}}}\AgdaSpace{}%
\AgdaSymbol{:}\AgdaSpace{}%
\AgdaDatatype{Dir}\AgdaSpace{}%
\AgdaSymbol{→}\AgdaSpace{}%
\AgdaDatatype{Term}\AgdaSpace{}%
\AgdaSymbol{→}\AgdaSpace{}%
\AgdaDatatype{Term}\AgdaSpace{}%
\AgdaSymbol{→}\AgdaSpace{}%
\AgdaSymbol{∀\{}\AgdaBound{A}\AgdaSpace{}%
\AgdaBound{A′}\AgdaSymbol{\}}\AgdaSpace{}%
\AgdaSymbol{→}\AgdaSpace{}%
\AgdaBound{A}\AgdaSpace{}%
\AgdaOperator{\AgdaDatatype{⊑}}\AgdaSpace{}%
\AgdaBound{A′}\AgdaSpace{}%
\AgdaSymbol{→}\AgdaSpace{}%
\AgdaRecord{Setᵒ}\<%
\\
\>[0]\AgdaBound{dir}\AgdaSpace{}%
\AgdaOperator{\AgdaFunction{∣}}\AgdaSpace{}%
\AgdaBound{M}\AgdaSpace{}%
\AgdaOperator{\AgdaFunction{⊑ᴸᴿₜ}}\AgdaSpace{}%
\AgdaBound{M′}\AgdaSpace{}%
\AgdaOperator{\AgdaFunction{⦂}}\AgdaSpace{}%
\AgdaBound{A⊑A′}\AgdaSpace{}%
\AgdaSymbol{=}\AgdaSpace{}%
\AgdaFunction{LRₜ⊎LRᵥ}\AgdaSpace{}%
\AgdaSymbol{(}\AgdaInductiveConstructor{inj₂}\AgdaSpace{}%
\AgdaSymbol{((\AgdaUnderscore{}}\AgdaSpace{}%
\AgdaOperator{\AgdaInductiveConstructor{,}}\AgdaSpace{}%
\AgdaSymbol{\AgdaUnderscore{}}\AgdaSpace{}%
\AgdaOperator{\AgdaInductiveConstructor{,}}\AgdaSpace{}%
\AgdaBound{A⊑A′}\AgdaSymbol{)}\AgdaSpace{}%
\AgdaOperator{\AgdaInductiveConstructor{,}}\AgdaSpace{}%
\AgdaBound{dir}\AgdaSpace{}%
\AgdaOperator{\AgdaInductiveConstructor{,}}\AgdaSpace{}%
\AgdaBound{M}\AgdaSpace{}%
\AgdaOperator{\AgdaInductiveConstructor{,}}\AgdaSpace{}%
\AgdaBound{M′}\AgdaSymbol{))}\<%
\\
%
\\[\AgdaEmptyExtraSkip]%
\>[0]\AgdaOperator{\AgdaFunction{\AgdaUnderscore{}⊑ᴸᴿₜ\AgdaUnderscore{}⦂\AgdaUnderscore{}}}\AgdaSpace{}%
\AgdaSymbol{:}\AgdaSpace{}%
\AgdaDatatype{Term}\AgdaSpace{}%
\AgdaSymbol{→}\AgdaSpace{}%
\AgdaDatatype{Term}\AgdaSpace{}%
\AgdaSymbol{→}\AgdaSpace{}%
\AgdaSymbol{∀\{}\AgdaBound{A}\AgdaSpace{}%
\AgdaBound{A′}\AgdaSymbol{\}}\AgdaSpace{}%
\AgdaSymbol{→}\AgdaSpace{}%
\AgdaBound{A}\AgdaSpace{}%
\AgdaOperator{\AgdaDatatype{⊑}}\AgdaSpace{}%
\AgdaBound{A′}\AgdaSpace{}%
\AgdaSymbol{→}\AgdaSpace{}%
\AgdaRecord{Setᵒ}\<%
\\
\>[0]\AgdaBound{M}\AgdaSpace{}%
\AgdaOperator{\AgdaFunction{⊑ᴸᴿₜ}}\AgdaSpace{}%
\AgdaBound{M′}\AgdaSpace{}%
\AgdaOperator{\AgdaFunction{⦂}}\AgdaSpace{}%
\AgdaBound{A⊑A′}\AgdaSpace{}%
\AgdaSymbol{=}\AgdaSpace{}%
\AgdaSymbol{(}\AgdaInductiveConstructor{≼}\AgdaSpace{}%
\AgdaOperator{\AgdaFunction{∣}}\AgdaSpace{}%
\AgdaBound{M}\AgdaSpace{}%
\AgdaOperator{\AgdaFunction{⊑ᴸᴿₜ}}\AgdaSpace{}%
\AgdaBound{M′}\AgdaSpace{}%
\AgdaOperator{\AgdaFunction{⦂}}\AgdaSpace{}%
\AgdaBound{A⊑A′}\AgdaSymbol{)}\AgdaSpace{}%
\AgdaOperator{\AgdaFunction{×ᵒ}}\AgdaSpace{}%
\AgdaSymbol{(}\AgdaInductiveConstructor{≽}\AgdaSpace{}%
\AgdaOperator{\AgdaFunction{∣}}\AgdaSpace{}%
\AgdaBound{M}\AgdaSpace{}%
\AgdaOperator{\AgdaFunction{⊑ᴸᴿₜ}}\AgdaSpace{}%
\AgdaBound{M′}\AgdaSpace{}%
\AgdaOperator{\AgdaFunction{⦂}}\AgdaSpace{}%
\AgdaBound{A⊑A′}\AgdaSymbol{)}\<%
\end{code}

The relations that we have defined so far, ⊑ᴸᴿᵥ and ⊑ᴸᴿₜ, only apply
to closed terms, that is, terms with no free variables.  We also need
to relate open terms. The standard way to do that is to apply two
substitutions to the two terms, replacing each free variable with
related values. We relate a pair of substitutions γ and γ′ with the
following definition, which says that the substitutions must be
pointwise related using the logical relation for values.

\begin{code}%
\>[0]\AgdaOperator{\AgdaFunction{\AgdaUnderscore{}∣\AgdaUnderscore{}⊨\AgdaUnderscore{}⊑ᴸᴿ\AgdaUnderscore{}}}\AgdaSpace{}%
\AgdaSymbol{:}\AgdaSpace{}%
\AgdaSymbol{(}\AgdaBound{Γ}\AgdaSpace{}%
\AgdaSymbol{:}\AgdaSpace{}%
\AgdaDatatype{List}\AgdaSpace{}%
\AgdaFunction{Prec}\AgdaSymbol{)}\AgdaSpace{}%
\AgdaSymbol{→}\AgdaSpace{}%
\AgdaDatatype{Dir}\AgdaSpace{}%
\AgdaSymbol{→}\AgdaSpace{}%
\AgdaFunction{Subst}\AgdaSpace{}%
\AgdaSymbol{→}\AgdaSpace{}%
\AgdaFunction{Subst}\AgdaSpace{}%
\AgdaSymbol{→}\AgdaSpace{}%
\AgdaDatatype{List}\AgdaSpace{}%
\AgdaRecord{Setᵒ}\<%
\\
\>[0]\AgdaInductiveConstructor{[]}\AgdaSpace{}%
\AgdaOperator{\AgdaFunction{∣}}\AgdaSpace{}%
\AgdaBound{dir}\AgdaSpace{}%
\AgdaOperator{\AgdaFunction{⊨}}\AgdaSpace{}%
\AgdaBound{γ}\AgdaSpace{}%
\AgdaOperator{\AgdaFunction{⊑ᴸᴿ}}\AgdaSpace{}%
\AgdaBound{γ′}\AgdaSpace{}%
\AgdaSymbol{=}\AgdaSpace{}%
\AgdaInductiveConstructor{[]}\<%
\\
\>[0]\AgdaSymbol{((\AgdaUnderscore{}}%
\>[1334I]\AgdaOperator{\AgdaInductiveConstructor{,}}\AgdaSpace{}%
\AgdaSymbol{\AgdaUnderscore{}}\AgdaSpace{}%
\AgdaOperator{\AgdaInductiveConstructor{,}}\AgdaSpace{}%
\AgdaBound{A⊑A′}\AgdaSymbol{)}\AgdaSpace{}%
\AgdaOperator{\AgdaInductiveConstructor{∷}}\AgdaSpace{}%
\AgdaBound{Γ}\AgdaSymbol{)}\AgdaSpace{}%
\AgdaOperator{\AgdaFunction{∣}}\AgdaSpace{}%
\AgdaBound{dir}\AgdaSpace{}%
\AgdaOperator{\AgdaFunction{⊨}}\AgdaSpace{}%
\AgdaBound{γ}\AgdaSpace{}%
\AgdaOperator{\AgdaFunction{⊑ᴸᴿ}}\AgdaSpace{}%
\AgdaBound{γ′}\AgdaSpace{}%
\AgdaSymbol{=}\<%
\\
\>[.][@{}l@{}]\<[1334I]%
\>[4]\AgdaSymbol{(}\AgdaBound{dir}\AgdaSpace{}%
\AgdaOperator{\AgdaFunction{∣}}\AgdaSpace{}%
\AgdaSymbol{(}\AgdaBound{γ}\AgdaSpace{}%
\AgdaNumber{0}\AgdaSymbol{)}\AgdaSpace{}%
\AgdaOperator{\AgdaFunction{⊑ᴸᴿᵥ}}\AgdaSpace{}%
\AgdaSymbol{(}\AgdaBound{γ′}\AgdaSpace{}%
\AgdaNumber{0}\AgdaSymbol{)}\AgdaSpace{}%
\AgdaOperator{\AgdaFunction{⦂}}\AgdaSpace{}%
\AgdaBound{A⊑A′}\AgdaSymbol{)}\AgdaSpace{}%
\AgdaOperator{\AgdaInductiveConstructor{∷}}\AgdaSpace{}%
\AgdaSymbol{(}\AgdaBound{Γ}\AgdaSpace{}%
\AgdaOperator{\AgdaFunction{∣}}\AgdaSpace{}%
\AgdaBound{dir}\AgdaSpace{}%
\AgdaOperator{\AgdaFunction{⊨}}\AgdaSpace{}%
\AgdaSymbol{(λ}\AgdaSpace{}%
\AgdaBound{x}\AgdaSpace{}%
\AgdaSymbol{→}\AgdaSpace{}%
\AgdaBound{γ}\AgdaSpace{}%
\AgdaSymbol{(}\AgdaInductiveConstructor{suc}\AgdaSpace{}%
\AgdaBound{x}\AgdaSymbol{))}\AgdaSpace{}%
\AgdaOperator{\AgdaFunction{⊑ᴸᴿ}}\AgdaSpace{}%
\AgdaSymbol{(λ}\AgdaSpace{}%
\AgdaBound{x}\AgdaSpace{}%
\AgdaSymbol{→}\AgdaSpace{}%
\AgdaBound{γ′}\AgdaSpace{}%
\AgdaSymbol{(}\AgdaInductiveConstructor{suc}\AgdaSpace{}%
\AgdaBound{x}\AgdaSymbol{)))}\<%
\end{code}

We then define two open terms $M$ and $M′$ to be logically related
if there are a pair of related subtitutions $γ$ and $γ′$ such that
applying them to $M$ and $M′$ produces related terms.

\begin{code}%
\>[0]\AgdaOperator{\AgdaFunction{\AgdaUnderscore{}∣\AgdaUnderscore{}⊨\AgdaUnderscore{}⊑ᴸᴿ\AgdaUnderscore{}⦂\AgdaUnderscore{}}}\AgdaSpace{}%
\AgdaSymbol{:}\AgdaSpace{}%
\AgdaDatatype{List}\AgdaSpace{}%
\AgdaFunction{Prec}\AgdaSpace{}%
\AgdaSymbol{→}\AgdaSpace{}%
\AgdaDatatype{Dir}\AgdaSpace{}%
\AgdaSymbol{→}\AgdaSpace{}%
\AgdaDatatype{Term}\AgdaSpace{}%
\AgdaSymbol{→}\AgdaSpace{}%
\AgdaDatatype{Term}\AgdaSpace{}%
\AgdaSymbol{→}\AgdaSpace{}%
\AgdaFunction{Prec}\AgdaSpace{}%
\AgdaSymbol{→}\AgdaSpace{}%
\AgdaPrimitive{Set}\<%
\\
\>[0]\AgdaBound{Γ}%
\>[1386I]\AgdaOperator{\AgdaFunction{∣}}\AgdaSpace{}%
\AgdaBound{dir}\AgdaSpace{}%
\AgdaOperator{\AgdaFunction{⊨}}\AgdaSpace{}%
\AgdaBound{M}\AgdaSpace{}%
\AgdaOperator{\AgdaFunction{⊑ᴸᴿ}}\AgdaSpace{}%
\AgdaBound{M′}\AgdaSpace{}%
\AgdaOperator{\AgdaFunction{⦂}}\AgdaSpace{}%
\AgdaSymbol{(\AgdaUnderscore{}}\AgdaSpace{}%
\AgdaOperator{\AgdaInductiveConstructor{,}}\AgdaSpace{}%
\AgdaSymbol{\AgdaUnderscore{}}\AgdaSpace{}%
\AgdaOperator{\AgdaInductiveConstructor{,}}\AgdaSpace{}%
\AgdaBound{A⊑A′}\AgdaSymbol{)}\AgdaSpace{}%
\AgdaSymbol{=}\AgdaSpace{}%
\AgdaSymbol{∀}\AgdaSpace{}%
\AgdaSymbol{(}\AgdaBound{γ}\AgdaSpace{}%
\AgdaBound{γ′}\AgdaSpace{}%
\AgdaSymbol{:}\AgdaSpace{}%
\AgdaFunction{Subst}\AgdaSymbol{)}\<%
\\
\>[1386I][@{}l@{\AgdaIndent{0}}]%
\>[3]\AgdaSymbol{→}\AgdaSpace{}%
\AgdaSymbol{(}\AgdaBound{Γ}\AgdaSpace{}%
\AgdaOperator{\AgdaFunction{∣}}\AgdaSpace{}%
\AgdaBound{dir}\AgdaSpace{}%
\AgdaOperator{\AgdaFunction{⊨}}\AgdaSpace{}%
\AgdaBound{γ}\AgdaSpace{}%
\AgdaOperator{\AgdaFunction{⊑ᴸᴿ}}\AgdaSpace{}%
\AgdaBound{γ′}\AgdaSymbol{)}\AgdaSpace{}%
\AgdaOperator{\AgdaFunction{⊢ᵒ}}\AgdaSpace{}%
\AgdaBound{dir}\AgdaSpace{}%
\AgdaOperator{\AgdaFunction{∣}}\AgdaSpace{}%
\AgdaSymbol{(}\AgdaOperator{\AgdaFunction{⟪}}\AgdaSpace{}%
\AgdaBound{γ}\AgdaSpace{}%
\AgdaOperator{\AgdaFunction{⟫}}\AgdaSpace{}%
\AgdaBound{M}\AgdaSymbol{)}\AgdaSpace{}%
\AgdaOperator{\AgdaFunction{⊑ᴸᴿₜ}}\AgdaSpace{}%
\AgdaSymbol{(}\AgdaOperator{\AgdaFunction{⟪}}\AgdaSpace{}%
\AgdaBound{γ′}\AgdaSpace{}%
\AgdaOperator{\AgdaFunction{⟫}}\AgdaSpace{}%
\AgdaBound{M′}\AgdaSymbol{)}\AgdaSpace{}%
\AgdaOperator{\AgdaFunction{⦂}}\AgdaSpace{}%
\AgdaBound{A⊑A′}\<%
\end{code}

We use the following notation for the conjunction of the two
directions and define the \textsf{proj} function for accessing each
direction.

\begin{code}%
\>[0]\AgdaOperator{\AgdaFunction{\AgdaUnderscore{}⊨\AgdaUnderscore{}⊑ᴸᴿ\AgdaUnderscore{}⦂\AgdaUnderscore{}}}\AgdaSpace{}%
\AgdaSymbol{:}\AgdaSpace{}%
\AgdaDatatype{List}\AgdaSpace{}%
\AgdaFunction{Prec}\AgdaSpace{}%
\AgdaSymbol{→}\AgdaSpace{}%
\AgdaDatatype{Term}\AgdaSpace{}%
\AgdaSymbol{→}\AgdaSpace{}%
\AgdaDatatype{Term}\AgdaSpace{}%
\AgdaSymbol{→}\AgdaSpace{}%
\AgdaFunction{Prec}\AgdaSpace{}%
\AgdaSymbol{→}\AgdaSpace{}%
\AgdaPrimitive{Set}\<%
\\
\>[0]\AgdaBound{Γ}\AgdaSpace{}%
\AgdaOperator{\AgdaFunction{⊨}}\AgdaSpace{}%
\AgdaBound{M}\AgdaSpace{}%
\AgdaOperator{\AgdaFunction{⊑ᴸᴿ}}\AgdaSpace{}%
\AgdaBound{M′}\AgdaSpace{}%
\AgdaOperator{\AgdaFunction{⦂}}\AgdaSpace{}%
\AgdaBound{c}\AgdaSpace{}%
\AgdaSymbol{=}\AgdaSpace{}%
\AgdaSymbol{(}\AgdaBound{Γ}\AgdaSpace{}%
\AgdaOperator{\AgdaFunction{∣}}\AgdaSpace{}%
\AgdaInductiveConstructor{≼}\AgdaSpace{}%
\AgdaOperator{\AgdaFunction{⊨}}\AgdaSpace{}%
\AgdaBound{M}\AgdaSpace{}%
\AgdaOperator{\AgdaFunction{⊑ᴸᴿ}}\AgdaSpace{}%
\AgdaBound{M′}\AgdaSpace{}%
\AgdaOperator{\AgdaFunction{⦂}}\AgdaSpace{}%
\AgdaBound{c}\AgdaSymbol{)}\AgdaSpace{}%
\AgdaOperator{\AgdaFunction{×}}\AgdaSpace{}%
\AgdaSymbol{(}\AgdaBound{Γ}\AgdaSpace{}%
\AgdaOperator{\AgdaFunction{∣}}\AgdaSpace{}%
\AgdaInductiveConstructor{≽}\AgdaSpace{}%
\AgdaOperator{\AgdaFunction{⊨}}\AgdaSpace{}%
\AgdaBound{M}\AgdaSpace{}%
\AgdaOperator{\AgdaFunction{⊑ᴸᴿ}}\AgdaSpace{}%
\AgdaBound{M′}\AgdaSpace{}%
\AgdaOperator{\AgdaFunction{⦂}}\AgdaSpace{}%
\AgdaBound{c}\AgdaSymbol{)}\<%
\\
%
\\[\AgdaEmptyExtraSkip]%
\>[0]\AgdaFunction{proj}\AgdaSpace{}%
\AgdaSymbol{:}\AgdaSpace{}%
\AgdaSymbol{∀}\AgdaSpace{}%
\AgdaSymbol{\{}\AgdaBound{Γ}\AgdaSymbol{\}\{}\AgdaBound{c}\AgdaSymbol{\}}\AgdaSpace{}%
\AgdaSymbol{→}\AgdaSpace{}%
\AgdaSymbol{(}\AgdaBound{dir}\AgdaSpace{}%
\AgdaSymbol{:}\AgdaSpace{}%
\AgdaDatatype{Dir}\AgdaSymbol{)}\AgdaSpace{}%
\AgdaSymbol{→}\AgdaSpace{}%
\AgdaSymbol{(}\AgdaBound{M}\AgdaSpace{}%
\AgdaBound{M′}\AgdaSpace{}%
\AgdaSymbol{:}\AgdaSpace{}%
\AgdaDatatype{Term}\AgdaSymbol{)}\AgdaSpace{}%
\AgdaSymbol{→}\AgdaSpace{}%
\AgdaBound{Γ}\AgdaSpace{}%
\AgdaOperator{\AgdaFunction{⊨}}\AgdaSpace{}%
\AgdaBound{M}\AgdaSpace{}%
\AgdaOperator{\AgdaFunction{⊑ᴸᴿ}}\AgdaSpace{}%
\AgdaBound{M′}\AgdaSpace{}%
\AgdaOperator{\AgdaFunction{⦂}}\AgdaSpace{}%
\AgdaBound{c}\AgdaSpace{}%
\AgdaSymbol{→}\AgdaSpace{}%
\AgdaBound{Γ}\AgdaSpace{}%
\AgdaOperator{\AgdaFunction{∣}}\AgdaSpace{}%
\AgdaBound{dir}\AgdaSpace{}%
\AgdaOperator{\AgdaFunction{⊨}}\AgdaSpace{}%
\AgdaBound{M}\AgdaSpace{}%
\AgdaOperator{\AgdaFunction{⊑ᴸᴿ}}\AgdaSpace{}%
\AgdaBound{M′}\AgdaSpace{}%
\AgdaOperator{\AgdaFunction{⦂}}\AgdaSpace{}%
\AgdaBound{c}\<%
\\
\>[0]\AgdaFunction{proj}\AgdaSpace{}%
\AgdaInductiveConstructor{≼}\AgdaSpace{}%
\AgdaBound{M}\AgdaSpace{}%
\AgdaBound{M′}\AgdaSpace{}%
\AgdaBound{M⊑M′}\AgdaSpace{}%
\AgdaSymbol{=}\AgdaSpace{}%
\AgdaField{proj₁}\AgdaSpace{}%
\AgdaBound{M⊑M′}\<%
\\
\>[0]\AgdaFunction{proj}\AgdaSpace{}%
\AgdaInductiveConstructor{≽}\AgdaSpace{}%
\AgdaBound{M}\AgdaSpace{}%
\AgdaBound{M′}\AgdaSpace{}%
\AgdaBound{M⊑M′}\AgdaSpace{}%
\AgdaSymbol{=}\AgdaSpace{}%
\AgdaField{proj₂}\AgdaSpace{}%
\AgdaBound{M⊑M′}\<%
\end{code}

\begin{code}[hide]%
\>[0]\AgdaFunction{LRₜ-def}\AgdaSpace{}%
\AgdaSymbol{:}\AgdaSpace{}%
\AgdaSymbol{∀\{}\AgdaBound{A}\AgdaSymbol{\}\{}\AgdaBound{A′}\AgdaSymbol{\}}\AgdaSpace{}%
\AgdaSymbol{→}\AgdaSpace{}%
\AgdaSymbol{(}\AgdaBound{A⊑A′}\AgdaSpace{}%
\AgdaSymbol{:}\AgdaSpace{}%
\AgdaBound{A}\AgdaSpace{}%
\AgdaOperator{\AgdaDatatype{⊑}}\AgdaSpace{}%
\AgdaBound{A′}\AgdaSymbol{)}\AgdaSpace{}%
\AgdaSymbol{→}\AgdaSpace{}%
\AgdaDatatype{Dir}\AgdaSpace{}%
\AgdaSymbol{→}\AgdaSpace{}%
\AgdaDatatype{Term}\AgdaSpace{}%
\AgdaSymbol{→}\AgdaSpace{}%
\AgdaDatatype{Term}\AgdaSpace{}%
\AgdaSymbol{→}\AgdaSpace{}%
\AgdaRecord{Setᵒ}\<%
\\
\>[0]\AgdaFunction{LRₜ-def}\AgdaSpace{}%
\AgdaBound{A⊑A′}\AgdaSpace{}%
\AgdaInductiveConstructor{≼}\AgdaSpace{}%
\AgdaBound{M}\AgdaSpace{}%
\AgdaBound{M′}\AgdaSpace{}%
\AgdaSymbol{=}\<%
\\
\>[0][@{}l@{\AgdaIndent{0}}]%
\>[3]\AgdaSymbol{(}\AgdaFunction{∃ᵒ[}\AgdaSpace{}%
\AgdaBound{N}\AgdaSpace{}%
\AgdaFunction{]}\AgdaSpace{}%
\AgdaSymbol{(}\AgdaBound{M}\AgdaSpace{}%
\AgdaOperator{\AgdaDatatype{⟶}}\AgdaSpace{}%
\AgdaBound{N}\AgdaSymbol{)}\AgdaOperator{\AgdaFunction{ᵒ}}\AgdaSpace{}%
\AgdaOperator{\AgdaFunction{×ᵒ}}\AgdaSpace{}%
\AgdaOperator{\AgdaFunction{▷ᵒ}}\AgdaSpace{}%
\AgdaSymbol{(}\AgdaInductiveConstructor{≼}\AgdaSpace{}%
\AgdaOperator{\AgdaFunction{∣}}\AgdaSpace{}%
\AgdaBound{N}\AgdaSpace{}%
\AgdaOperator{\AgdaFunction{⊑ᴸᴿₜ}}\AgdaSpace{}%
\AgdaBound{M′}\AgdaSpace{}%
\AgdaOperator{\AgdaFunction{⦂}}\AgdaSpace{}%
\AgdaBound{A⊑A′}\AgdaSymbol{))}\<%
\\
%
\>[3]\AgdaOperator{\AgdaFunction{⊎ᵒ}}\AgdaSpace{}%
\AgdaSymbol{(}\AgdaBound{M′}\AgdaSpace{}%
\AgdaOperator{\AgdaDatatype{↠}}\AgdaSpace{}%
\AgdaInductiveConstructor{blame}\AgdaSymbol{)}\AgdaOperator{\AgdaFunction{ᵒ}}\<%
\\
%
\>[3]\AgdaOperator{\AgdaFunction{⊎ᵒ}}\AgdaSpace{}%
\AgdaSymbol{((}\AgdaDatatype{Value}\AgdaSpace{}%
\AgdaBound{M}\AgdaSymbol{)}\AgdaOperator{\AgdaFunction{ᵒ}}\AgdaSpace{}%
\AgdaOperator{\AgdaFunction{×ᵒ}}\AgdaSpace{}%
\AgdaSymbol{(}\AgdaFunction{∃ᵒ[}\AgdaSpace{}%
\AgdaBound{V′}\AgdaSpace{}%
\AgdaFunction{]}\AgdaSpace{}%
\AgdaSymbol{(}\AgdaBound{M′}\AgdaSpace{}%
\AgdaOperator{\AgdaDatatype{↠}}\AgdaSpace{}%
\AgdaBound{V′}\AgdaSymbol{)}\AgdaOperator{\AgdaFunction{ᵒ}}\AgdaSpace{}%
\AgdaOperator{\AgdaFunction{×ᵒ}}\AgdaSpace{}%
\AgdaSymbol{(}\AgdaDatatype{Value}\AgdaSpace{}%
\AgdaBound{V′}\AgdaSymbol{)}\AgdaOperator{\AgdaFunction{ᵒ}}\AgdaSpace{}%
\AgdaOperator{\AgdaFunction{×ᵒ}}\AgdaSpace{}%
\AgdaSymbol{(}\AgdaInductiveConstructor{≼}\AgdaSpace{}%
\AgdaOperator{\AgdaFunction{∣}}\AgdaSpace{}%
\AgdaBound{M}\AgdaSpace{}%
\AgdaOperator{\AgdaFunction{⊑ᴸᴿᵥ}}\AgdaSpace{}%
\AgdaBound{V′}\AgdaSpace{}%
\AgdaOperator{\AgdaFunction{⦂}}\AgdaSpace{}%
\AgdaBound{A⊑A′}\AgdaSymbol{)))}\<%
\\
\>[0]\AgdaFunction{LRₜ-def}\AgdaSpace{}%
\AgdaBound{A⊑A′}\AgdaSpace{}%
\AgdaInductiveConstructor{≽}\AgdaSpace{}%
\AgdaBound{M}\AgdaSpace{}%
\AgdaBound{M′}\AgdaSpace{}%
\AgdaSymbol{=}\<%
\\
\>[0][@{}l@{\AgdaIndent{0}}]%
\>[3]\AgdaSymbol{(}\AgdaFunction{∃ᵒ[}\AgdaSpace{}%
\AgdaBound{N′}\AgdaSpace{}%
\AgdaFunction{]}\AgdaSpace{}%
\AgdaSymbol{(}\AgdaBound{M′}\AgdaSpace{}%
\AgdaOperator{\AgdaDatatype{⟶}}\AgdaSpace{}%
\AgdaBound{N′}\AgdaSymbol{)}\AgdaOperator{\AgdaFunction{ᵒ}}\AgdaSpace{}%
\AgdaOperator{\AgdaFunction{×ᵒ}}\AgdaSpace{}%
\AgdaOperator{\AgdaFunction{▷ᵒ}}\AgdaSpace{}%
\AgdaSymbol{(}\AgdaInductiveConstructor{≽}\AgdaSpace{}%
\AgdaOperator{\AgdaFunction{∣}}\AgdaSpace{}%
\AgdaBound{M}\AgdaSpace{}%
\AgdaOperator{\AgdaFunction{⊑ᴸᴿₜ}}\AgdaSpace{}%
\AgdaBound{N′}\AgdaSpace{}%
\AgdaOperator{\AgdaFunction{⦂}}\AgdaSpace{}%
\AgdaBound{A⊑A′}\AgdaSymbol{))}\<%
\\
%
\>[3]\AgdaOperator{\AgdaFunction{⊎ᵒ}}\AgdaSpace{}%
\AgdaSymbol{(}\AgdaDatatype{Blame}\AgdaSpace{}%
\AgdaBound{M′}\AgdaSymbol{)}\AgdaOperator{\AgdaFunction{ᵒ}}\<%
\\
%
\>[3]\AgdaOperator{\AgdaFunction{⊎ᵒ}}\AgdaSpace{}%
\AgdaSymbol{((}\AgdaDatatype{Value}\AgdaSpace{}%
\AgdaBound{M′}\AgdaSymbol{)}\AgdaOperator{\AgdaFunction{ᵒ}}\AgdaSpace{}%
\AgdaOperator{\AgdaFunction{×ᵒ}}\AgdaSpace{}%
\AgdaSymbol{(}\AgdaFunction{∃ᵒ[}\AgdaSpace{}%
\AgdaBound{V}\AgdaSpace{}%
\AgdaFunction{]}\AgdaSpace{}%
\AgdaSymbol{(}\AgdaBound{M}\AgdaSpace{}%
\AgdaOperator{\AgdaDatatype{↠}}\AgdaSpace{}%
\AgdaBound{V}\AgdaSymbol{)}\AgdaOperator{\AgdaFunction{ᵒ}}\AgdaSpace{}%
\AgdaOperator{\AgdaFunction{×ᵒ}}\AgdaSpace{}%
\AgdaSymbol{(}\AgdaDatatype{Value}\AgdaSpace{}%
\AgdaBound{V}\AgdaSymbol{)}\AgdaOperator{\AgdaFunction{ᵒ}}\AgdaSpace{}%
\AgdaOperator{\AgdaFunction{×ᵒ}}\AgdaSpace{}%
\AgdaSymbol{(}\AgdaInductiveConstructor{≽}\AgdaSpace{}%
\AgdaOperator{\AgdaFunction{∣}}\AgdaSpace{}%
\AgdaBound{V}\AgdaSpace{}%
\AgdaOperator{\AgdaFunction{⊑ᴸᴿᵥ}}\AgdaSpace{}%
\AgdaBound{M′}\AgdaSpace{}%
\AgdaOperator{\AgdaFunction{⦂}}\AgdaSpace{}%
\AgdaBound{A⊑A′}\AgdaSymbol{)))}\<%
\end{code}
\begin{code}[hide]%
\>[0]\AgdaFunction{LRₜ-stmt}\AgdaSpace{}%
\AgdaSymbol{:}\AgdaSpace{}%
\AgdaSymbol{∀\{}\AgdaBound{A}\AgdaSymbol{\}\{}\AgdaBound{A′}\AgdaSymbol{\}\{}\AgdaBound{A⊑A′}\AgdaSpace{}%
\AgdaSymbol{:}\AgdaSpace{}%
\AgdaBound{A}\AgdaSpace{}%
\AgdaOperator{\AgdaDatatype{⊑}}\AgdaSpace{}%
\AgdaBound{A′}\AgdaSymbol{\}\{}\AgdaBound{dir}\AgdaSymbol{\}\{}\AgdaBound{M}\AgdaSymbol{\}\{}\AgdaBound{M′}\AgdaSymbol{\}}\<%
\\
\>[0][@{}l@{\AgdaIndent{0}}]%
\>[3]\AgdaSymbol{→}\AgdaSpace{}%
\AgdaBound{dir}\AgdaSpace{}%
\AgdaOperator{\AgdaFunction{∣}}\AgdaSpace{}%
\AgdaBound{M}\AgdaSpace{}%
\AgdaOperator{\AgdaFunction{⊑ᴸᴿₜ}}\AgdaSpace{}%
\AgdaBound{M′}\AgdaSpace{}%
\AgdaOperator{\AgdaFunction{⦂}}\AgdaSpace{}%
\AgdaBound{A⊑A′}\AgdaSpace{}%
\AgdaOperator{\AgdaFunction{≡ᵒ}}\AgdaSpace{}%
\AgdaFunction{LRₜ-def}\AgdaSpace{}%
\AgdaBound{A⊑A′}\AgdaSpace{}%
\AgdaBound{dir}\AgdaSpace{}%
\AgdaBound{M}\AgdaSpace{}%
\AgdaBound{M′}\<%
\end{code}
\begin{code}[hide]%
\>[0]\AgdaFunction{LRₜ-stmt}\AgdaSpace{}%
\AgdaSymbol{\{}\AgdaBound{A}\AgdaSymbol{\}\{}\AgdaBound{A′}\AgdaSymbol{\}\{}\AgdaBound{A⊑A′}\AgdaSymbol{\}\{}\AgdaBound{dir}\AgdaSymbol{\}\{}\AgdaBound{M}\AgdaSymbol{\}\{}\AgdaBound{M′}\AgdaSymbol{\}}\AgdaSpace{}%
\AgdaSymbol{=}\<%
\\
\>[0][@{}l@{\AgdaIndent{0}}]%
\>[2]\AgdaBound{dir}\AgdaSpace{}%
\AgdaOperator{\AgdaFunction{∣}}\AgdaSpace{}%
\AgdaBound{M}\AgdaSpace{}%
\AgdaOperator{\AgdaFunction{⊑ᴸᴿₜ}}\AgdaSpace{}%
\AgdaBound{M′}\AgdaSpace{}%
\AgdaOperator{\AgdaFunction{⦂}}\AgdaSpace{}%
\AgdaBound{A⊑A′}%
\>[43]\AgdaOperator{\AgdaFunction{⩦⟨}}\AgdaSpace{}%
\AgdaFunction{≡ᵒ-refl}\AgdaSpace{}%
\AgdaInductiveConstructor{refl}\AgdaSpace{}%
\AgdaOperator{\AgdaFunction{⟩}}\<%
\\
%
\>[2]\AgdaFunction{μᵒ}\AgdaSpace{}%
\AgdaFunction{pre-LRₜ⊎LRᵥ}\AgdaSpace{}%
\AgdaSymbol{(}\AgdaFunction{X₂}\AgdaSpace{}%
\AgdaBound{dir}\AgdaSymbol{)}%
\>[43]\AgdaOperator{\AgdaFunction{⩦⟨}}\AgdaSpace{}%
\AgdaFunction{fixpointᵒ}\AgdaSpace{}%
\AgdaFunction{pre-LRₜ⊎LRᵥ}\AgdaSpace{}%
\AgdaSymbol{(}\AgdaFunction{X₂}\AgdaSpace{}%
\AgdaBound{dir}\AgdaSymbol{)}\AgdaSpace{}%
\AgdaOperator{\AgdaFunction{⟩}}\<%
\\
%
\>[2]\AgdaField{\#}\AgdaSpace{}%
\AgdaSymbol{(}\AgdaFunction{pre-LRₜ⊎LRᵥ}\AgdaSpace{}%
\AgdaSymbol{(}\AgdaFunction{X₂}\AgdaSpace{}%
\AgdaBound{dir}\AgdaSymbol{))}\AgdaSpace{}%
\AgdaSymbol{(}\AgdaFunction{LRₜ⊎LRᵥ}\AgdaSpace{}%
\AgdaOperator{\AgdaInductiveConstructor{,}}\AgdaSpace{}%
\AgdaFunction{ttᵖ}\AgdaSymbol{)}\AgdaSpace{}%
\AgdaOperator{\AgdaFunction{⩦⟨}}\AgdaSpace{}%
\AgdaFunction{EQ}\AgdaSymbol{\{}\AgdaBound{dir}\AgdaSymbol{\}}\AgdaSpace{}%
\AgdaOperator{\AgdaFunction{⟩}}\<%
\\
%
\>[2]\AgdaFunction{LRₜ-def}\AgdaSpace{}%
\AgdaBound{A⊑A′}\AgdaSpace{}%
\AgdaBound{dir}\AgdaSpace{}%
\AgdaBound{M}\AgdaSpace{}%
\AgdaBound{M′}%
\>[43]\AgdaOperator{\AgdaFunction{∎}}\<%
\\
%
\>[2]\AgdaKeyword{where}\<%
\\
%
\>[2]\AgdaFunction{c}\AgdaSpace{}%
\AgdaSymbol{=}\AgdaSpace{}%
\AgdaSymbol{(}\AgdaBound{A}\AgdaSpace{}%
\AgdaOperator{\AgdaInductiveConstructor{,}}\AgdaSpace{}%
\AgdaBound{A′}\AgdaSpace{}%
\AgdaOperator{\AgdaInductiveConstructor{,}}\AgdaSpace{}%
\AgdaBound{A⊑A′}\AgdaSymbol{)}\<%
\\
%
\>[2]\AgdaFunction{X₁}\AgdaSpace{}%
\AgdaSymbol{:}\AgdaSpace{}%
\AgdaDatatype{Dir}\AgdaSpace{}%
\AgdaSymbol{→}\AgdaSpace{}%
\AgdaFunction{LR-type}\<%
\\
%
\>[2]\AgdaFunction{X₁}\AgdaSpace{}%
\AgdaSymbol{=}\AgdaSpace{}%
\AgdaSymbol{λ}\AgdaSpace{}%
\AgdaBound{dir}\AgdaSpace{}%
\AgdaSymbol{→}\AgdaSpace{}%
\AgdaInductiveConstructor{inj₁}\AgdaSpace{}%
\AgdaSymbol{(}\AgdaFunction{c}\AgdaSpace{}%
\AgdaOperator{\AgdaInductiveConstructor{,}}\AgdaSpace{}%
\AgdaBound{dir}\AgdaSpace{}%
\AgdaOperator{\AgdaInductiveConstructor{,}}\AgdaSpace{}%
\AgdaBound{M}\AgdaSpace{}%
\AgdaOperator{\AgdaInductiveConstructor{,}}\AgdaSpace{}%
\AgdaBound{M′}\AgdaSymbol{)}\<%
\\
%
\>[2]\AgdaFunction{X₂}\AgdaSpace{}%
\AgdaSymbol{=}\AgdaSpace{}%
\AgdaSymbol{λ}\AgdaSpace{}%
\AgdaBound{dir}\AgdaSpace{}%
\AgdaSymbol{→}\AgdaSpace{}%
\AgdaInductiveConstructor{inj₂}\AgdaSpace{}%
\AgdaSymbol{(}\AgdaFunction{c}\AgdaSpace{}%
\AgdaOperator{\AgdaInductiveConstructor{,}}\AgdaSpace{}%
\AgdaBound{dir}\AgdaSpace{}%
\AgdaOperator{\AgdaInductiveConstructor{,}}\AgdaSpace{}%
\AgdaBound{M}\AgdaSpace{}%
\AgdaOperator{\AgdaInductiveConstructor{,}}\AgdaSpace{}%
\AgdaBound{M′}\AgdaSymbol{)}\<%
\\
%
\>[2]\AgdaFunction{EQ}\AgdaSpace{}%
\AgdaSymbol{:}\AgdaSpace{}%
\AgdaSymbol{∀\{}\AgdaBound{dir}\AgdaSymbol{\}}\AgdaSpace{}%
\AgdaSymbol{→}\AgdaSpace{}%
\AgdaField{\#}\AgdaSpace{}%
\AgdaSymbol{(}\AgdaFunction{pre-LRₜ⊎LRᵥ}\AgdaSpace{}%
\AgdaSymbol{(}\AgdaFunction{X₂}\AgdaSpace{}%
\AgdaBound{dir}\AgdaSymbol{))}\AgdaSpace{}%
\AgdaSymbol{(}\AgdaFunction{LRₜ⊎LRᵥ}\AgdaSpace{}%
\AgdaOperator{\AgdaInductiveConstructor{,}}\AgdaSpace{}%
\AgdaFunction{ttᵖ}\AgdaSymbol{)}\AgdaSpace{}%
\AgdaOperator{\AgdaFunction{≡ᵒ}}\AgdaSpace{}%
\AgdaFunction{LRₜ-def}\AgdaSpace{}%
\AgdaBound{A⊑A′}\AgdaSpace{}%
\AgdaBound{dir}\AgdaSpace{}%
\AgdaBound{M}\AgdaSpace{}%
\AgdaBound{M′}\<%
\\
%
\>[2]\AgdaFunction{EQ}\AgdaSpace{}%
\AgdaSymbol{\{}\AgdaInductiveConstructor{≼}\AgdaSymbol{\}}\AgdaSpace{}%
\AgdaSymbol{=}%
\>[1708I]\AgdaFunction{cong-⊎ᵒ}\AgdaSpace{}%
\AgdaSymbol{(}\AgdaFunction{≡ᵒ-refl}\AgdaSpace{}%
\AgdaInductiveConstructor{refl}\AgdaSymbol{)}\AgdaSpace{}%
\AgdaSymbol{(}\AgdaFunction{cong-⊎ᵒ}\AgdaSpace{}%
\AgdaSymbol{(}\AgdaFunction{≡ᵒ-refl}\AgdaSpace{}%
\AgdaInductiveConstructor{refl}\AgdaSymbol{)}\AgdaSpace{}%
\AgdaSymbol{(}\AgdaFunction{cong-×ᵒ}\AgdaSpace{}%
\AgdaSymbol{(}\AgdaFunction{≡ᵒ-refl}\AgdaSpace{}%
\AgdaInductiveConstructor{refl}\AgdaSymbol{)}\<%
\\
\>[1708I][@{}l@{\AgdaIndent{0}}]%
\>[13]\AgdaSymbol{(}\AgdaFunction{cong-∃}\AgdaSpace{}%
\AgdaSymbol{λ}\AgdaSpace{}%
\AgdaBound{V′}\AgdaSpace{}%
\AgdaSymbol{→}\AgdaSpace{}%
\AgdaFunction{cong-×ᵒ}\AgdaSpace{}%
\AgdaSymbol{(}\AgdaFunction{≡ᵒ-refl}\AgdaSpace{}%
\AgdaInductiveConstructor{refl}\AgdaSymbol{)}\AgdaSpace{}%
\AgdaSymbol{(}\AgdaFunction{cong-×ᵒ}\AgdaSpace{}%
\AgdaSymbol{(}\AgdaFunction{≡ᵒ-refl}\AgdaSpace{}%
\AgdaInductiveConstructor{refl}\AgdaSymbol{)}\<%
\\
\>[13][@{}l@{\AgdaIndent{0}}]%
\>[14]\AgdaSymbol{((}\AgdaFunction{≡ᵒ-sym}\AgdaSpace{}%
\AgdaSymbol{(}\AgdaFunction{fixpointᵒ}\AgdaSpace{}%
\AgdaFunction{pre-LRₜ⊎LRᵥ}\AgdaSpace{}%
\AgdaSymbol{(}\AgdaInductiveConstructor{inj₁}\AgdaSpace{}%
\AgdaSymbol{(}\AgdaFunction{c}\AgdaSpace{}%
\AgdaOperator{\AgdaInductiveConstructor{,}}\AgdaSpace{}%
\AgdaInductiveConstructor{≼}\AgdaSpace{}%
\AgdaOperator{\AgdaInductiveConstructor{,}}\AgdaSpace{}%
\AgdaBound{M}\AgdaSpace{}%
\AgdaOperator{\AgdaInductiveConstructor{,}}\AgdaSpace{}%
\AgdaBound{V′}\AgdaSymbol{)))))))))}\<%
\\
%
\>[2]\AgdaFunction{EQ}\AgdaSpace{}%
\AgdaSymbol{\{}\AgdaInductiveConstructor{≽}\AgdaSymbol{\}}\AgdaSpace{}%
\AgdaSymbol{=}%
\>[1738I]\AgdaFunction{cong-⊎ᵒ}\AgdaSpace{}%
\AgdaSymbol{(}\AgdaFunction{≡ᵒ-refl}\AgdaSpace{}%
\AgdaInductiveConstructor{refl}\AgdaSymbol{)}\AgdaSpace{}%
\AgdaSymbol{(}\AgdaFunction{cong-⊎ᵒ}\AgdaSpace{}%
\AgdaSymbol{(}\AgdaFunction{≡ᵒ-refl}\AgdaSpace{}%
\AgdaInductiveConstructor{refl}\AgdaSymbol{)}\<%
\\
\>[1738I][@{}l@{\AgdaIndent{0}}]%
\>[12]\AgdaSymbol{(}\AgdaFunction{cong-×ᵒ}\AgdaSpace{}%
\AgdaSymbol{(}\AgdaFunction{≡ᵒ-refl}\AgdaSpace{}%
\AgdaInductiveConstructor{refl}\AgdaSymbol{)}\AgdaSpace{}%
\AgdaSymbol{(}\AgdaFunction{cong-∃}\AgdaSpace{}%
\AgdaSymbol{λ}\AgdaSpace{}%
\AgdaBound{V}\AgdaSpace{}%
\AgdaSymbol{→}\AgdaSpace{}%
\AgdaFunction{cong-×ᵒ}\AgdaSpace{}%
\AgdaSymbol{(}\AgdaFunction{≡ᵒ-refl}\AgdaSpace{}%
\AgdaInductiveConstructor{refl}\AgdaSymbol{)}\<%
\\
\>[12][@{}l@{\AgdaIndent{0}}]%
\>[14]\AgdaSymbol{(}\AgdaFunction{cong-×ᵒ}\AgdaSpace{}%
\AgdaSymbol{(}\AgdaFunction{≡ᵒ-refl}\AgdaSpace{}%
\AgdaInductiveConstructor{refl}\AgdaSymbol{)}\<%
\\
\>[14][@{}l@{\AgdaIndent{0}}]%
\>[15]\AgdaSymbol{(}\AgdaFunction{≡ᵒ-sym}\AgdaSpace{}%
\AgdaSymbol{(}\AgdaFunction{fixpointᵒ}\AgdaSpace{}%
\AgdaFunction{pre-LRₜ⊎LRᵥ}\AgdaSpace{}%
\AgdaSymbol{(}\AgdaInductiveConstructor{inj₁}\AgdaSpace{}%
\AgdaSymbol{(}\AgdaFunction{c}\AgdaSpace{}%
\AgdaOperator{\AgdaInductiveConstructor{,}}\AgdaSpace{}%
\AgdaInductiveConstructor{≽}\AgdaSpace{}%
\AgdaOperator{\AgdaInductiveConstructor{,}}\AgdaSpace{}%
\AgdaBound{V}\AgdaSpace{}%
\AgdaOperator{\AgdaInductiveConstructor{,}}\AgdaSpace{}%
\AgdaBound{M′}\AgdaSymbol{))))))))}\<%
\end{code}
\begin{code}[hide]%
\>[0]\AgdaFunction{LRₜ-suc}\AgdaSpace{}%
\AgdaSymbol{:}\AgdaSpace{}%
\AgdaSymbol{∀\{}\AgdaBound{A}\AgdaSymbol{\}\{}\AgdaBound{A′}\AgdaSymbol{\}\{}\AgdaBound{A⊑A′}\AgdaSpace{}%
\AgdaSymbol{:}\AgdaSpace{}%
\AgdaBound{A}\AgdaSpace{}%
\AgdaOperator{\AgdaDatatype{⊑}}\AgdaSpace{}%
\AgdaBound{A′}\AgdaSymbol{\}\{}\AgdaBound{dir}\AgdaSymbol{\}\{}\AgdaBound{M}\AgdaSymbol{\}\{}\AgdaBound{M′}\AgdaSymbol{\}\{}\AgdaBound{k}\AgdaSymbol{\}}\<%
\\
\>[0][@{}l@{\AgdaIndent{0}}]%
\>[2]\AgdaSymbol{→}\AgdaSpace{}%
\AgdaField{\#}\AgdaSymbol{(}\AgdaBound{dir}\AgdaSpace{}%
\AgdaOperator{\AgdaFunction{∣}}\AgdaSpace{}%
\AgdaBound{M}\AgdaSpace{}%
\AgdaOperator{\AgdaFunction{⊑ᴸᴿₜ}}\AgdaSpace{}%
\AgdaBound{M′}\AgdaSpace{}%
\AgdaOperator{\AgdaFunction{⦂}}\AgdaSpace{}%
\AgdaBound{A⊑A′}\AgdaSymbol{)}\AgdaSpace{}%
\AgdaSymbol{(}\AgdaInductiveConstructor{suc}\AgdaSpace{}%
\AgdaBound{k}\AgdaSymbol{)}\AgdaSpace{}%
\AgdaOperator{\AgdaFunction{⇔}}\AgdaSpace{}%
\AgdaField{\#}\AgdaSymbol{(}\AgdaFunction{LRₜ-def}\AgdaSpace{}%
\AgdaBound{A⊑A′}\AgdaSpace{}%
\AgdaBound{dir}\AgdaSpace{}%
\AgdaBound{M}\AgdaSpace{}%
\AgdaBound{M′}\AgdaSymbol{)}\AgdaSpace{}%
\AgdaSymbol{(}\AgdaInductiveConstructor{suc}\AgdaSpace{}%
\AgdaBound{k}\AgdaSymbol{)}\<%
\end{code}
\begin{code}[hide]%
\>[0]\AgdaFunction{LRₜ-suc}\AgdaSpace{}%
\AgdaSymbol{\{}\AgdaBound{A}\AgdaSymbol{\}\{}\AgdaBound{A′}\AgdaSymbol{\}\{}\AgdaBound{A⊑A′}\AgdaSymbol{\}\{}\AgdaBound{dir}\AgdaSymbol{\}\{}\AgdaBound{M}\AgdaSymbol{\}\{}\AgdaBound{M′}\AgdaSymbol{\}\{}\AgdaBound{k}\AgdaSymbol{\}}\AgdaSpace{}%
\AgdaSymbol{=}\<%
\\
\>[0][@{}l@{\AgdaIndent{0}}]%
\>[3]\AgdaFunction{≡ᵒ⇒⇔}\AgdaSymbol{\{}\AgdaArgument{k}\AgdaSpace{}%
\AgdaSymbol{=}\AgdaSpace{}%
\AgdaInductiveConstructor{suc}\AgdaSpace{}%
\AgdaBound{k}\AgdaSymbol{\}}\AgdaSpace{}%
\AgdaSymbol{(}\AgdaFunction{LRₜ-stmt}\AgdaSymbol{\{}\AgdaBound{A}\AgdaSymbol{\}\{}\AgdaBound{A′}\AgdaSymbol{\}\{}\AgdaBound{A⊑A′}\AgdaSymbol{\}\{}\AgdaBound{dir}\AgdaSymbol{\}\{}\AgdaBound{M}\AgdaSymbol{\}\{}\AgdaBound{M′}\AgdaSymbol{\})}\<%
\end{code}

The definition of ⊑ᴸᴿᵥ included several clauses that ensured that the
related values are indeed syntactic values. Here we make use of that
to prove that indeed, logically related values are syntactic values.

\begin{code}%
\>[0]\AgdaFunction{LRᵥ⇒Value}\AgdaSpace{}%
\AgdaSymbol{:}\AgdaSpace{}%
\AgdaSymbol{∀}\AgdaSpace{}%
\AgdaSymbol{\{}\AgdaBound{k}\AgdaSymbol{\}\{}\AgdaBound{dir}\AgdaSymbol{\}\{}\AgdaBound{A}\AgdaSymbol{\}\{}\AgdaBound{A′}\AgdaSymbol{\}}\AgdaSpace{}%
\AgdaSymbol{(}\AgdaBound{A⊑A′}\AgdaSpace{}%
\AgdaSymbol{:}\AgdaSpace{}%
\AgdaBound{A}\AgdaSpace{}%
\AgdaOperator{\AgdaDatatype{⊑}}\AgdaSpace{}%
\AgdaBound{A′}\AgdaSymbol{)}\AgdaSpace{}%
\AgdaBound{M}\AgdaSpace{}%
\AgdaBound{M′}\<%
\\
\>[0][@{}l@{\AgdaIndent{0}}]%
\>[3]\AgdaSymbol{→}\AgdaSpace{}%
\AgdaField{\#}\AgdaSpace{}%
\AgdaSymbol{(}\AgdaBound{dir}\AgdaSpace{}%
\AgdaOperator{\AgdaFunction{∣}}\AgdaSpace{}%
\AgdaBound{M}\AgdaSpace{}%
\AgdaOperator{\AgdaFunction{⊑ᴸᴿᵥ}}\AgdaSpace{}%
\AgdaBound{M′}\AgdaSpace{}%
\AgdaOperator{\AgdaFunction{⦂}}\AgdaSpace{}%
\AgdaBound{A⊑A′}\AgdaSymbol{)}\AgdaSpace{}%
\AgdaSymbol{(}\AgdaInductiveConstructor{suc}\AgdaSpace{}%
\AgdaBound{k}\AgdaSymbol{)}%
\>[41]\AgdaSymbol{→}%
\>[44]\AgdaDatatype{Value}\AgdaSpace{}%
\AgdaBound{M}\AgdaSpace{}%
\AgdaOperator{\AgdaFunction{×}}\AgdaSpace{}%
\AgdaDatatype{Value}\AgdaSpace{}%
\AgdaBound{M′}\<%
\end{code}
\begin{code}[hide]%
\>[0]\AgdaFunction{LRᵥ⇒Value}\AgdaSpace{}%
\AgdaSymbol{\{}\AgdaBound{k}\AgdaSymbol{\}\{}\AgdaBound{dir}\AgdaSymbol{\}}\AgdaSpace{}%
\AgdaInductiveConstructor{unk⊑unk}\AgdaSpace{}%
\AgdaSymbol{(}\AgdaBound{V}\AgdaSpace{}%
\AgdaOperator{\AgdaInductiveConstructor{⟨}}\AgdaSpace{}%
\AgdaBound{G}\AgdaSpace{}%
\AgdaOperator{\AgdaInductiveConstructor{!⟩}}\AgdaSymbol{)}\AgdaSpace{}%
\AgdaSymbol{(}\AgdaBound{V′}\AgdaSpace{}%
\AgdaOperator{\AgdaInductiveConstructor{⟨}}\AgdaSpace{}%
\AgdaBound{H}\AgdaSpace{}%
\AgdaOperator{\AgdaInductiveConstructor{!⟩}}\AgdaSymbol{)}\AgdaSpace{}%
\AgdaBound{𝒱MM′}\<%
\\
\>[0][@{}l@{\AgdaIndent{0}}]%
\>[4]\AgdaKeyword{with}\AgdaSpace{}%
\AgdaBound{G}\AgdaSpace{}%
\AgdaOperator{\AgdaFunction{≡ᵍ}}\AgdaSpace{}%
\AgdaBound{H}\<%
\\
\>[0]\AgdaSymbol{...}\AgdaSpace{}%
\AgdaSymbol{|}\AgdaSpace{}%
\AgdaInductiveConstructor{no}\AgdaSpace{}%
\AgdaBound{neq}\AgdaSpace{}%
\AgdaSymbol{=}\AgdaSpace{}%
\AgdaFunction{⊥-elim}\AgdaSpace{}%
\AgdaBound{𝒱MM′}\<%
\\
\>[0]\AgdaSymbol{...}%
\>[1838I]\AgdaSymbol{|}\AgdaSpace{}%
\AgdaInductiveConstructor{yes}\AgdaSpace{}%
\AgdaInductiveConstructor{refl}\<%
\\
\>[.][@{}l@{}]\<[1838I]%
\>[4]\AgdaKeyword{with}\AgdaSpace{}%
\AgdaBound{𝒱MM′}\<%
\\
\>[0]\AgdaSymbol{...}\AgdaSpace{}%
\AgdaSymbol{|}\AgdaSpace{}%
\AgdaBound{v}\AgdaSpace{}%
\AgdaOperator{\AgdaInductiveConstructor{,}}\AgdaSpace{}%
\AgdaBound{v′}\AgdaSpace{}%
\AgdaOperator{\AgdaInductiveConstructor{,}}\AgdaSpace{}%
\AgdaSymbol{\AgdaUnderscore{}}\AgdaSpace{}%
\AgdaSymbol{=}\AgdaSpace{}%
\AgdaSymbol{(}\AgdaBound{v}\AgdaSpace{}%
\AgdaOperator{\AgdaInductiveConstructor{〈}}\AgdaSpace{}%
\AgdaBound{G}\AgdaSpace{}%
\AgdaOperator{\AgdaInductiveConstructor{〉}}\AgdaSymbol{)}\AgdaSpace{}%
\AgdaOperator{\AgdaInductiveConstructor{,}}\AgdaSpace{}%
\AgdaSymbol{(}\AgdaBound{v′}\AgdaSpace{}%
\AgdaOperator{\AgdaInductiveConstructor{〈}}\AgdaSpace{}%
\AgdaBound{G}\AgdaSpace{}%
\AgdaOperator{\AgdaInductiveConstructor{〉}}\AgdaSymbol{)}\<%
\\
\>[0]\AgdaFunction{LRᵥ⇒Value}\AgdaSpace{}%
\AgdaSymbol{\{}\AgdaBound{k}\AgdaSymbol{\}\{}\AgdaInductiveConstructor{≼}\AgdaSymbol{\}}\AgdaSpace{}%
\AgdaSymbol{(}\AgdaInductiveConstructor{unk⊑}\AgdaSymbol{\{}\AgdaBound{H}\AgdaSymbol{\}\{}\AgdaBound{A′}\AgdaSymbol{\}}\AgdaSpace{}%
\AgdaBound{d}\AgdaSymbol{)}\AgdaSpace{}%
\AgdaSymbol{(}\AgdaBound{V}\AgdaSpace{}%
\AgdaOperator{\AgdaInductiveConstructor{⟨}}\AgdaSpace{}%
\AgdaBound{G}\AgdaSpace{}%
\AgdaOperator{\AgdaInductiveConstructor{!⟩}}\AgdaSymbol{)}\AgdaSpace{}%
\AgdaBound{V′}\AgdaSpace{}%
\AgdaBound{𝒱VGV′}\<%
\\
\>[0][@{}l@{\AgdaIndent{0}}]%
\>[4]\AgdaKeyword{with}\AgdaSpace{}%
\AgdaBound{G}\AgdaSpace{}%
\AgdaOperator{\AgdaFunction{≡ᵍ}}\AgdaSpace{}%
\AgdaBound{H}\<%
\\
\>[0]\AgdaSymbol{...}%
\>[1870I]\AgdaSymbol{|}\AgdaSpace{}%
\AgdaInductiveConstructor{yes}\AgdaSpace{}%
\AgdaInductiveConstructor{refl}\<%
\\
\>[.][@{}l@{}]\<[1870I]%
\>[4]\AgdaKeyword{with}\AgdaSpace{}%
\AgdaBound{𝒱VGV′}\<%
\\
\>[0]\AgdaSymbol{...}\AgdaSpace{}%
\AgdaSymbol{|}\AgdaSpace{}%
\AgdaBound{v}\AgdaSpace{}%
\AgdaOperator{\AgdaInductiveConstructor{,}}\AgdaSpace{}%
\AgdaBound{v′}\AgdaSpace{}%
\AgdaOperator{\AgdaInductiveConstructor{,}}\AgdaSpace{}%
\AgdaSymbol{\AgdaUnderscore{}}\AgdaSpace{}%
\AgdaSymbol{=}\AgdaSpace{}%
\AgdaSymbol{(}\AgdaBound{v}\AgdaSpace{}%
\AgdaOperator{\AgdaInductiveConstructor{〈}}\AgdaSpace{}%
\AgdaSymbol{\AgdaUnderscore{}}\AgdaSpace{}%
\AgdaOperator{\AgdaInductiveConstructor{〉}}\AgdaSymbol{)}\AgdaSpace{}%
\AgdaOperator{\AgdaInductiveConstructor{,}}\AgdaSpace{}%
\AgdaBound{v′}\<%
\\
\>[0]\AgdaFunction{LRᵥ⇒Value}\AgdaSpace{}%
\AgdaSymbol{\{}\AgdaBound{k}\AgdaSymbol{\}\{}\AgdaInductiveConstructor{≽}\AgdaSymbol{\}}\AgdaSpace{}%
\AgdaSymbol{(}\AgdaInductiveConstructor{unk⊑}\AgdaSymbol{\{}\AgdaBound{H}\AgdaSymbol{\}\{}\AgdaBound{A′}\AgdaSymbol{\}}\AgdaSpace{}%
\AgdaBound{d}\AgdaSymbol{)}\AgdaSpace{}%
\AgdaSymbol{(}\AgdaBound{V}\AgdaSpace{}%
\AgdaOperator{\AgdaInductiveConstructor{⟨}}\AgdaSpace{}%
\AgdaBound{G}\AgdaSpace{}%
\AgdaOperator{\AgdaInductiveConstructor{!⟩}}\AgdaSymbol{)}\AgdaSpace{}%
\AgdaBound{V′}\AgdaSpace{}%
\AgdaBound{𝒱VGV′}\<%
\\
\>[0][@{}l@{\AgdaIndent{0}}]%
\>[4]\AgdaKeyword{with}\AgdaSpace{}%
\AgdaBound{G}\AgdaSpace{}%
\AgdaOperator{\AgdaFunction{≡ᵍ}}\AgdaSpace{}%
\AgdaBound{H}\<%
\\
\>[0]\AgdaSymbol{...}%
\>[1899I]\AgdaSymbol{|}\AgdaSpace{}%
\AgdaInductiveConstructor{yes}\AgdaSpace{}%
\AgdaInductiveConstructor{refl}\<%
\\
\>[.][@{}l@{}]\<[1899I]%
\>[4]\AgdaKeyword{with}\AgdaSpace{}%
\AgdaBound{𝒱VGV′}\<%
\\
\>[0]\AgdaSymbol{...}\AgdaSpace{}%
\AgdaSymbol{|}\AgdaSpace{}%
\AgdaBound{v}\AgdaSpace{}%
\AgdaOperator{\AgdaInductiveConstructor{,}}\AgdaSpace{}%
\AgdaBound{v′}\AgdaSpace{}%
\AgdaOperator{\AgdaInductiveConstructor{,}}\AgdaSpace{}%
\AgdaSymbol{\AgdaUnderscore{}}\AgdaSpace{}%
\AgdaSymbol{=}\AgdaSpace{}%
\AgdaSymbol{(}\AgdaBound{v}\AgdaSpace{}%
\AgdaOperator{\AgdaInductiveConstructor{〈}}\AgdaSpace{}%
\AgdaSymbol{\AgdaUnderscore{}}\AgdaSpace{}%
\AgdaOperator{\AgdaInductiveConstructor{〉}}\AgdaSymbol{)}\AgdaSpace{}%
\AgdaOperator{\AgdaInductiveConstructor{,}}\AgdaSpace{}%
\AgdaBound{v′}\<%
\\
\>[0]\AgdaFunction{LRᵥ⇒Value}\AgdaSpace{}%
\AgdaSymbol{\{}\AgdaBound{k}\AgdaSymbol{\}\{}\AgdaBound{dir}\AgdaSymbol{\}}\AgdaSpace{}%
\AgdaSymbol{(}\AgdaInductiveConstructor{unk⊑}\AgdaSymbol{\{}\AgdaBound{H}\AgdaSymbol{\}\{}\AgdaBound{A′}\AgdaSymbol{\}}\AgdaSpace{}%
\AgdaBound{d}\AgdaSymbol{)}\AgdaSpace{}%
\AgdaSymbol{(}\AgdaBound{V}\AgdaSpace{}%
\AgdaOperator{\AgdaInductiveConstructor{⟨}}\AgdaSpace{}%
\AgdaBound{G}\AgdaSpace{}%
\AgdaOperator{\AgdaInductiveConstructor{!⟩}}\AgdaSymbol{)}\AgdaSpace{}%
\AgdaBound{V′}\AgdaSpace{}%
\AgdaBound{𝒱VGV′}\<%
\\
\>[0][@{}l@{\AgdaIndent{0}}]%
\>[4]\AgdaSymbol{|}\AgdaSpace{}%
\AgdaInductiveConstructor{no}\AgdaSpace{}%
\AgdaBound{neq}\AgdaSpace{}%
\AgdaSymbol{=}\AgdaSpace{}%
\AgdaFunction{⊥-elim}\AgdaSpace{}%
\AgdaBound{𝒱VGV′}\<%
\\
\>[0]\AgdaFunction{LRᵥ⇒Value}\AgdaSpace{}%
\AgdaSymbol{\{}\AgdaBound{k}\AgdaSymbol{\}\{}\AgdaBound{dir}\AgdaSymbol{\}}\AgdaSpace{}%
\AgdaSymbol{(}\AgdaInductiveConstructor{base⊑}\AgdaSymbol{\{}\AgdaBound{ι}\AgdaSymbol{\})}\AgdaSpace{}%
\AgdaSymbol{(}\AgdaInductiveConstructor{\$}\AgdaSpace{}%
\AgdaBound{c}\AgdaSymbol{)}\AgdaSpace{}%
\AgdaSymbol{(}\AgdaInductiveConstructor{\$}\AgdaSpace{}%
\AgdaBound{c′}\AgdaSymbol{)}\AgdaSpace{}%
\AgdaInductiveConstructor{refl}\AgdaSpace{}%
\AgdaSymbol{=}\AgdaSpace{}%
\AgdaSymbol{(}\AgdaInductiveConstructor{\$̬}\AgdaSpace{}%
\AgdaBound{c}\AgdaSymbol{)}\AgdaSpace{}%
\AgdaOperator{\AgdaInductiveConstructor{,}}\AgdaSpace{}%
\AgdaSymbol{(}\AgdaInductiveConstructor{\$̬}\AgdaSpace{}%
\AgdaBound{c}\AgdaSymbol{)}\<%
\\
\>[0]\AgdaFunction{LRᵥ⇒Value}\AgdaSpace{}%
\AgdaSymbol{\{}\AgdaBound{k}\AgdaSymbol{\}\{}\AgdaBound{dir}\AgdaSymbol{\}}\AgdaSpace{}%
\AgdaSymbol{(}\AgdaInductiveConstructor{fun⊑}\AgdaSpace{}%
\AgdaBound{A⊑A′}\AgdaSpace{}%
\AgdaBound{B⊑B′}\AgdaSymbol{)}\AgdaSpace{}%
\AgdaSymbol{(}\AgdaInductiveConstructor{ƛ}\AgdaSpace{}%
\AgdaBound{N}\AgdaSymbol{)}\AgdaSpace{}%
\AgdaSymbol{(}\AgdaInductiveConstructor{ƛ}\AgdaSpace{}%
\AgdaBound{N′}\AgdaSymbol{)}\AgdaSpace{}%
\AgdaBound{𝒱VV′}\AgdaSpace{}%
\AgdaSymbol{=}\<%
\\
\>[0][@{}l@{\AgdaIndent{0}}]%
\>[4]\AgdaSymbol{(}\AgdaOperator{\AgdaInductiveConstructor{ƛ̬}}\AgdaSpace{}%
\AgdaBound{N}\AgdaSymbol{)}\AgdaSpace{}%
\AgdaOperator{\AgdaInductiveConstructor{,}}\AgdaSpace{}%
\AgdaSymbol{(}\AgdaOperator{\AgdaInductiveConstructor{ƛ̬}}\AgdaSpace{}%
\AgdaBound{N′}\AgdaSymbol{)}\<%
\end{code}

If two values are related via ⊑ᴸᴿᵥ, then they are also related via
⊑ᴸᴿₜ.

\begin{code}[hide]%
\>[0]\AgdaFunction{LRᵥ⇒LRₜ-step}\AgdaSpace{}%
\AgdaSymbol{:}\AgdaSpace{}%
\AgdaSymbol{∀\{}\AgdaBound{A}\AgdaSymbol{\}\{}\AgdaBound{A′}\AgdaSymbol{\}\{}\AgdaBound{A⊑A′}\AgdaSpace{}%
\AgdaSymbol{:}\AgdaSpace{}%
\AgdaBound{A}\AgdaSpace{}%
\AgdaOperator{\AgdaDatatype{⊑}}\AgdaSpace{}%
\AgdaBound{A′}\AgdaSymbol{\}\{}\AgdaBound{V}\AgdaSpace{}%
\AgdaBound{V′}\AgdaSymbol{\}\{}\AgdaBound{dir}\AgdaSymbol{\}\{}\AgdaBound{k}\AgdaSymbol{\}}\<%
\\
\>[0][@{}l@{\AgdaIndent{0}}]%
\>[3]\AgdaSymbol{→}\AgdaSpace{}%
\AgdaField{\#}\AgdaSymbol{(}\AgdaBound{dir}\AgdaSpace{}%
\AgdaOperator{\AgdaFunction{∣}}\AgdaSpace{}%
\AgdaBound{V}\AgdaSpace{}%
\AgdaOperator{\AgdaFunction{⊑ᴸᴿᵥ}}\AgdaSpace{}%
\AgdaBound{V′}\AgdaSpace{}%
\AgdaOperator{\AgdaFunction{⦂}}\AgdaSpace{}%
\AgdaBound{A⊑A′}\AgdaSymbol{)}\AgdaSpace{}%
\AgdaBound{k}%
\>[34]\AgdaSymbol{→}%
\>[37]\AgdaField{\#}\AgdaSymbol{(}\AgdaBound{dir}\AgdaSpace{}%
\AgdaOperator{\AgdaFunction{∣}}\AgdaSpace{}%
\AgdaBound{V}\AgdaSpace{}%
\AgdaOperator{\AgdaFunction{⊑ᴸᴿₜ}}\AgdaSpace{}%
\AgdaBound{V′}\AgdaSpace{}%
\AgdaOperator{\AgdaFunction{⦂}}\AgdaSpace{}%
\AgdaBound{A⊑A′}\AgdaSymbol{)}\AgdaSpace{}%
\AgdaBound{k}\<%
\\
\>[0]\AgdaFunction{LRᵥ⇒LRₜ-step}\AgdaSpace{}%
\AgdaSymbol{\{}\AgdaBound{A}\AgdaSymbol{\}\{}\AgdaBound{A′}\AgdaSymbol{\}\{}\AgdaBound{A⊑A′}\AgdaSymbol{\}\{}\AgdaBound{V}\AgdaSymbol{\}}\AgdaSpace{}%
\AgdaSymbol{\{}\AgdaBound{V′}\AgdaSymbol{\}}\AgdaSpace{}%
\AgdaSymbol{\{}\AgdaBound{dir}\AgdaSymbol{\}}\AgdaSpace{}%
\AgdaSymbol{\{}\AgdaInductiveConstructor{zero}\AgdaSymbol{\}}\AgdaSpace{}%
\AgdaBound{𝒱VV′k}\AgdaSpace{}%
\AgdaSymbol{=}\<%
\\
\>[0][@{}l@{\AgdaIndent{0}}]%
\>[3]\AgdaField{tz}\AgdaSpace{}%
\AgdaSymbol{(}\AgdaBound{dir}\AgdaSpace{}%
\AgdaOperator{\AgdaFunction{∣}}\AgdaSpace{}%
\AgdaBound{V}\AgdaSpace{}%
\AgdaOperator{\AgdaFunction{⊑ᴸᴿₜ}}\AgdaSpace{}%
\AgdaBound{V′}\AgdaSpace{}%
\AgdaOperator{\AgdaFunction{⦂}}\AgdaSpace{}%
\AgdaBound{A⊑A′}\AgdaSymbol{)}\<%
\\
\>[0]\AgdaFunction{LRᵥ⇒LRₜ-step}\AgdaSpace{}%
\AgdaSymbol{\{}\AgdaBound{A}\AgdaSymbol{\}\{}\AgdaBound{A′}\AgdaSymbol{\}\{}\AgdaBound{A⊑A′}\AgdaSymbol{\}\{}\AgdaBound{V}\AgdaSymbol{\}}\AgdaSpace{}%
\AgdaSymbol{\{}\AgdaBound{V′}\AgdaSymbol{\}}\AgdaSpace{}%
\AgdaSymbol{\{}\AgdaInductiveConstructor{≼}\AgdaSymbol{\}}\AgdaSpace{}%
\AgdaSymbol{\{}\AgdaInductiveConstructor{suc}\AgdaSpace{}%
\AgdaBound{k}\AgdaSymbol{\}}\AgdaSpace{}%
\AgdaBound{𝒱VV′sk}\AgdaSpace{}%
\AgdaSymbol{=}\<%
\\
\>[0][@{}l@{\AgdaIndent{0}}]%
\>[2]\AgdaFunction{⇔-fro}\AgdaSpace{}%
\AgdaSymbol{(}\AgdaFunction{LRₜ-suc}\AgdaSymbol{\{}\AgdaArgument{dir}\AgdaSpace{}%
\AgdaSymbol{=}\AgdaSpace{}%
\AgdaInductiveConstructor{≼}\AgdaSymbol{\})}\<%
\\
%
\>[2]\AgdaSymbol{(}\AgdaKeyword{let}\AgdaSpace{}%
\AgdaSymbol{(}\AgdaBound{v}\AgdaSpace{}%
\AgdaOperator{\AgdaInductiveConstructor{,}}\AgdaSpace{}%
\AgdaBound{v′}\AgdaSymbol{)}\AgdaSpace{}%
\AgdaSymbol{=}\AgdaSpace{}%
\AgdaFunction{LRᵥ⇒Value}\AgdaSpace{}%
\AgdaBound{A⊑A′}\AgdaSpace{}%
\AgdaBound{V}\AgdaSpace{}%
\AgdaBound{V′}\AgdaSpace{}%
\AgdaBound{𝒱VV′sk}\AgdaSpace{}%
\AgdaKeyword{in}\<%
\\
%
\>[2]\AgdaSymbol{(}\AgdaInductiveConstructor{inj₂}\AgdaSpace{}%
\AgdaSymbol{(}\AgdaInductiveConstructor{inj₂}\AgdaSpace{}%
\AgdaSymbol{(}\AgdaBound{v}\AgdaSpace{}%
\AgdaOperator{\AgdaInductiveConstructor{,}}\AgdaSpace{}%
\AgdaSymbol{(}\AgdaBound{V′}\AgdaSpace{}%
\AgdaOperator{\AgdaInductiveConstructor{,}}\AgdaSpace{}%
\AgdaSymbol{(}\AgdaBound{V′}\AgdaSpace{}%
\AgdaOperator{\AgdaInductiveConstructor{END}}\AgdaSymbol{)}\AgdaSpace{}%
\AgdaOperator{\AgdaInductiveConstructor{,}}\AgdaSpace{}%
\AgdaBound{v′}\AgdaSpace{}%
\AgdaOperator{\AgdaInductiveConstructor{,}}\AgdaSpace{}%
\AgdaBound{𝒱VV′sk}\AgdaSymbol{)))))}\<%
\\
\>[0]\AgdaFunction{LRᵥ⇒LRₜ-step}\AgdaSpace{}%
\AgdaSymbol{\{}\AgdaBound{A}\AgdaSymbol{\}\{}\AgdaBound{A′}\AgdaSymbol{\}\{}\AgdaBound{A⊑A′}\AgdaSymbol{\}\{}\AgdaBound{V}\AgdaSymbol{\}}\AgdaSpace{}%
\AgdaSymbol{\{}\AgdaBound{V′}\AgdaSymbol{\}}\AgdaSpace{}%
\AgdaSymbol{\{}\AgdaInductiveConstructor{≽}\AgdaSymbol{\}}\AgdaSpace{}%
\AgdaSymbol{\{}\AgdaInductiveConstructor{suc}\AgdaSpace{}%
\AgdaBound{k}\AgdaSymbol{\}}\AgdaSpace{}%
\AgdaBound{𝒱VV′sk}\AgdaSpace{}%
\AgdaSymbol{=}\<%
\\
\>[0][@{}l@{\AgdaIndent{0}}]%
\>[2]\AgdaFunction{⇔-fro}\AgdaSpace{}%
\AgdaSymbol{(}\AgdaFunction{LRₜ-suc}\AgdaSymbol{\{}\AgdaArgument{dir}\AgdaSpace{}%
\AgdaSymbol{=}\AgdaSpace{}%
\AgdaInductiveConstructor{≽}\AgdaSymbol{\})}\<%
\\
%
\>[2]\AgdaSymbol{(}\AgdaKeyword{let}\AgdaSpace{}%
\AgdaSymbol{(}\AgdaBound{v}\AgdaSpace{}%
\AgdaOperator{\AgdaInductiveConstructor{,}}\AgdaSpace{}%
\AgdaBound{v′}\AgdaSymbol{)}\AgdaSpace{}%
\AgdaSymbol{=}\AgdaSpace{}%
\AgdaFunction{LRᵥ⇒Value}\AgdaSpace{}%
\AgdaBound{A⊑A′}\AgdaSpace{}%
\AgdaBound{V}\AgdaSpace{}%
\AgdaBound{V′}\AgdaSpace{}%
\AgdaBound{𝒱VV′sk}\AgdaSpace{}%
\AgdaKeyword{in}\<%
\\
%
\>[2]\AgdaInductiveConstructor{inj₂}\AgdaSpace{}%
\AgdaSymbol{(}\AgdaInductiveConstructor{inj₂}\AgdaSpace{}%
\AgdaSymbol{(}\AgdaBound{v′}\AgdaSpace{}%
\AgdaOperator{\AgdaInductiveConstructor{,}}\AgdaSpace{}%
\AgdaBound{V}\AgdaSpace{}%
\AgdaOperator{\AgdaInductiveConstructor{,}}\AgdaSpace{}%
\AgdaSymbol{(}\AgdaBound{V}\AgdaSpace{}%
\AgdaOperator{\AgdaInductiveConstructor{END}}\AgdaSymbol{)}\AgdaSpace{}%
\AgdaOperator{\AgdaInductiveConstructor{,}}\AgdaSpace{}%
\AgdaBound{v}\AgdaSpace{}%
\AgdaOperator{\AgdaInductiveConstructor{,}}\AgdaSpace{}%
\AgdaBound{𝒱VV′sk}\AgdaSymbol{)))}\<%
\end{code}
\begin{code}%
\>[0]\AgdaFunction{LRᵥ⇒LRₜ}\AgdaSpace{}%
\AgdaSymbol{:}\AgdaSpace{}%
\AgdaSymbol{∀\{}\AgdaBound{A}\AgdaSymbol{\}\{}\AgdaBound{A′}\AgdaSymbol{\}\{}\AgdaBound{A⊑A′}\AgdaSpace{}%
\AgdaSymbol{:}\AgdaSpace{}%
\AgdaBound{A}\AgdaSpace{}%
\AgdaOperator{\AgdaDatatype{⊑}}\AgdaSpace{}%
\AgdaBound{A′}\AgdaSymbol{\}\{}\AgdaBound{𝒫}\AgdaSymbol{\}\{}\AgdaBound{V}\AgdaSpace{}%
\AgdaBound{V′}\AgdaSymbol{\}\{}\AgdaBound{dir}\AgdaSymbol{\}}\<%
\\
\>[0][@{}l@{\AgdaIndent{0}}]%
\>[3]\AgdaSymbol{→}\AgdaSpace{}%
\AgdaBound{𝒫}\AgdaSpace{}%
\AgdaOperator{\AgdaFunction{⊢ᵒ}}\AgdaSpace{}%
\AgdaBound{dir}\AgdaSpace{}%
\AgdaOperator{\AgdaFunction{∣}}\AgdaSpace{}%
\AgdaBound{V}\AgdaSpace{}%
\AgdaOperator{\AgdaFunction{⊑ᴸᴿᵥ}}\AgdaSpace{}%
\AgdaBound{V′}\AgdaSpace{}%
\AgdaOperator{\AgdaFunction{⦂}}\AgdaSpace{}%
\AgdaBound{A⊑A′}%
\>[34]\AgdaSymbol{→}%
\>[37]\AgdaBound{𝒫}\AgdaSpace{}%
\AgdaOperator{\AgdaFunction{⊢ᵒ}}\AgdaSpace{}%
\AgdaBound{dir}\AgdaSpace{}%
\AgdaOperator{\AgdaFunction{∣}}\AgdaSpace{}%
\AgdaBound{V}\AgdaSpace{}%
\AgdaOperator{\AgdaFunction{⊑ᴸᴿₜ}}\AgdaSpace{}%
\AgdaBound{V′}\AgdaSpace{}%
\AgdaOperator{\AgdaFunction{⦂}}\AgdaSpace{}%
\AgdaBound{A⊑A′}\<%
\end{code}
\begin{code}[hide]%
\>[0]\AgdaFunction{LRᵥ⇒LRₜ}\AgdaSpace{}%
\AgdaSymbol{\{}\AgdaBound{A}\AgdaSymbol{\}\{}\AgdaBound{A′}\AgdaSymbol{\}\{}\AgdaBound{A⊑A′}\AgdaSymbol{\}\{}\AgdaBound{𝒫}\AgdaSymbol{\}\{}\AgdaBound{V}\AgdaSymbol{\}\{}\AgdaBound{V′}\AgdaSymbol{\}\{}\AgdaBound{dir}\AgdaSymbol{\}}\AgdaSpace{}%
\AgdaBound{⊢𝒱VV′}\AgdaSpace{}%
\AgdaSymbol{=}\AgdaSpace{}%
\AgdaFunction{⊢ᵒ-intro}\AgdaSpace{}%
\AgdaSymbol{λ}\AgdaSpace{}%
\AgdaBound{k}\AgdaSpace{}%
\AgdaBound{𝒫k}\AgdaSpace{}%
\AgdaSymbol{→}\<%
\\
\>[0][@{}l@{\AgdaIndent{0}}]%
\>[2]\AgdaFunction{LRᵥ⇒LRₜ-step}\AgdaSymbol{\{}\AgdaArgument{V}\AgdaSpace{}%
\AgdaSymbol{=}\AgdaSpace{}%
\AgdaBound{V}\AgdaSymbol{\}\{}\AgdaBound{V′}\AgdaSymbol{\}\{}\AgdaBound{dir}\AgdaSymbol{\}\{}\AgdaBound{k}\AgdaSymbol{\}}\AgdaSpace{}%
\AgdaSymbol{(}\AgdaFunction{⊢ᵒ-elim}\AgdaSpace{}%
\AgdaBound{⊢𝒱VV′}\AgdaSpace{}%
\AgdaBound{k}\AgdaSpace{}%
\AgdaBound{𝒫k}\AgdaSymbol{)}\<%
\end{code}


\input{PeterFundamental}
\input{PeterGG}

\section{Conclusion and Acknowledgments}
\label{sec:conclusion}

This paper presented the first mechanized proof of the gradual
guarantee using step-indexed logical relations. One naturally wonders
how using step-indexed logical relations compares to a
simulation-based proof. One rough comparison is the number of lines of
code in Agda. Wadler, Thiemann, and I developed a simulation-based
proof of the gradual guarantee for a similar cast calculus in Agda,
which came in at 3,200 LOC. The logical-relations proof presented here
was significantly shorter, at 2,300 LOC, though it makes use of the
Abstract Binding Tree Library (900 LOC) and the Step-Indexed Logic
Library (2100 LOC). These LOC numbers confirm my feeling that the
total effort to create the SIL and ABT libraries and prove the gradual
guarantee via logical relations was higher than to prove the gradual
guarantee via simulation. However, if one discounts the SIL and ABT
libraries because they are reusable and language independent, then the
remaining effort to prove the gradual guarantee via logical relations
was lower than via simulation.

As mentioned at various points in this paper, there are some rough
edges to the SIL and ABT libraries, primarily due to challenges
regarding ``leaky abstractions''. For SIL, we mentioned how Agda's
output shows normalized versions of the SIL formulas, which exposes
the underlying encodings and are too large to be readable. We have
created a new version of SIL that uses Agda's \texttt{abstract}
feature and look forward to updating the proof of the gradual
guarantee to use the new version of SIL. Regarding the ABT library,
there are also challenges regarding (1) Agda output not always using
the concise pattern syntax and (2) Agda's automated case splitting
does not work for ABT-generated languages.

This work was conducted in the context of a collaboration with Peter
Thiemann and Philip Wadler where we have been exploring how to
mechanize blame calculi in Agda and study the polymorphic blame
calculus. My understanding of step-indexed logical relations was
improved by reading Peter Thiemann's proof in Agda of type safety for
a typed $\lambda$-calculus with \textsf{fix} using step-indexed
logical relations. The initiative to build an Agda version of the LSLR
logic came from Philip Wadler.


%\nocite{*}
\bibliographystyle{eptcs}
\bibliography{all}
\end{document}

% LocalWords:  Siek al GTLC Isabelle Agda Ahmed parametricity Coq de
% LocalWords:  metatheory Bruijn Dreyer Birkedal LSLR SIL Wadler LOC
% LocalWords:  Thiemann ABT Agda's Thiemann's
